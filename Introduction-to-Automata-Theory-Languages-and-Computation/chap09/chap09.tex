% !TeX encoding = UTF-8
% !TeX program = XeLaTeX
% !TeX spellcheck = LaTeX

\documentclass[a4paper]{article}

\usepackage{amsmath,amsfonts,amssymb}
\usepackage{mathrsfs}
\usepackage{bm}
\usepackage{extarrows}
\usepackage{geometry}
\usepackage{ntheorem}
\usepackage{hyperref}
\usepackage[ruled]{algorithm2e}
\usepackage{caption,subcaption}
\usepackage{tikz}

\usetikzlibrary{automata}

\geometry{left=2cm,right=2cm,top=2cm,bottom=2cm}

\def\UrlBreaks{\do\A\do\B\do\C\do\D\do\E\do\F\do\G\do\H\do\I\do\J\do\K\do\L\do\M\do\N\do\O\do\P\do\Q\do\R\do\S\do\T\do\U\do\V\do\W\do\X\do\Y\do\Z\do\[\do\\\do\]\do\^\do\_\do\`\do\a\do\b\do\c\do\d\do\e\do\f\do\g\do\h\do\i\do\j\do\k\do\l\do\m\do\n\do\o\do\p\do\q\do\r\do\s\do\t\do\u\do\v\do\w\do\x\do\y\do\z\do\0\do\1\do\2\do\3\do\4\do\5\do\6\do\7\do\8\do\9\do\.\do\@\do\\\do\/\do\!\do\_\do\|\do\;\do\>\do\]\do\)\do\,\do\?\do\'\do+\do\=\do\#}

\newtheorem{theorem}{Theorem}
\newtheorem{lemma}{Lemma}
\newtheorem{proposition}{Proposition}
\newtheorem{corollary}{Corollary}
\newtheorem{claim}{Claim}
\newtheorem{conjecture}{conjecture}
\newtheorem{definition}{Definition}
\newtheorem{construction}{Construction}
\newtheorem*{proof}{Proof}
\newtheorem*{answer}{Answer}
\newtheorem*{example}{Example}
\newtheorem*{counterexample}{Counterexample}

\newenvironment{exercise}[1]{
	\par
	\noindent\textbf{Exercise #1.}\quad
}{
	\par
	\bigskip
}
\newenvironment{problem}[1]{
	\par
	\noindent\textbf{Problem #1.}\quad
}{
	\par
	\bigskip
}

\DeclareMathAccent{\widehat}{\mathord}{largesymbols}{"62}
\DeclareMathOperator*{\argmax}{\arg\,\max}
\DeclareMathOperator*{\argmin}{\arg\,\min}
\DeclareMathOperator{\E}{\mathbb E}
\DeclareMathOperator{\Var}{\mathrm{Var}}
\DeclareMathOperator{\tr}{\mathrm{tr}}
\DeclareMathOperator{\poly}{\mathrm{poly}}
\newcommand{\abs}[1]{\left| #1 \right|}
\newcommand{\vabs}[1]{\left\| #1 \right\|}
\newcommand{\abra}[1]{\left\langle #1 \right\rangle}
\newcommand{\pbra}[1]{\left( #1 \right)}
\newcommand{\cbra}[1]{\left\{ #1 \right\}}
\newcommand{\sbra}[1]{\left[ #1 \right]}
\newcommand{\floorbra}[1]{\left\lfloor #1 \right\rfloor}
\newcommand{\ceilbra}[1]{\left\lceil #1 \right\rceil}
\newcommand{\bin}{\{0,1\}}
\newcommand{\ZPP}{\mathtt{ZPP}}
\newcommand{\RP}{\mathtt{RP}}
\newcommand{\coRP}{\mathtt{co}\text{-}\mathtt{RP}}
\newcommand{\per}{\text{per}}
\newcommand{\sgn}{\text{sgn}}
\newcommand{\Fbb}{\mathbb{F}}
\newcommand{\Nbb}{\mathbb{N}}
\newcommand{\Rbb}{\mathbb{R}}
\newcommand{\Zbb}{\mathbb{Z}}
\newcommand{\Sset}{\mathbb{S}}
\newcommand{\Fset}{\mathbb{F}}
\newcommand{\Nset}{\mathbb{N}}
\newcommand{\Zset}{\mathbb{Z}}
\newcommand{\Uset}{\mathbb{U}}
\newcommand{\Acal}{\mathcal{A}}
\newcommand{\Bcal}{\mathcal{B}}
\newcommand{\Ccal}{\mathcal{C}}
\newcommand{\Fcal}{\mathcal{F}}
\newcommand{\Gcal}{\mathcal{G}}
\newcommand{\qd}[2]{{\left(\frac{#1}{#2}\right)}}
\newcommand{\dv}{\ |\ }
\newcommand{\mylm}{\xLongrightarrow[lm]{}}
\newcommand{\myrm}{\xLongrightarrow[rm]{}}

\bibliographystyle{plainnat}

\title{Exercise Set --- Chapter $9$}
\date{}

\begin{document}

\maketitle

\begin{exercise}{9.2.4} 
        Since $L_i$ is RE, let $M_i$ be the TM accepting $L_i$.
        Design $k$-tape TM $\overline{M}_s$ for $L_s$:
        \begin{enumerate}
            \item Copy the input string $w$ to every tape
            \item Simulate $M_i$ on the $i$-th tape simultaneously
            \item Accept if $M_s$ on the $s$-th tape accepts;
                reject if $M_i,i\neq s$ on the $i$-th tape accepts
        \end{enumerate}
        Since $L_1\cup L_2\cup\cdots\cup L_k=\Sigma^*$, $w$ must be in $L_t$, therefore
        $M_t$ will terminate, thus $\overline{M}_s$ always halts, indicating $L_s$ is recursive.
\end{exercise}

\begin{exercise}{9.2.5} 
    Both $L'$ and $\overline{L}'$ are non-RE.
    \begin{proof}\hspace{0pt}\\
        \noindent \textbf{$L'$ is non-RE.}\par
        Assume $L'$ is RE and TM $M$ accepts it. Construct $\overline{M}$ for $\overline{L}$:
        \begin{enumerate}
            \item Modify the input string $w$ into $1w$
            \item Simulate $M$ on the new input
            \item Accept if $M$ accepts; reject otherwise
        \end{enumerate}
        Therefore $\overline{L}$ will be RE, a contradiction.\\
        \noindent \textbf{$\overline{L}'$ is non-RE.}\par
        Assume $\overline{L}'=\{0w|w\notin L\}\cup\{1w|w\in L\}$ is RE and TM $M$ accepts it.
        Construct $\overline{M}$ for $\overline{L}$:
        \begin{enumerate}
            \item Modify the input string $w$ into $0w$
            \item Simulate $M$ on the new input
            \item Accept if $M$ accepts; reject otherwise
        \end{enumerate}
        Therefore $\overline{L}$ will be RE, a contradiction.
    \end{proof}
\end{exercise}

\begin{exercise}{9.2.6} \hspace{0pt}\\
\textbf{d)} Both recursive languages and RE are closed under Kleene closure.
    \begin{proof}\hspace{0pt}\\
        \textbf{Recursive Language:}\par
        Assume $L$ is recursive and TM $M$ accepts it. Construct $3$-tape TM $N$ for $L^*$:
        \begin{enumerate}
            \item Enumerate all possible dividing ways for input string $w$ on second tape
            \item For $i$-th way, start from round $1$
            \item In $j$-th round of $i$-th way, copy the $j$-th piece to third tape
                based on the $i$-th dividing way
            \item Simulate $M$ on the third tape
            \item If all pieces in $i$-th way are done,
                \begin{itemize}
                    \item $M$ accepted all pieces in the $i$-th way, then accept
                    \item otherwise, try next dividing way in \textit{Step $2$}
                \end{itemize}
                otherwise enter next round in \textit{Step $3$}
            \item Reject
        \end{enumerate}
        Since $L$ is recursive, every piece check will halt, thus $N$ always halts.\\
        \textbf{RE:}\par
        Assume $L$ is RE and TM $M$ accepts it. Construct $3$-tape NTM $N$ for $L^*$:
        \begin{enumerate}
            \item Guess the dividing way for input string $w$ on the second tape
            \item In $i$-th round, copy the $i$-th dividing piece to third tape
            \item Simulate $M$ on the third tape
            \item If all pieces are done,
                \begin{itemize}
                    \item $M$ accepted all pieces, then accept
                    \item otherwise, reject
                \end{itemize}
                otherwise enter next round in \textit{Step $2$}
        \end{enumerate}
    \end{proof}
\textbf{e)} RE is closed under homomorphism, while recursive languages are not.
    \begin{proof} Let $\sigma$ be the homomorphism.\\
        \textbf{RE:}\par
        Assume $L$ is RE, TM $M$ accepts it, and $L'$ is $\sigma(L)$.
        Construct $3$-tape NTM $N$ for $L'$:
        \begin{enumerate}
            \item Guess the original string $w'$ for input string $w$ on the second tape and copy it to third tape
            \item Check whether $\sigma(w')=w$ on second tape
            \item Simulate $M$ on third tape
            \item Accept if $M$ accepts and the check is passed; reject otherwise
        \end{enumerate}
        \textbf{Recursive Language:}\par
        Let $L=\{(M,w,2^k)|M\text{ accepts }w\text{ in $k$ steps}\}$, where $M$ is the binary representation of a TM
        and $w$ is a binary string. Let $\sigma(2)=\varepsilon$, then $\sigma(L)=\{(M,w,\varepsilon)|M\text{ accepts }w\}$.
        Since $L_u$ is not recursive language, $\sigma(L)$ is not either.
    \end{proof}
\textbf{f)} Both RE and recursive languages are closed under inverse homomorphism.
    \begin{proof} Let $\sigma$ be the homomorphism.\\
        \textbf{Recursive Language:}\par
        Assume $\sigma(L)$ is RE, TM $M$ accepts $\sigma(L)$. Construct TM $N$ for $L$:
        \begin{enumerate}
            \item Convert the input string $w$ into $\sigma(w)$
            \item Simulate $M$ on $\sigma(w)$
            \item Accept if $M$ accepts; reject otherwise
        \end{enumerate}
        \textbf{RE:} Basically the same construction.
    \end{proof}
\end{exercise}

\begin{exercise}{9.3.5} $L=\big\{(M_1,M_2,k)\big|\ |L(M_1)\cap L(M_2)|\geqslant k\big\}$ is RE, but not recursive.
    \begin{proof}
        To show it is RE, construct a $4$-tape NTM $M$ for $L$:
        \begin{enumerate}
            \item Guess $k$ strings on the second tape
            \item In $i$-th round, copy the $i$-th string to third and fourth tape
            \item Simulate $M_1$ on the third tape
            \item Simulate $M_2$ on the fourth tape
            \item If all strings are done,
                \begin{itemize}
                    \item $M_1,M_2$ accepted all $k$ strings, then accept
                    \item otherwise, reject
                \end{itemize}
                otherwise enter next round in \textit{Step $2$}
        \end{enumerate}
        Assume it is recursive and $N$ accepts it.
        Let $M_{all}$ be the TM accepts $\Sigma^*$. Then construct $M_{ne}$ for $L_{ne}$:
        \begin{enumerate}
            \item Modify the input $M$ into $(M,M_{all},1)$
            \item Simulate $N$ on the new input
            \item Accept if $N$ accepts; reject otherwise
        \end{enumerate}
        Since $L(M_{all})=\Sigma^*$, $|L(M)\cap L(M_{all})|\geqslant 1$ is tantamount to $L(M)$ being non-empty,
        which is $M\in L_{ne}$. Since $M_{ne}$ always halts, $L_{ne}$ is recursive, a contradiction.
    \end{proof}
\end{exercise}

\begin{exercise}{9.3.7} \hspace{0pt}\\
\textbf{a)} $L=\big\{(M,w)\big|\ M\text{ started with }w\text{ does not halt}\big\}$ is not RE.
    \begin{proof}
        Assume $L$ is RE and TM $N$ accepts it. Construct TM $N_d$ for $L_d$:
        \begin{enumerate}
            \item Modify the input $M$ into $(M,w_M)$, where $w_M$ is the binary representation of TM $M$
            \item Simulate $N$ on the new input
            \item Reject if $N$ accepts
            \item If $N$ rejects, it means $M$ will terminate with input $w_M$; so
                simulate $M$ on $w_M$
            \item Accept if $M$ rejects; reject otherwise
        \end{enumerate}
        Then $L_d$ is RE, a contradiction.
    \end{proof}
\textbf{b)} $L=\big\{(M_1,M_2)\big|\ L(M_1)\cap L(M_2)=\varnothing\big\}$ is not RE.
    \begin{proof}
        Assume $L$ is RE and TM $N$ accepts it. Let $M_{all}$ be the TM accepts $\Sigma^*$.
        Then construct $N_{e}$ for $L_e$:
        \begin{enumerate}
            \item Modify the input $M$ into $(M,M_{all})$
            \item Simulate $N$ on the new input
            \item Accept if $N$ accepts; reject otherwise
        \end{enumerate}
        Then $L_e$ is RE, a contradiction.
    \end{proof}
\textbf{c)} $L=\big\{(M_1,M_2,M_3)\big|L(M_1)=L(M_2)L(M_3)\big\}$ is not RE.
    \begin{proof}
        Assume $L$ is RE and TM $N$ accepts it. Let $M_{empty}$ be the TM accepts $\varnothing$.
        Then construct $N_{e}$ for $L_e$:
        \begin{enumerate}
            \item Modify the input $M$ into $(M,M_{empty},M_{empty})$
            \item Simulate $N$ on the new input
            \item Accept if $N$ accepts; reject otherwise
        \end{enumerate}
        Then $L_e$ is RE, a contradiction.
    \end{proof}
\end{exercise}

\begin{exercise}{9.4.2} PCP is undecidable if $\Sigma=\{0,1\}$.
    \begin{proof}
        Let PCP2 be the PCP with $\Sigma=\{0,1\}$. Assume PCP2 is decidable.
        For any PCP $T$, let $\Sigma'=\{c_1,c_2,\cdots,c_n\}$ be its character set.
        Define mapping $\sigma:\Sigma'\to\Sigma^*$ as follows
        $$
        \sigma(c_k)=\underbrace{0\cdots 0}_{k-1}1\underbrace{0\cdots 0}_{n-k}.
        $$
        Translate every string $w$ into $\sigma(w)$, then we have a PCP2 $T'$.\par
        Since every $\sigma(c_k)$ has same length, the interpretation is unique. This
        reduction implies the decidability of PCP, a contradiction.
    \end{proof}
\end{exercise}

\begin{exercise}{9.5.2} \hspace{0pt}\\
    \textbf{Part $1$:}\par
    Let $L$ be $\overline{L_A}\cup\overline{L_B}$.
    Then $L$ is regular language iff $L=(\Sigma\cup I)^*$.
    \begin{proof}\hspace{0pt}\\
        $\Leftarrow$: Obvious.\\
        $\Rightarrow$: \par
        Assume $L$ is regular language but not $(\Sigma\cup I)^*$.
        Then $\overline{L}=L_A\cap L_B$ is regular but not empty,
        which means there must exists a solution to
        PCP, say $wx\in\overline{L}$, where $w$ is the string from $\Sigma^*$ and $x$ is the corresponding index string.
        Therefore repeat the construction, $w^kx^k,k\in\Nset_+$ are also solutions.\par
        Since $L'=\{w^ix^j|i,j\in\Nset_+\}$ is regular, $L'\cap\overline{L}=\{w^kx^k|k\in\Nset_+\}$ is also
        regular, a contradiction.
    \end{proof}
    \textbf{Part $2$:}\par
    It is undecidable whether a CFG generates a regular language.
    \begin{proof}
        Assume it is decidable, it can be employed to check whether $\overline{L_A}\cup\overline{L_B}$,
        which is a CFL, is regular. However, \textbf{Part $1$} showed that it is equivalent to PCP. Thus a contradiction.
    \end{proof}
\end{exercise}

\end{document}
