% !TeX encoding = UTF-8
% !TeX program = XeLaTeX
% !TeX spellcheck = LaTeX

\documentclass[a4paper]{article}

\usepackage{amsmath,amsfonts,amssymb}
\usepackage{mathrsfs}
\usepackage{bm}
\usepackage{extarrows}
\usepackage{geometry}
\usepackage{ntheorem}
\usepackage{hyperref}
\usepackage[ruled]{algorithm2e}
\usepackage{caption,subcaption}
\usepackage{tikz}

\usetikzlibrary{automata}

\geometry{left=2cm,right=2cm,top=2cm,bottom=2cm}

\def\UrlBreaks{\do\A\do\B\do\C\do\D\do\E\do\F\do\G\do\H\do\I\do\J\do\K\do\L\do\M\do\N\do\O\do\P\do\Q\do\R\do\S\do\T\do\U\do\V\do\W\do\X\do\Y\do\Z\do\[\do\\\do\]\do\^\do\_\do\`\do\a\do\b\do\c\do\d\do\e\do\f\do\g\do\h\do\i\do\j\do\k\do\l\do\m\do\n\do\o\do\p\do\q\do\r\do\s\do\t\do\u\do\v\do\w\do\x\do\y\do\z\do\0\do\1\do\2\do\3\do\4\do\5\do\6\do\7\do\8\do\9\do\.\do\@\do\\\do\/\do\!\do\_\do\|\do\;\do\>\do\]\do\)\do\,\do\?\do\'\do+\do\=\do\#}

\newtheorem{theorem}{Theorem}
\newtheorem{lemma}{Lemma}
\newtheorem{proposition}{Proposition}
\newtheorem{corollary}{Corollary}
\newtheorem{claim}{Claim}
\newtheorem{conjecture}{conjecture}
\newtheorem{definition}{Definition}
\newtheorem{construction}{Construction}
\newtheorem*{proof}{Proof}
\newtheorem*{answer}{Answer}
\newtheorem*{example}{Example}
\newtheorem*{counterexample}{Counterexample}

\newenvironment{exercise}[1]{
	\par
	\noindent\textbf{Exercise #1.}\quad
}{
	\par
	\bigskip
}
\newenvironment{problem}[1]{
	\par
	\noindent\textbf{Problem #1.}\quad
}{
	\par
	\bigskip
}

\DeclareMathAccent{\widehat}{\mathord}{largesymbols}{"62}
\DeclareMathOperator*{\argmax}{\arg\,\max}
\DeclareMathOperator*{\argmin}{\arg\,\min}
\DeclareMathOperator{\E}{\mathbb E}
\DeclareMathOperator{\Var}{\mathrm{Var}}
\DeclareMathOperator{\tr}{\mathrm{tr}}
\DeclareMathOperator{\poly}{\mathrm{poly}}
\newcommand{\abs}[1]{\left| #1 \right|}
\newcommand{\vabs}[1]{\left\| #1 \right\|}
\newcommand{\abra}[1]{\left\langle #1 \right\rangle}
\newcommand{\pbra}[1]{\left( #1 \right)}
\newcommand{\cbra}[1]{\left\{ #1 \right\}}
\newcommand{\sbra}[1]{\left[ #1 \right]}
\newcommand{\floorbra}[1]{\left\lfloor #1 \right\rfloor}
\newcommand{\ceilbra}[1]{\left\lceil #1 \right\rceil}
\newcommand{\bin}{\{0,1\}}
\newcommand{\ZPP}{\mathtt{ZPP}}
\newcommand{\RP}{\mathtt{RP}}
\newcommand{\coRP}{\mathtt{co}\text{-}\mathtt{RP}}
\newcommand{\per}{\text{per}}
\newcommand{\sgn}{\text{sgn}}
\newcommand{\Fbb}{\mathbb{F}}
\newcommand{\Nbb}{\mathbb{N}}
\newcommand{\Rbb}{\mathbb{R}}
\newcommand{\Zbb}{\mathbb{Z}}
\newcommand{\Sset}{\mathbb{S}}
\newcommand{\Fset}{\mathbb{F}}
\newcommand{\Nset}{\mathbb{N}}
\newcommand{\Zset}{\mathbb{Z}}
\newcommand{\Uset}{\mathbb{U}}
\newcommand{\Acal}{\mathcal{A}}
\newcommand{\Bcal}{\mathcal{B}}
\newcommand{\Ccal}{\mathcal{C}}
\newcommand{\Fcal}{\mathcal{F}}
\newcommand{\Gcal}{\mathcal{G}}
\newcommand{\qd}[2]{{\left(\frac{#1}{#2}\right)}}
\newcommand{\dv}{\ |\ }
\newcommand{\mylm}{\xLongrightarrow[lm]{}}
\newcommand{\myrm}{\xLongrightarrow[rm]{}}

\bibliographystyle{plainnat}

\title{Exercise Set --- Chapter $5$}
\date{}

\begin{document}

\maketitle

\begin{exercise}{5.1.5} $G=(V,T,P,S)=(\{S,S'\},\{0,1\},P,S)$, where $P$ is
\begin{align*}
     S &\to S'\dv SS\dv S+S\dv S^*\dv (S)\\
     S' &\to e\dv 0\dv 1 \dv \varnothing
\end{align*}
\end{exercise}

\begin{exercise}{5.2.2}
    Let $T$ be the parse tree built step by step with the $m$-steps derivation;
    so $T$ has $m$ non-leaf nodes.
    Since $|w|=n$ and there is no $\varepsilon$ as right side in any production rule,
    $T$ has $n$ leaf nodes.
    As a result, the parse tree has $n+m$ nodes.
\end{exercise}

\begin{exercise}{5.2.3}
    Let $T$ be the parse tree built step by step with the $m$-steps derivation;
    so $T$ has $m$ non-leaf nodes.
    Since $|w|=n,n>0$, $T$ has $n$ non-$\varepsilon$ leaf nodes.
    \begin{itemize}
        \item If $m>1$, then the root of parse tree can not give rise to $\varepsilon$.
            So the number of $\varepsilon$ leaf nodes is no more than $m-1$.
            As a result, the parse tree has no more than $n+2m-1$ nodes.
        \item If $m=1$ and $w\neq\varepsilon$, then $w$ is derived in one step; thus
            the number of nodes in parse tree is no more than $n+2m-1$.
    \end{itemize}
\end{exercise}

\begin{exercise}{5.4.3} $G=(V,T,P,S)=(\{S,S_1,S_2\},\{a,b\},P,S)$, where $P$ is
\begin{align*}
    S &\to SaS_2bS_1\dv S_1\\
    S_1 &\to aS_1\dv \varepsilon\\
    S_2 &\to aS_2bS_2\dv \varepsilon\\
\end{align*}
\end{exercise}

\begin{exercise}{5.4.7}\hspace{0pt}\\
\textbf{a)}
\begin{itemize}
\item \textit{Leftmost derivation:}
    \begin{align*}
        E&\mylm +EE\mylm +*EEE\mylm +*-EEEE\\
        &\mylm +*-xEEE\mylm +*-xyEE\mylm +*-xyxE\mylm +*-xyxy
    \end{align*}
\item \textit{Rightmost derivation:}
    \begin{align*}
        E&\myrm +EE\myrm +Ey\myrm +*EEy\myrm +*Exy\\
        &\myrm +*-EExy\myrm +*-Eyxy\myrm +*-xyxy
    \end{align*}
\item \textit{Parse tree:}
    \begin{center}
    \begin{tikzpicture}[level/.style={level distance=10mm}]
        \node {E}[level 1/.style={sibling distance=40mm},
        level 2/.style={sibling distance=30mm},
        level 3/.style={sibling distance=20mm},
        level 4/.style={sibling distance=10mm},
        ]
        child {node {+}}
        child {node {E}
            child {node {*}
            }
            child {node {E}
                child {node {-}
                }
                child {node {E}
                    child {node {x}
                    }
                }
                child {node {E}
                    child {node {y}}
                }
            }
            child {node {E}
                child {node {x}
                }
            }
        }
        child {node {E}
            child {node {y}}
        };
    \end{tikzpicture}
    \end{center}
\end{itemize}
\textbf{b)}
    \begin{proof}
        Assume this grammar is ambiguous, there exists a $w$ and it has two distinct
        leftmost derivations, say $D_1$ and $D_2$.\par
        Since both $D_1,D_2$ are leftmost derivation and they are distinct, they
        must separate somewhere during the derivation. Thus
        \begin{align*}
            D_1: E&\xLongrightarrow[lm]{*} t_1t_2\cdots t_kET_1T_2\cdots T_m\\
            &\mylm t_1t_2\cdots t_kE_1EE\cdots E\\
            D_2: E&\xLongrightarrow[lm]{*} t_1t_2\cdots t_kET_1T_2\cdots T_m\\
            &\mylm t_1t_2\cdots t_kE_2EE\cdots E,\\
            & t_k\in\{+,-,*,x,y\}
        \end{align*}
        where $E_1,E_2\in\{+EE,*EE,-EE,x,y\},E_1\neq E_2$.
        However, after this step, they will have different prefix as the $(k+1)$th character
        is different; at least one of them will be different from $w$. A contradiction.
    \end{proof}
\end{exercise}

\end{document}
