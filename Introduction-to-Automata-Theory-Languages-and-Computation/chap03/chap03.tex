% !TeX encoding = UTF-8
% !TeX program = XeLaTeX
% !TeX spellcheck = LaTeX

\documentclass[a4paper]{article}

\usepackage{amsmath,amsfonts,amssymb}
\usepackage{mathrsfs}
\usepackage{bm}
\usepackage{extarrows}
\usepackage{geometry}
\usepackage{ntheorem}
\usepackage{hyperref}
\usepackage[ruled]{algorithm2e}
\usepackage{caption,subcaption}
\usepackage{tikz}

\usetikzlibrary{automata}

\geometry{left=2cm,right=2cm,top=2cm,bottom=2cm}

\def\UrlBreaks{\do\A\do\B\do\C\do\D\do\E\do\F\do\G\do\H\do\I\do\J\do\K\do\L\do\M\do\N\do\O\do\P\do\Q\do\R\do\S\do\T\do\U\do\V\do\W\do\X\do\Y\do\Z\do\[\do\\\do\]\do\^\do\_\do\`\do\a\do\b\do\c\do\d\do\e\do\f\do\g\do\h\do\i\do\j\do\k\do\l\do\m\do\n\do\o\do\p\do\q\do\r\do\s\do\t\do\u\do\v\do\w\do\x\do\y\do\z\do\0\do\1\do\2\do\3\do\4\do\5\do\6\do\7\do\8\do\9\do\.\do\@\do\\\do\/\do\!\do\_\do\|\do\;\do\>\do\]\do\)\do\,\do\?\do\'\do+\do\=\do\#}

\newtheorem{theorem}{Theorem}
\newtheorem{lemma}{Lemma}
\newtheorem{proposition}{Proposition}
\newtheorem{corollary}{Corollary}
\newtheorem{claim}{Claim}
\newtheorem{conjecture}{conjecture}
\newtheorem{definition}{Definition}
\newtheorem{construction}{Construction}
\newtheorem*{proof}{Proof}
\newtheorem*{answer}{Answer}
\newtheorem*{example}{Example}
\newtheorem*{counterexample}{Counterexample}

\newenvironment{exercise}[1]{
	\par
	\noindent\textbf{Exercise #1.}\quad
}{
	\par
	\bigskip
}
\newenvironment{problem}[1]{
	\par
	\noindent\textbf{Problem #1.}\quad
}{
	\par
	\bigskip
}

\DeclareMathAccent{\widehat}{\mathord}{largesymbols}{"62}
\DeclareMathOperator*{\argmax}{\arg\,\max}
\DeclareMathOperator*{\argmin}{\arg\,\min}
\DeclareMathOperator{\E}{\mathbb E}
\DeclareMathOperator{\Var}{\mathrm{Var}}
\DeclareMathOperator{\tr}{\mathrm{tr}}
\DeclareMathOperator{\poly}{\mathrm{poly}}
\newcommand{\abs}[1]{\left| #1 \right|}
\newcommand{\vabs}[1]{\left\| #1 \right\|}
\newcommand{\abra}[1]{\left\langle #1 \right\rangle}
\newcommand{\pbra}[1]{\left( #1 \right)}
\newcommand{\cbra}[1]{\left\{ #1 \right\}}
\newcommand{\sbra}[1]{\left[ #1 \right]}
\newcommand{\floorbra}[1]{\left\lfloor #1 \right\rfloor}
\newcommand{\ceilbra}[1]{\left\lceil #1 \right\rceil}
\newcommand{\bin}{\{0,1\}}
\newcommand{\ZPP}{\mathtt{ZPP}}
\newcommand{\RP}{\mathtt{RP}}
\newcommand{\coRP}{\mathtt{co}\text{-}\mathtt{RP}}
\newcommand{\per}{\text{per}}
\newcommand{\sgn}{\text{sgn}}
\newcommand{\Fbb}{\mathbb{F}}
\newcommand{\Nbb}{\mathbb{N}}
\newcommand{\Rbb}{\mathbb{R}}
\newcommand{\Zbb}{\mathbb{Z}}
\newcommand{\Sset}{\mathbb{S}}
\newcommand{\Fset}{\mathbb{F}}
\newcommand{\Nset}{\mathbb{N}}
\newcommand{\Zset}{\mathbb{Z}}
\newcommand{\Uset}{\mathbb{U}}
\newcommand{\Acal}{\mathcal{A}}
\newcommand{\Bcal}{\mathcal{B}}
\newcommand{\Ccal}{\mathcal{C}}
\newcommand{\Fcal}{\mathcal{F}}
\newcommand{\Gcal}{\mathcal{G}}
\newcommand{\qd}[2]{{\left(\frac{#1}{#2}\right)}}

\bibliographystyle{plainnat}

\title{Exercise Set --- Chapter $3$}
\date{}

\begin{document}

\maketitle

\noindent\textbf{Clarification:} In the following work,
\begin{itemize}
\item if the set of the states of a finite state automaton $A$ is $Q$,
then $Q=\{q_0,q_1,\cdots,q_{|Q|-1}\}$. So $q_t\in Q$ if and only if $t\in\{0,1,\cdots,|Q|-1\}$.
\end{itemize}

\begin{exercise}{3.2.6} Let the new $\varepsilon$-NFA be $B$ and $L=L(A)$.\\
    \textbf{a)} $L(B)=LL^*$\\
    \textbf{b)} $L(B)=\{x\in\Sigma^*|\exists y\in\Sigma^*, yx\in L\}$\\
    \textbf{c)} $L(B)=\{x\in\Sigma^*|\exists y\in\Sigma^*, xy\in L\}$\\
    \textbf{d)} $L(B)=\{x\in\Sigma^*|\exists y\in\Sigma^*, yxy\in L\}$
\end{exercise}

\begin{exercise}{3.2.7} $\{1\},\{2\},\{3\},\{1,2\}$\end{exercise}

\begin{exercise}{3.2.8}
    Since $A=(Q,\Sigma,\delta,q_0,F)$ is DFA,
    different strings mean different paths.\par
    Let $f[i][j][k]$ be the number of different strings of length $k$ going from $q_i$
    to $q_j$. Let $\delta[i][j]$ be the number of ways to go from $q_i$ to $q_j$ in one step, which means 
    $$
    \delta[i][j]=\sum_{c\in\Sigma} [\delta(q_i,c)=q_j].
    $$\par
    Then the transfer equation is
    \begin{gather*}
        f[i][j][k]=\sum_{t=0}^{|Q|-1} f[i][t][k-1]\cdot\delta[t][j].
    \end{gather*}
    The final answer will be $\sum_{q_t\in F} f[0][t][n]$; and 
    time complexity is $O(|Q|^3n)$.
\end{exercise}

\end{document}
