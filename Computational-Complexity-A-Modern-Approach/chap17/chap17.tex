
% !TeX encoding = UTF-8
% !TeX program = XeLaTeX
% !TeX spellcheck = LaTeX

\documentclass[a4paper]{article}

\usepackage{amsthm,amsmath,amsfonts,amssymb}
\usepackage{mathrsfs}
\usepackage{bm}
\usepackage{geometry}
% \usepackage{ntheorem}
\usepackage{hyperref}
\usepackage[ruled]{algorithm2e}
\usepackage{caption,subcaption}

\geometry{left=2cm,right=2cm,top=2cm,bottom=2cm}

\def\UrlBreaks{\do\A\do\B\do\C\do\D\do\E\do\F\do\G\do\H\do\I\do\J\do\K\do\L\do\M\do\N\do\O\do\P\do\Q\do\R\do\S\do\T\do\U\do\V\do\W\do\X\do\Y\do\Z\do\[\do\\\do\]\do\^\do\_\do\`\do\a\do\b\do\c\do\d\do\e\do\f\do\g\do\h\do\i\do\j\do\k\do\l\do\m\do\n\do\o\do\p\do\q\do\r\do\s\do\t\do\u\do\v\do\w\do\x\do\y\do\z\do\0\do\1\do\2\do\3\do\4\do\5\do\6\do\7\do\8\do\9\do\.\do\@\do\\\do\/\do\!\do\_\do\|\do\;\do\>\do\]\do\)\do\,\do\?\do\'\do+\do\=\do\#}

\newtheorem{theorem}{Theorem}
\newtheorem{lemma}{Lemma}
\newtheorem{proposition}{Proposition}
\newtheorem{corollary}{Corollary}
\newtheorem{claim}{Claim}
\newtheorem{conjecture}{conjecture}
\newtheorem{definition}{Definition}
\newtheorem{construction}{Construction}
\newtheorem*{answer}{Answer}
\newtheorem*{refute}{Refute}
\newtheorem*{example}{Example}
\newtheorem*{counterexample}{Counterexample}

\newenvironment{exercise}[1]{
	\par
	\noindent\textbf{Exercise #1.}\quad
}{
	\par
	\bigskip
}


\DeclareMathAccent{\widehat}{\mathord}{largesymbols}{"62}
\newcommand{\abs}[1]{\left| #1 \right|}
\newcommand{\pbra}[1]{\left( #1 \right)}
\newcommand{\cbra}[1]{\left\{ #1 \right\}}
\newcommand{\sbra}[1]{\left[ #1 \right]}
\newcommand{\bin}{\{0,1\}}
\newcommand{\SAT}{\texttt{SAT}}


\title{Exercise Set --- Chapter 17}
\date{}

\begin{document}
    \maketitle

    \begin{exercise}{17.1}
        Let $\SAT_0$ be the event that the Boolean formula is satisfied.
        We assume $\Pr\sbra{\SAT_0}>0$, otherwise $\Pr\sbra{x_1=1\mid\SAT_0}$ is undefined.
        Then we find a satisfying assignment $y$ as follows where $i$ starts from $1$ to $n$.
        \begin{itemize}
            \item If $\Pr\sbra{x_i=1\mid\SAT_{i-1}}>0$, then set $y_i=1$; otherwise set $y_i=0$.
            \item Let $\SAT_i$ be the event that the Boolean formula, when setting $x_1=y_1,\ldots,x_i=y_i$, is satisfied.
        \end{itemize}
        Observe that 
        \begin{align*}
            &\Pr\sbra{x_1=y_1\mid\SAT_0}\Pr\sbra{\SAT_0}=\Pr\sbra{x_1=y_1}\Pr\sbra{\SAT_0\mid x_1=y_1}=\frac12\Pr\sbra{\SAT_1},\\
            &\Pr\sbra{x_2=y_2\mid\SAT_1}\Pr\sbra{\SAT_1}=\frac12\Pr\sbra{\SAT_2},\\
            &\ldots,\\
            &\Pr\sbra{x_n=y_n\mid\SAT_{n-1}}\Pr\sbra{\SAT_{n-1}}=\frac12\Pr\sbra{\SAT_n}=\frac12,
        \end{align*}
        where each conditional probability is well defined as $y$ is a satisfying assignment.
        By telescoping product, we have
        $$
        \#\SAT=2^n\cdot\Pr\sbra{\SAT_0}=\frac1{\Pr\sbra{x_1=y_1\mid\SAT_0}}\frac1{\Pr\sbra{x_2=y_2\mid\SAT_1}}\cdots\frac1{\Pr\sbra{x_n=y_n\mid\SAT_{n-1}}}.
        $$
    \end{exercise}

\end{document}
