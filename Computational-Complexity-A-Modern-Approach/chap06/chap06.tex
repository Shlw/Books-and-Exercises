% !TeX encoding = UTF-8
% !TeX program = XeLaTeX
% !TeX spellcheck = LaTeX

\documentclass[a4paper]{article}

\usepackage{amsmath,amsfonts,amssymb}
\usepackage{mathrsfs}
\usepackage{bm}
\usepackage{geometry}
\usepackage{ntheorem}
\usepackage{hyperref}
\usepackage[ruled]{algorithm2e}
\usepackage{caption,subcaption}

\geometry{left=2cm,right=2cm,top=2cm,bottom=2cm}

\def\UrlBreaks{\do\A\do\B\do\C\do\D\do\E\do\F\do\G\do\H\do\I\do\J\do\K\do\L\do\M\do\N\do\O\do\P\do\Q\do\R\do\S\do\T\do\U\do\V\do\W\do\X\do\Y\do\Z\do\[\do\\\do\]\do\^\do\_\do\`\do\a\do\b\do\c\do\d\do\e\do\f\do\g\do\h\do\i\do\j\do\k\do\l\do\m\do\n\do\o\do\p\do\q\do\r\do\s\do\t\do\u\do\v\do\w\do\x\do\y\do\z\do\0\do\1\do\2\do\3\do\4\do\5\do\6\do\7\do\8\do\9\do\.\do\@\do\\\do\/\do\!\do\_\do\|\do\;\do\>\do\]\do\)\do\,\do\?\do\'\do+\do\=\do\#}

\newtheorem{theorem}{Theorem}
\newtheorem{lemma}{Lemma}
\newtheorem{proposition}{Proposition}
\newtheorem{corollary}{Corollary}
\newtheorem{claim}{Claim}
\newtheorem{conjecture}{conjecture}
\newtheorem{definition}{Definition}
\newtheorem{construction}{Construction}
\newtheorem*{proof}{Proof}
\newtheorem*{answer}{Answer}
\newtheorem*{refute}{Refute}
\newtheorem*{example}{Example}
\newtheorem*{counterexample}{Counterexample}

\newenvironment{exercise}[1]{
	\par
	\noindent\textbf{Exercise #1.}\quad
}{
	\par
	\bigskip
}


\DeclareMathAccent{\widehat}{\mathord}{largesymbols}{"62}
\newcommand{\abs}[1]{\left| #1 \right|}
\newcommand{\pbra}[1]{\left( #1 \right)}
\newcommand{\cbra}[1]{\left\{ #1 \right\}}
\newcommand{\sbra}[1]{\left[ #1 \right]}
\newcommand{\bin}{\{0,1\}}

\title{Exercise Set --- Chapter $6$}
\date{}

\begin{document}

\maketitle

\begin{exercise}{6.1}
    \begin{itemize}
    \item Logspace-uniform circuits $\subseteq$ $\mathbf{P}$-uniform circuits $=$ $\mathbf{P}$.
    \item It is possible to simulate every TM $M$ with an oblivious TM $M'$ such that $f(n,i,j)$, which represents the position
        on $n$-length input in the $i$-th tape after $j$ steps, can be computed in logarithmic space.
            Thus the construction in Theorem 6.6 can be logspace, and $\mathbf{P}$ $\subseteq$ Logspace-uniform circuits.
    \end{itemize}
\end{exercise}

\begin{exercise}{6.4}
Prove a language has logspace-uniform circuits of polynomial size iff it is in P.

The proof is similar to Theorem 6.6 ($P \subseteq P/backslash poly$).

\begin{itemize}
\item $\Rightarrow$  If $L$ has logspace-uniform circuit of $s$ size, then we can gain $size(n), type(n,i)$ and $edge(n,i,j)$, then we can compute the results of the circuit with poly$(s)$ time, i.e., $L \in P$.

\item $\Leftarrow$ Since $L$ can be simulated by an TM M in time $T(n)$ (in poly($n$)), then M can be simulated by an oblivious TM $M'$, which runs in time $T(n)^{2}$. Since  the snapshot $z_{i}$ is only related to time $T$ and $z_{i-1}$, thus we can compute $z_{i}$ in constant-sized circuit, then we have a $O(T(n))$ size circuit to compute L, and the circuit is logspace-uniform.
\end{itemize}

\end{exercise}


\begin{exercise}{6.6}
    According to Theorem 6.19, if $\textbf{NP} \subseteq \textbf{P}_{\textbf{/poly}}$, then $\textbf{PH} = \mathbf{\Sigma}_2^p$. Thus, the conclusion of Exercise 6.5 also holds with $\mathbf{PH}$ replaced by $\mathbf{\Sigma}_2^p$. Otherwise, there exists $L \in \textbf{NP}$ such that $L \not\in \textbf{P}_{\textbf{/poly}}$. Namely, the circuit complexity is $\Omega(n^k)$ for any constant $k$.
\end{exercise}

\end{document}
