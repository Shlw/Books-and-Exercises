% !TeX encoding = UTF-8
% !TeX program = XeLaTeX
% !TeX spellcheck = LaTeX

\documentclass[a4paper]{article}

\usepackage{amsmath,amsfonts,amssymb}
\usepackage{mathrsfs}
\usepackage{bm}
\usepackage{geometry}
\usepackage{ntheorem}
\usepackage{hyperref}
\usepackage[ruled]{algorithm2e}
\usepackage{caption,subcaption}

\geometry{left=2cm,right=2cm,top=2cm,bottom=2cm}

\def\UrlBreaks{\do\A\do\B\do\C\do\D\do\E\do\F\do\G\do\H\do\I\do\J\do\K\do\L\do\M\do\N\do\O\do\P\do\Q\do\R\do\S\do\T\do\U\do\V\do\W\do\X\do\Y\do\Z\do\[\do\\\do\]\do\^\do\_\do\`\do\a\do\b\do\c\do\d\do\e\do\f\do\g\do\h\do\i\do\j\do\k\do\l\do\m\do\n\do\o\do\p\do\q\do\r\do\s\do\t\do\u\do\v\do\w\do\x\do\y\do\z\do\0\do\1\do\2\do\3\do\4\do\5\do\6\do\7\do\8\do\9\do\.\do\@\do\\\do\/\do\!\do\_\do\|\do\;\do\>\do\]\do\)\do\,\do\?\do\'\do+\do\=\do\#}

\newtheorem{theorem}{Theorem}
\newtheorem{lemma}{Lemma}
\newtheorem{proposition}{Proposition}
\newtheorem{corollary}{Corollary}
\newtheorem{claim}{Claim}
\newtheorem{conjecture}{conjecture}
\newtheorem{definition}{Definition}
\newtheorem{construction}{Construction}
\newtheorem*{proof}{Proof}
\newtheorem*{answer}{Answer}
\newtheorem*{refute}{Refute}
\newtheorem*{example}{Example}
\newtheorem*{counterexample}{Counterexample}

\newenvironment{exercise}[1]{
	\par
	\noindent\textbf{Exercise #1.}\quad
}{
	\par
	\bigskip
}


\DeclareMathAccent{\widehat}{\mathord}{largesymbols}{"62}
\newcommand{\abs}[1]{\left| #1 \right|}
\newcommand{\pbra}[1]{\left( #1 \right)}
\newcommand{\cbra}[1]{\left\{ #1 \right\}}
\newcommand{\sbra}[1]{\left[ #1 \right]}
\newcommand{\bin}{\{0,1\}}

\title{Exercise Set --- Chapter $4$}
\date{}

\begin{document}

\maketitle


\begin{exercise}{4.3}
Since $L$ is not the empty set or $\{0,1\}^{*}$, there exists $x\in L$ and $y\not\in L$. In the following, we prove that $path(s,t)\leq_{p} L$. We can solve path$(s,t)$ in polynomial-time, and if it the yes case, we corresponding to $x$ in $L$, otherwise corresponding to $y$ in $L$.
\end{exercise}

\begin{exercise}{4.4}
    Let $\texttt{\sf SCD}=\cbra{\llcorner G\lrcorner\middle| G\text{ is a strongly connected digraph}}$.
    First, $\texttt{\sf SCD}\in NL$, since we can enumerate all vertex pair $(s,t)$ and run $\texttt{\sf PATH}(G,s,t)$.

    Second, we construct a log-space reduction from \texttt{\sf PATH} to show \texttt{\sf SCD} is $NL$-hard.
    The reduction simply adds two directed edge $i\to s,t\to i$ for any $i\neq s,t$. 
    It is easy to verify its validity.
\end{exercise}

\begin{exercise}{4.11}
We prove the corollary by the following two steps.
\begin{itemize}
\item[(i)] To prove $NSPACE(S(n))=CO-NSPACE(S(n))$, it suffices to prove that $NSPACE(S(n))\subset CONSPACE(S(n))$.

 Because  if $NSPACE(S(n))\subset CONSPACE(S(n))$, then for every language $L\in CO-NSPACE(S(n))$, by definition, $\overline{L}\in NSPACE(S(n))\in CO-NSPACE(S(n))$.  Then $L=\overline{\overline{L}}\in NSPACE(S(n)).$ Then $CO-NSPACE(S(n))\subset NSPACE(S(n))$ and we conclude $CO-NSPACE(S(n))=NSPACE(S(n))$.

\item[(ii)]
$NSPACE(S(n))\subset CONSPACE(S(n))$ means if $L\in NSPACE(S(n))$, then $\overline{L}\in NSPACE(S(n))$. 

To prove this, we construct Turing machine $M_{\overline{L}}$, such that for $x\in \overline{L}$, there exists a read-only-once certificate $u\in \{0,1\}^{poly(2^S(n))}$ such that $M_{\overline{L}}(x,u)=1$ and $M$ only use $S(n)$ space and only read $u$ for one time.

Because $L\in NSPACE(S(n))$, there exists a non-deterministic $S(n)$ space turing machine $M$ such that $x\in L$ iff $M$ accepts $x$. If we use $G_{M,x}$ to represent the underlying configuration graph, it's equivalent to say $x\in L$ iff $C_{M,x,start}, C_{M,x,accept}$ are connected. Then $x\in\overline{L}$ iff $C_{M,x,start}, C_{M,x,accept}$ are disconnected. The construction for $M_{\overline{L}}$ and $u$ are similar to . Write
$$
C_i=\{ v\in G_{M,x}| v \text{is reachable by $C_{M,x,start}$ in $i$ steps} \}
$$ and for convenience, we label the the configuration by $v_j, 0\leq j\leq 2^{O(S(n))}$, each vertices is represented by a $O(s(n)$ string. We consider the following two sub-routines.

\begin{itemize}
\item[(1)] Given $|C_i|$, we can construct a certificate to prove whether a given vertices $v\in C_{i+1}$ or not. The checking procedure uses $O(S(n))$ space. First, if someone claim $v\in C_{i+1}$ it's easy to verify by providing the path $v_0,...,v_k$ for $v_k=v, k\leq i$. If someone claim $v\not\in C_{i+1}$, he would provide the all the vertices $u_j$ in $C_{i+1}$ in descending order and its path connecting $v_0, v$ in $i$ steps. The checking procedure, that is $M_{\overline{L}}$, will
    \begin{itemize}
    \item[(a)] Maintain $u_j$ and check wether $u_{j}+1$  is greater than $u_j$. Using $O(S(n))$ space.
    \item[(b)] Maintain a number to count the total number of certificates, which is supposed to $|C_{i}|$. Using $O(S(n))$ space.
    \item[(c)] When reading $u\in C_{i+1}$, checking whether $v\in Neighbour(u)$. 
    \end{itemize} 
\item[(2)] Given $|C_i|$, we construct a certificate to prove the claimed $|C_{i+1}|$. This is done by listing the certificate for all $v\in G_{M,x}$, whether $v\in C_{i+1}$ or not, by using $(1)$.
\end{itemize}
\end{itemize}

\end{exercise}

\end{document}
