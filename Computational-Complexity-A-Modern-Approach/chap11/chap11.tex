% !TeX encoding = UTF-8
% !TeX spellcheck = LaTeX

\documentclass[a4paper]{article}

\usepackage{amsthm,amsmath,amsfonts,amssymb}
\usepackage{mathrsfs}
\usepackage{bm}
\usepackage{geometry}
% \usepackage{ntheorem}
\usepackage{hyperref}
\usepackage{caption,subcaption}

\geometry{left=2cm,right=2cm,top=2cm,bottom=2cm}

\def\UrlBreaks{\do\A\do\B\do\C\do\D\do\E\do\F\do\G\do\H\do\I\do\J\do\K\do\L\do\M\do\N\do\O\do\P\do\Q\do\R\do\S\do\T\do\U\do\V\do\W\do\X\do\Y\do\Z\do\[\do\\\do\]\do\^\do\_\do\`\do\a\do\b\do\c\do\d\do\e\do\f\do\g\do\h\do\i\do\j\do\k\do\l\do\m\do\n\do\o\do\p\do\q\do\r\do\s\do\t\do\u\do\v\do\w\do\x\do\y\do\z\do\0\do\1\do\2\do\3\do\4\do\5\do\6\do\7\do\8\do\9\do\.\do\@\do\\\do\/\do\!\do\_\do\|\do\;\do\>\do\]\do\)\do\,\do\?\do\'\do+\do\=\do\#}

\newtheorem{theorem}{Theorem}
\newtheorem{lemma}{Lemma}
\newtheorem{proposition}{Proposition}
\newtheorem{corollary}{Corollary}
\newtheorem{claim}{Claim}
\newtheorem{conjecture}{conjecture}
\newtheorem{definition}{Definition}
\newtheorem{construction}{Construction}
\newtheorem*{answer}{Answer}
\newtheorem*{refute}{Refute}
\newtheorem*{example}{Example}
\newtheorem*{counterexample}{Counterexample}

\newenvironment{exercise}[1]{
	\par
	\noindent\textbf{Exercise #1.}\quad
}{
	\par
	\bigskip
}


\DeclareMathAccent{\widehat}{\mathord}{largesymbols}{"62}
\newcommand{\abs}[1]{\left| #1 \right|}
\newcommand{\pbra}[1]{\left( #1 \right)}
\newcommand{\cbra}[1]{\left\{ #1 \right\}}
\newcommand{\sbra}[1]{\left[ #1 \right]}
\newcommand{\bin}{\{0,1\}}


\def\sD{\textsf{D}}
\def\sE{\textsf{E}}
\def\cD{\mathcal{D}}

\usepackage{algorithm,xspace}
\usepackage{algorithmicx}
\usepackage[noend]{algpseudocode}

\title{Exercise Set --- Chapter 11}
\date{}

\begin{document}

    \maketitle

    \begin{exercise}{11.5}
    	In Page 51 and Page 52, the proof of Theorem 2.14 introduce a way to reduce 3SAT to INDSET. Follow the way,we can transform in poly-time every $m$-clause 3CNF formula $\varphi$ into a $7m$-vertex graph $G$ such that if and only if $G$ has an independent set of size at least $m$.
 
		We add something in the construction:make the $7$ vertices which are show the same clause are connected to each other. Make $f$ is it, so $f(\varphi)$ is an $n$-vertex graph, and the largest independent set has size $var(\varphi)n/7$.
    \end{exercise}

    \begin{exercise}{11.9} $ $
        \begin{itemize}
            \item $\text{L-PCP}(\log n)\subseteq\text{NP}$:
                Assume $\mathcal A\in\text{L-PCP}(\log n)$ and $V_{\mathcal A}$ be the corresponding checker.
                For any input $x$, we can guess the certificate $y$, 
                which is of polynomial length since $V_{\mathcal A}$ is logspace.
                Then simulate $V_{\mathcal A}$ on all $2^{r(n)}=\text{poly}(n)$ possible randomness and count the number of accepted
                ones.
                If $x\in\mathcal A$, there exists $y$ such that all $2^{r(n)}$ cases accept.
                Otherwise, no $y$ would achieve success rate better than $\frac12$ as assumption.
            \item $\text{NP}\subseteq\text{L-PCP}(\log n)$:
                Since all languages in $\text{NP}$ can be reduced to \textsc{3-SAT} in logspace, it suffices to prove 
                $\textsc{3-SAT}\in\text{L-PCP}(\log n)$.

                Assume there are $n$ variables and $m$ clauses. The certificate $y$ consists of $nm$ bits and partitioned into
                $a_1a_2\cdots a_m$, where $a_1=\cdots=a_m$ is a satisfiable assignment.
                Then we describe $V_\textsc{3-SAT}$ in two parts.
                \begin{itemize}
                    \item \textbf{Satisfiability Checker:}
                        $V_\textsc{3-SAT}$ simply checks the satisfiability of the $i$-th clause in $a_i$, which can be done
                        as in one pass.
                    \item \textbf{Validity Checker:}
                        To make sure the certificate is consistent, i.e., $a_1=\cdots=a_m$, $V_\textsc{3-SAT}$ checks the validity
                        along the one pass.

                        With $O(\log n)$ random bits, 
                        it selects a random prime $1\leq p\leq n^c$ and a random integer $x\in[n^d]$, 
                        where $c,d$ are constants sufficiently large.
                        For $i\in[m]$, it computes 
                        $$
                        f_i=\sum_{j=1}^na_i[j]x^j\pmod{p}
                        $$
                        and checks if $f_i=f_{i+1}$, which can be done in logspace.
                        Assume $a_i\neq a_{i+1}$, 
                        \begin{align*}
                            &\Pr\sbra{f_i=f_j}\\
                            \leq&\Pr\sbra{f_i=f_j\middle|\sum_{j=1}^n(a_i[j]-a_{i+1}[j])x^j\neq0}
                            +\Pr\sbra{\sum_{j=1}^n(a_i[j]-a_{i+1}[j])x^j=0}\\
                            \leq&
                            \frac{\displaystyle\max_{t\in\sbra{n^{dn+1}}}\#\cbra{\text{prime }p\middle|p\text{ divides }t}}
                            {\pi(n^c)}
                            +\frac{n+1}{n^d}\\
                            \leq&
                            \frac{(dn+1)\log n}{\frac{n^c}{c\log n}}+\frac{n+1}{n^d}.
                        \end{align*}
                        Thus by union bound, the probability that fake $y$ fools $V_\textsc{3-SAT}$ is upper bounded by
                        $$
                            m\times\pbra{\frac{(dn+1)\log^2 n}{cn^c}+\frac{n+1}{n^d}}=o(1)
                        $$
                        as $c,d$ sufficiently large.
                \end{itemize}
        \end{itemize}
    \end{exercise}


    \begin{exercise}{11.10}
    	Given input $x$ and random bits $s$, $V_{x, s}(\pi)$ is a function which outputs $1$ if and only the prover accepts $x$ under $\pi, s$. Denote $R_{x, s}$ as the range of bits read by $V_{x, s}$ on $\pi$. Note that $V_{x, r}(\pi)$ and $R_{x, r}$ can be computed by a log-space Turing machine. Thus, $x \in L$ if there exists $\pi$ s.t. $V_{x, s}(\pi) = 1$ for all $s$, and $x \not\in L$ if for all $\pi$, $|\{s \mid V_{x, s}(\pi) = 1\}| \leq 2^{r(n)}/2$. These 2 cases can be distinguished by the following algorithm.

    	\begin{algorithm}[H]
        \caption{A log-space membership checking algorithm}\label{alg:soc}
        \begin{algorithmic}
            \Function{Check}{$x, \ell, r$} \Comment{\textsc{Check} can be computed in log-space when $r - \ell = O(\log n)$.}
            	\State Output $|\{s \in \{0,1\}^{r(n)} \mid R_{x, s} \subseteq [\ell, r]\}|$
            	\State Output $\max_\pi|\{s \in \{0,1\}^{r(n)} \mid R_{x, s} \subseteq [\ell, r], V_{x,s}(\pi) = 1\}|$
            \EndFunction
            \\
            \State $tot \gets 0, val \gets 0$
            \State $\ell \gets 1, r \gets 2a\log n$
            \While{$\ell \leq |\pi|$}
            	\State $(t, v) \gets $ \Call{Check}{$x, \ell, r$}
            	\State $tot \gets tot + t, val \gets val + v$
            	\State $\ell \gets r + 1, r \gets \ell + 2a\log n - 1$
            \EndWhile
            \\
            \State $tot' \gets 0, val' \gets 0$
            \State $\ell \gets a\log n + 1, r \gets 3a\log n$
            \While{$\ell \leq |\pi|$}
            	\State $(t, v) \gets $ \Call{Check}{$x, \ell, r$}
            	\State $tot' \gets tot' + t, val' \gets val' + v$
            	\State $\ell \gets r + 1, r \gets \ell + 2a\log n - 1$
            \EndWhile
            \\
            \If{$tot > tot'$}
      			\State \Return $\mathbf 1_{tot = val}$
            \Else
            	\State \Return $\mathbf 1_{tot' = val'}$
            \EndIf
        \end{algorithmic}
    \end{algorithm}
    \end{exercise}

\end{document}
