% !TeX encoding = UTF-8
% !TeX program = XeLaTeX
% !TeX spellcheck = LaTeX

\documentclass[a4paper]{article}

\usepackage{amsthm,amsmath,amsfonts,amssymb}
\usepackage{mathrsfs}
\usepackage{bm}
\usepackage{geometry}
% \usepackage{ntheorem}
\usepackage{hyperref}
\usepackage[ruled]{algorithm2e}
\usepackage{caption,subcaption}

\geometry{left=2cm,right=2cm,top=2cm,bottom=2cm}

\def\UrlBreaks{\do\A\do\B\do\C\do\D\do\E\do\F\do\G\do\H\do\I\do\J\do\K\do\L\do\M\do\N\do\O\do\P\do\Q\do\R\do\S\do\T\do\U\do\V\do\W\do\X\do\Y\do\Z\do\[\do\\\do\]\do\^\do\_\do\`\do\a\do\b\do\c\do\d\do\e\do\f\do\g\do\h\do\i\do\j\do\k\do\l\do\m\do\n\do\o\do\p\do\q\do\r\do\s\do\t\do\u\do\v\do\w\do\x\do\y\do\z\do\0\do\1\do\2\do\3\do\4\do\5\do\6\do\7\do\8\do\9\do\.\do\@\do\\\do\/\do\!\do\_\do\|\do\;\do\>\do\]\do\)\do\,\do\?\do\'\do+\do\=\do\#}

\newtheorem{theorem}{Theorem}
\newtheorem{lemma}{Lemma}
\newtheorem{proposition}{Proposition}
\newtheorem{corollary}{Corollary}
\newtheorem{claim}{Claim}
\newtheorem{conjecture}{conjecture}
\newtheorem{definition}{Definition}
\newtheorem{construction}{Construction}
\newtheorem*{answer}{Answer}
\newtheorem*{refute}{Refute}
\newtheorem*{example}{Example}
\newtheorem*{counterexample}{Counterexample}

\newenvironment{exercise}[1]{
	\par
	\noindent\textbf{Exercise #1.}\quad
}{
	\par
	\bigskip
}


\DeclareMathAccent{\widehat}{\mathord}{largesymbols}{"62}
\newcommand{\abs}[1]{\left| #1 \right|}
\newcommand{\pbra}[1]{\left( #1 \right)}
\newcommand{\cbra}[1]{\left\{ #1 \right\}}
\newcommand{\sbra}[1]{\left[ #1 \right]}
\newcommand{\bin}{\{0,1\}}

\title{Exercise Set --- Chapter 9}
\date{}

\begin{document}

    \maketitle

\begin{exercise}{9.2}
If for any two messages $x,x'$ such that $E_{U_{n}}(x)$ has the same distributions to $E_{U_{n}}(x')$. Then there exists $r$, s.t. $x$ and $x'$ are encrypted by the same key $k$. i.e., $\Pr[E_{k}(x) = y] = \Pr[E_{k'}(x)] = y$. In the other hand, if $E_{k}(x) = E_{k}(x') = y$, then we can not obtain the correct plaintext with $y$. Thus there exists $x,x'$, $E_{U_{n}}(x)$ is not the same distributions as $E_{U_{n}}(x')$. 
\end{exercise}

	\begin{exercise}{9.3}
        Since $\mathrm E_k(\cdot)$ is one-time pad encryption, the distribution of $\mathrm E_k(\cdot)$ is uniform.
        Then the desired inequality follows immediately.
	\end{exercise}

\begin{exercise}{9.15}
\begin{itemize}
\item [a.] There exists indistinguishable function $f^{-1}$, and polynomial $A$ s.t.,
\begin{align*}
|\Pr[A(f^{-1}(f(X_{n})))] - A(f^{-1}(f(Y_{n})))]|<\epsilon(n)
\end{align*}
i.e., $f(X_{n})$ and $f(Y_{n})$ are indistinguishable.
\item[b.] $\Leftarrow$ By definition 9.8.

$\Rightarrow$ Obvious?
\item[c.] 
\end{itemize}
\end{exercise}

%   \begin{exercise}{9.16}
%       Let 
%       $$
%       L=\cbra{(y,r,b)\in\bin^n\times\bin^n\times\bin\middle|b=r\odot f^{-1}(y)},
%       $$
%       then
%       $$
%       \overline L=\cbra{(y,r,b)\in\bin^n\times\bin^n\times\bin\middle|b\neq r\odot f^{-1}(y)}.
%       $$
%       Since $f$ is a permutation, $L$ is polynomially sampleable, 
%       as we first randomly select $r,x$ then let $b=x\odot t,y=f(x)$.
%        Similar argument works for $\overline L$ as well.
%
%    \end{exercise}

\end{document}
