% !TeX encoding = UTF-8
% !TeX program = XeLaTeX
% !TeX spellcheck = LaTeX

\documentclass[a4paper]{article}

\usepackage{amsthm,amsmath,amsfonts,amssymb}
\usepackage{mathrsfs}
\usepackage{bm}
\usepackage{geometry}
% \usepackage{ntheorem}
\usepackage{hyperref}
\usepackage[ruled]{algorithm2e}
\usepackage{caption,subcaption}

\geometry{left=2cm,right=2cm,top=2cm,bottom=2cm}

\def\UrlBreaks{\do\A\do\B\do\C\do\D\do\E\do\F\do\G\do\H\do\I\do\J\do\K\do\L\do\M\do\N\do\O\do\P\do\Q\do\R\do\S\do\T\do\U\do\V\do\W\do\X\do\Y\do\Z\do\[\do\\\do\]\do\^\do\_\do\`\do\a\do\b\do\c\do\d\do\e\do\f\do\g\do\h\do\i\do\j\do\k\do\l\do\m\do\n\do\o\do\p\do\q\do\r\do\s\do\t\do\u\do\v\do\w\do\x\do\y\do\z\do\0\do\1\do\2\do\3\do\4\do\5\do\6\do\7\do\8\do\9\do\.\do\@\do\\\do\/\do\!\do\_\do\|\do\;\do\>\do\]\do\)\do\,\do\?\do\'\do+\do\=\do\#}

\newtheorem{theorem}{Theorem}
\newtheorem{lemma}{Lemma}
\newtheorem{proposition}{Proposition}
\newtheorem{corollary}{Corollary}
\newtheorem{claim}{Claim}
\newtheorem{conjecture}{conjecture}
\newtheorem{definition}{Definition}
\newtheorem{construction}{Construction}
\newtheorem*{answer}{Answer}
\newtheorem*{refute}{Refute}
\newtheorem*{example}{Example}
\newtheorem*{counterexample}{Counterexample}

\newenvironment{exercise}[1]{
	\par
	\noindent\textbf{Exercise #1.}\quad
}{
	\par
	\bigskip
}


\DeclareMathAccent{\widehat}{\mathord}{largesymbols}{"62}
\newcommand{\abs}[1]{\left| #1 \right|}
\newcommand{\pbra}[1]{\left( #1 \right)}
\newcommand{\cbra}[1]{\left\{ #1 \right\}}
\newcommand{\sbra}[1]{\left[ #1 \right]}
\newcommand{\bin}{\{0,1\}}


\def\sD{\textsf{D}}
\def\sE{\textsf{E}}
\def\cD{\mathcal{D}}

\title{Exercise Set --- Chapter 9}
\date{}

\begin{document}

    \maketitle

	\begin{exercise}{9.2}
		If for any two messages $x,x'$ such that $E_{U_{n}}(x)$ has the same distributions to $E_{U_{n}}(x')$. Then there exists $r$, s.t. $x$ and $x'$ are encrypted by the same key $k$. i.e., $\Pr[E_{k}(x) = y] = \Pr[E_{k'}(x)] = y$. In the other hand, if $E_{k}(x) = E_{k}(x') = y$, then we can not obtain the correct plaintext with $y$. Thus there exists $x,x'$, $E_{U_{n}}(x)$ is not the same distributions as $E_{U_{n}}(x')$. 
	\end{exercise}

	\begin{exercise}{9.3}
        Since $\mathrm E_k(\cdot)$ is one-time pad encryption, the distribution of $\mathrm E_k(\cdot)$ is uniform.
        Then the desired inequality follows immediately.
	\end{exercise}

	\begin{exercise}{9.4}
		Define a distinguisher $A$ for each $x$ as 
		\[
			A_x(y): = \left\{\begin{array}{cl}
				x & \Pr[\sD_{U_n}(y) = x] \geq 2^{-n},\\
				0 & \text{otherwise.}
			\end{array}\right.
		\]
		Define $\text{error}(x)$ as the failure probability to distinguish $0$ and $x$. Thus,
		\begin{align*}
			\text{error}(x) 
				= &\frac12 \Pr_{k_1,r} [A_x(\sE_{k_1,r}(0))=x] + \frac12 \Pr_{k_1,r} [A_x(\sE_{k_1,r}(x))=0]\\
				= &\frac12 \Pr_{k_1,r} \left[\Pr_{k_2}[\sD_{k_2}(\sE_{k_1,r}(0))=x] \geq 2^{-n}\right] + \frac12 \Pr_{k_1,r} \left[\Pr_{k_2}[\sD_{k_2}(\sE_{k_1,r}(x))=x] < 2^{-n}\right]\\
				\leq &\frac12 \Pr_{k_1,k_2,r} \left[\sD_{k_2}(\sE_{k_1,r}(0))=x\right]\cdot 2^n
		\end{align*}
		Furthermore, we have
		\[
			\sum_x \text{error}(x) \leq \sum_x \Pr_{k_1,k_2,r} \left[\sD_{k_2}(\sE_{k_1,r}(0))=x\right]\cdot 2^{n - 1} \leq 2^{n - 1},
		\]
		which implies there exists $x$ such that $\text{error}(x) \leq 2^{n - m - 1}$ as required.
	\end{exercise}

	\begin{exercise}{9.12}
		Suppose $g(n)$ is a one-way permutation,then construct the function $f(n)$:
		\begin{equation}
			f(z,x)=0^ng(x);	
			f(0^n,x)=0^{2n}.
		\end{equation}

		the function $f(x)$ is a one-way function, but for every $c>0$,$f(x)^{n^c}$ is not a one-way function.Next we give the proof.  
		
		(1)$f(n)$ is a one-way function.
		
			Assume $f(n)$ is not a one-way function ,then there is a ploy-time algorithm $A$ and 
			\begin{equation}
				Pr_{s\in\{0,1\}^{2n};u=f(s)}(A(u)=s' \ s.t.\  f(s')=u)\geq\epsilon_1(2n)
			\end{equation}
			
			then we can construct a ploy-time algorithm $B$ by $A$ :
			
			\begin{center}
				\fbox{\shortstack[l]{$A$:\\input $u(|u|=2n)$;\\output $s(|s|=2n)$}}
				\qquad\qquad\qquad
				\fbox{\shortstack[l]{$B$:\\input $y(|y|=n)$;\\$A(0^ny)=S^1S^2\dots S^{2n}$;\\output $x=S^{n+1}\dots S^{2n}$}};
				
			\end{center}
		We name $P_1=Pr_{s\in\{0,1\}^{2n};u=f(s)}(A(u)=s' \ s.t.\  f(s')=u\ |\ u=0^{2n});P_2=Pr_{s\in\{0,1\}^{2n};u=f(s)}(A(u)=s' \ s.t.\  f(s')=u\ |\ u\ne0^{2n});P_3=Pr_{x\in\{0,1\}^{n};y=g(x)}(B(y)=x'\ s.t.\ g(x')=y\ |\ y=0^n)$
		
		\begin{equation}
		Pr_{s\in\{0,1\}^{2n};u=f(s)}(A(u)=s' \ s.t.\  f(s')=u)=\frac{1}{2^n}P_1+\frac{2^n-1}{2^n}P_2
		\end{equation}
		
		\begin{equation}
		Pr_{x\in\{0,1\}^{n};y=g(x)}(B(y)=x'\ s.t.\ g(x')=y)=\frac{1}{2^n}P_3+\frac{2^n-1}{2^n}P_2\geq\epsilon_1(2n)-\frac{1}{2^n}
		\end{equation}
		
		For every fixed negligible function$\epsilon_2(n)$,that $\forall c_2,\exists N_2,\forall n>N_2$,there is $\epsilon_2<\frac{1}{n^c_2}$.we can construct another negligible function$\epsilon_1(2n)$,that $\exists N_1,\forall 2n>N_1$,there is $\epsilon_1(2n)=\epsilon_1(n)-\frac{1}{2^n}<\frac{1}{n^{C_2}}+\frac{1}{2^n}<\frac{1}{(2n)^{\frac{C_2}{2}}}$.
		
		so,
		\begin{equation}
		Pr_{x\in\{0,1\}^{n};y=g(x)}(B(y)=x'\ s.t.\ g(x')=y)\geq \epsilon_2(n)
		\end{equation}
		
		It's contrast to that $g(x)$ is a one-way function, so f(x) is also a one-way function.
		
		(2)$\forall c>0$($n^c$ is still polynomial);$f^{n^c}(s)$ is not one-way function.Because
		\begin{equation}
		Pr_{s\in\{0,1\}^{2n};u=f^{n^c}(s)}(A(u)=s' \ s.t.\  f^{n^c}(s')=u)=1
		\end{equation}
			
	\end{exercise}

	\begin{exercise}{9.15}
		\begin{itemize}
			\item [a.] 
				There exists indistinguishable function $f^{-1}$, and polynomial $A$ s.t.,
				\begin{align*}
					|\Pr[A(f^{-1}(f(X_{n})))] - A(f^{-1}(f(Y_{n})))]|<\epsilon(n)
				\end{align*}
				i.e., $f(X_{n})$ and $f(Y_{n})$ are indistinguishable.

			\item[b.] 
				$\Leftarrow$ By definition 9.8.

				$\Rightarrow$ Obvious?

			\item[c.] 

		\end{itemize}
	\end{exercise}

   \begin{exercise}{9.16}
       Assume $f$ is a one-way permutation.
       Let 
       $$
       L=\cbra{(y,r,b)\in\bin^n\times\bin^n\times\bin\middle|b=r\odot f^{-1}(y)},
       $$
       then
       $$
       \overline L=\cbra{(y,r,b)\in\bin^n\times\bin^n\times\bin\middle|b\neq r\odot f^{-1}(y)}.
       $$
       Since $f$ is a permutation, $L$ is polynomially sampleable, 
       as we first randomly select $r,x$ then let $b=r\odot x,y=f(x)$.
       Similar argument works for $\overline L$ as well.
       Then by Goldreich-Levin Theorem, uniform distribution over $L$ is computationally indistinguishable from 
       uniform distribution over $\overline L$.

       Now we move to \textsc{3-Coloring}. Since $L\in\textbf{NP}$, there is a polynomial-time algorithm $\mathcal A$
       such that $s\in L$ iff $\mathcal A(s)\in\textsc{3-Coloring}$.
       Thus, the uniform distribution over $L$ induces a polynomially sampleable distribution over \textsc{3-Coloring}
       and ditto for $\overline L$ and $\overline{\textsc{3-Coloring}}$.
       Meanwhile, these two distributions are polynomial-time indistinguishable, following from the hardness of 
       distinguishing $L$ and $\overline L$.

    \end{exercise}

\end{document}
