% !TeX encoding = UTF-8
% !TeX program = XeLaTeX
% !TeX spellcheck = LaTeX

\documentclass[a4paper]{article}

\usepackage{amsthm,amsmath,amsfonts,amssymb}
\usepackage{mathrsfs}
\usepackage{bm}
\usepackage{geometry}
% \usepackage{ntheorem}
\usepackage{hyperref}
\usepackage[ruled]{algorithm2e}
\usepackage{caption,subcaption}

\geometry{left=2cm,right=2cm,top=2cm,bottom=2cm}

\def\UrlBreaks{\do\A\do\B\do\C\do\D\do\E\do\F\do\G\do\H\do\I\do\J\do\K\do\L\do\M\do\N\do\O\do\P\do\Q\do\R\do\S\do\T\do\U\do\V\do\W\do\X\do\Y\do\Z\do\[\do\\\do\]\do\^\do\_\do\`\do\a\do\b\do\c\do\d\do\e\do\f\do\g\do\h\do\i\do\j\do\k\do\l\do\m\do\n\do\o\do\p\do\q\do\r\do\s\do\t\do\u\do\v\do\w\do\x\do\y\do\z\do\0\do\1\do\2\do\3\do\4\do\5\do\6\do\7\do\8\do\9\do\.\do\@\do\\\do\/\do\!\do\_\do\|\do\;\do\>\do\]\do\)\do\,\do\?\do\'\do+\do\=\do\#}

\newtheorem{theorem}{Theorem}
\newtheorem{lemma}{Lemma}
\newtheorem{proposition}{Proposition}
\newtheorem{corollary}{Corollary}
\newtheorem{claim}{Claim}
\newtheorem{conjecture}{conjecture}
\newtheorem{definition}{Definition}
\newtheorem{construction}{Construction}
\newtheorem*{answer}{Answer}
\newtheorem*{refute}{Refute}
\newtheorem*{example}{Example}
\newtheorem*{counterexample}{Counterexample}

\newenvironment{exercise}[1]{
	\par
	\noindent\textbf{Exercise #1.}\quad
}{
	\par
	\bigskip
}


\DeclareMathAccent{\widehat}{\mathord}{largesymbols}{"62}
\newcommand{\abs}[1]{\left| #1 \right|}
\newcommand{\pbra}[1]{\left( #1 \right)}
\newcommand{\cbra}[1]{\left\{ #1 \right\}}
\newcommand{\sbra}[1]{\left[ #1 \right]}
\newcommand{\bin}{\{0,1\}}


\def\sD{\textsf{D}}
\def\sE{\textsf{E}}
\def\cD{\mathcal{D}}

\title{Exercise Set --- Chapter 10}
\date{}

\begin{document}

    \maketitle

    \begin{exercise}{10.8}
		题目不可证。改为 $10\log T$ 是 trival 的。		
	\end{exercise}

	\begin{exercise}{10.12}
        If $N,A$ are coprime, then there exists $u,v\in\mathbb Z$ such that $uN+vA=1$.
        Thus $uBN+vAB=B$. Now observe that $N$ divides $uBN+vAB$ as claimed, then $N$ must divide $B$.
	\end{exercise}

	\begin{exercise}{10.13}
		\begin{itemize}
			\item [\textbf{(a)}] Consider the following partition of $\mathbb Z_p^*$:
			\[
				\{\{1, p - 1\}, \{2, p - 2\}, \ldots, \{(p - 1)/2, (p + 1)/2\}\}.
			\]
			Since that $x_1^2 - x_2^2 \equiv (x_1 - x_2)(x_1 + x_2) \equiv 0 \pmod p$, it maps $x_1, x_2$ to the same element if and only if they are in the same pair.

			\item [\textbf{(b)}] Actually, if $2 \mid \ord(x)$, $\ord(x^2) = \ord(x)/2$.

			\item [\textbf{(c)}] Let $N = n_1\cdot n_2 \cdot \ldots \cdot n_k$ where $n_i$ is power of a prime and $n_i \perp n_j$ for all distinct $i, j \in [k]$. Assume $n_1 < \ldots < n_k$. If $n_1 \neq 2$ or $k > 2$, $r / 2$ can be divided by $\text{lcm}(\varphi(n_1), \ldots, \varphi(n_k))$, which means $x^{r/2} = 1$ for all $x \in \mathbb Z_{N}^*$. Thus, it suffices to proof $\Pr[x_2^{\varphi(n_2)/2} = 1] \geq 1/4$ due to Chinese Remainder Theorem, where $x_2$ is uniformly sampled from $\mathbb Z_{n_2}^*$. Let $x = a\cdot p + b$. We have:
			\[
				(a\cdot p + b)^{p^{k - 1}(p - 1)/2} \equiv (b^{(p - 1)/2})^{p^{k - 1}} \pmod {p^k}.
			\]
			Note that $(b^{(p - 1)/2})^{p^{k - 1}} \bmod {p^k} = b^{(p - 1)/2} \bmod p$. If $b$ is a quadratic residue modulo $p$, $b^{(p - 1)/2} \equiv 1 \pmod p$ holds, since there exists $b'$ s.t. $b'^2 = b$ and $b'^{(p - 1)} = 1$. According to \textbf{(a)}, there are $(p - 1)/2$ quadratic residues, which finishes the proof.

			% Note that $x^{r/2} \neq -1$ means $\ord(x) = r / c$ where $c$ is an odd. Let $N = n_1\cdot n_2 \cdot \ldots \cdot n_k$ where $n_i \perp n_j$ for all distinct $i, j \in [k]$. Due to Chinese Remainder Theorem, there exist $x_1 \in \mathbb Z_{n_1}^*, \ldots, x_k \in \mathbb Z_{n_k}^*$ such that $x \equiv x_i \pmod {n_i}$ for all $i \in [k]$. Whereas, $x^{r/2} \neq -1$ holds if and only if there exists $i \in [k]$, such that $\ord(x_i) \mid \varphi(n_i)/2$ holds. Since $k \geq 2$ and $x_i$ is uniformly sampled from $\mathbb Z_{n_i}^*$ for all $i \in [k]$, it suffices to proof $\Pr[x_1^{\varphi(n_1)/2} = 1] \geq 1/2$.
		\end{itemize}
	\end{exercise}

\end{document}
