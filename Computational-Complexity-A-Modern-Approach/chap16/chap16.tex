
% !TeX encoding = UTF-8
% !TeX program = XeLaTeX
% !TeX spellcheck = LaTeX

\documentclass[a4paper]{article}

\usepackage{amsthm,amsmath,amsfonts,amssymb}
\usepackage{mathrsfs}
\usepackage{bm}
\usepackage{geometry}
% \usepackage{ntheorem}
\usepackage{hyperref}
\usepackage[ruled]{algorithm2e}
\usepackage{caption,subcaption}

\geometry{left=2cm,right=2cm,top=2cm,bottom=2cm}

\def\UrlBreaks{\do\A\do\B\do\C\do\D\do\E\do\F\do\G\do\H\do\I\do\J\do\K\do\L\do\M\do\N\do\O\do\P\do\Q\do\R\do\S\do\T\do\U\do\V\do\W\do\X\do\Y\do\Z\do\[\do\\\do\]\do\^\do\_\do\`\do\a\do\b\do\c\do\d\do\e\do\f\do\g\do\h\do\i\do\j\do\k\do\l\do\m\do\n\do\o\do\p\do\q\do\r\do\s\do\t\do\u\do\v\do\w\do\x\do\y\do\z\do\0\do\1\do\2\do\3\do\4\do\5\do\6\do\7\do\8\do\9\do\.\do\@\do\\\do\/\do\!\do\_\do\|\do\;\do\>\do\]\do\)\do\,\do\?\do\'\do+\do\=\do\#}

\newtheorem{theorem}{Theorem}
\newtheorem{lemma}{Lemma}
\newtheorem{proposition}{Proposition}
\newtheorem{corollary}{Corollary}
\newtheorem{claim}{Claim}
\newtheorem{conjecture}{conjecture}
\newtheorem{definition}{Definition}
\newtheorem{construction}{Construction}
\newtheorem*{answer}{Answer}
\newtheorem*{refute}{Refute}
\newtheorem*{example}{Example}
\newtheorem*{counterexample}{Counterexample}

\newenvironment{exercise}[1]{
	\par
	\noindent\textbf{Exercise #1.}\quad
}{
	\par
	\bigskip
}


\DeclareMathAccent{\widehat}{\mathord}{largesymbols}{"62}
\newcommand{\abs}[1]{\left| #1 \right|}
\newcommand{\pbra}[1]{\left( #1 \right)}
\newcommand{\cbra}[1]{\left\{ #1 \right\}}
\newcommand{\sbra}[1]{\left[ #1 \right]}
\newcommand{\bin}{\{0,1\}}


\def\sD{\textsf{D}}
\def\sE{\textsf{E}}
\def\cD{\mathcal{D}}

\title{Exercise Set --- Chapter 16}
\date{}

\begin{document}
    \maketitle

    \begin{exercise}{16.1}
		任选一个函数$f$,存在一个拥有$S$个门的布尔线路计算$f$.对于该线路内的任一逻辑门$gate_i$,可用$l_i$长的代数直线程序实现,即可用$3l_i$规模的代数路线模拟,则该函数可使用$\sum_i 3i\cdot l_i$规模的代数路线模拟,令$c=3\cdot max{l_i,1/l_i}$,可知该代数路线的规模介于$c\cdot S$与$c/S$之间。
	\end{exercise}

    \begin{exercise}{16.4}
        \begin{itemize}
            \item[(a)]
                Let $\#\texttt{ADDs}$ be the number of additions in the $k\times k$ algorithm.
                We may assume $n=k^\ell$ and then use divide-and-conquer:
                $$
                \begin{bmatrix}
                    \bm A_{1,1} & \cdots & \bm A_{1,k} \\
                    \vdots & \ddots & \vdots \\
                    \bm A_{k,1} & \cdots & \bm A_{k,k}
                \end{bmatrix}
                \begin{bmatrix}
                    \bm B_{1,1} & \cdots & \bm B_{1,k} \\
                    \vdots & \ddots & \vdots \\
                    \bm B_{k,1} & \cdots & \bm B_{k,k}
                \end{bmatrix}
                =
                \begin{bmatrix}
                    \bm C_{1,1} & \cdots & \bm C_{1,k} \\
                    \vdots & \ddots & \vdots \\
                    \bm C_{k,1} & \cdots & \bm C_{k,k}
                \end{bmatrix},
                $$
                where each $\bm C_{i,j}$ comes from the $k\times k$ algorithm and 
                uses recursion for the multiplications required by $\bm C_{i,j}$.
                Thus 
                \begin{align*}
                    S(k^\ell)
                    &=k^\omega S(k^{\ell-1})+k^{2(\ell-1)}\cdot\#\texttt{ADDs}\\
                    &=n^{\omega}+\#\texttt{ADDs}\cdot\sum_{i=1}^\ell k^{2(\ell-i)}\cdot k^{(i-1)\omega}\\
                    &=O(n^\omega).
                \end{align*}
            \item[(b)]
                By (a) and Strassen's algorithm\footnote{\url{https://en.wikipedia.org/wiki/Strassen_algorithm}}, we can compute
                matrix multiplication of $2\times2$ matrices using $7$ multiplication gates. Thus $\omega=\log_27<2.81$.
        \end{itemize}
    \end{exercise}

	\begin{exercise}{16.12}
		对于$\forall L\in P/poly$,存在一个多项式规模的线路族${C_n}$判定L,该线路族可以编码为一个实数:令该线路族内每个线路编码成的二进制串位数固定为$n^c$位(多余位补零,c是一个足够大的实数),而后转化为该线路族的二进制编码,形成表示该线路族的实数。
		
		同时在BBS模型上,该实数可作为某个状态所关联的函数$p(x)$的系数,即该线路族的每个线路的多项式编码都蕴含在该系数上。对于BBS模型输入$|x|=n$,只需要抽取出表示$C_n$线路的多项式编码,即可在多项式时间上判定$x$是否属于L。
		
		由于BBS模型增加了判断$a>0$的能力,只需要将该系数进行多项式次数的除以2,加多项次$MOD 2$,每次取余数用分支测试查看所得值是否大于0,即得对应$n$输入的线路$C_n$的多项式长度的编码。然后用多项式时间模拟该线路,既可以判定输入是否属于L。
		
		故题给BBS模型可以判定P/poly内所有问题。		
	\end{exercise}


\end{document}
