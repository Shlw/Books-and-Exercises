% !TeX encoding = UTF-8
% !TeX program = XeLaTeX
% !TeX spellcheck = LaTeX

\documentclass[a4paper]{article}

\usepackage{amsthm,amsmath,amsfonts,amssymb}
\usepackage{mathrsfs}
\usepackage{bm}
\usepackage{geometry}
% \usepackage{ntheorem}
\usepackage{hyperref}
\usepackage[ruled]{algorithm2e}
\usepackage{caption,subcaption}

\geometry{left=2cm,right=2cm,top=2cm,bottom=2cm}

\def\UrlBreaks{\do\A\do\B\do\C\do\D\do\E\do\F\do\G\do\H\do\I\do\J\do\K\do\L\do\M\do\N\do\O\do\P\do\Q\do\R\do\S\do\T\do\U\do\V\do\W\do\X\do\Y\do\Z\do\[\do\\\do\]\do\^\do\_\do\`\do\a\do\b\do\c\do\d\do\e\do\f\do\g\do\h\do\i\do\j\do\k\do\l\do\m\do\n\do\o\do\p\do\q\do\r\do\s\do\t\do\u\do\v\do\w\do\x\do\y\do\z\do\0\do\1\do\2\do\3\do\4\do\5\do\6\do\7\do\8\do\9\do\.\do\@\do\\\do\/\do\!\do\_\do\|\do\;\do\>\do\]\do\)\do\,\do\?\do\'\do+\do\=\do\#}

\newtheorem{theorem}{Theorem}
\newtheorem{lemma}{Lemma}
\newtheorem{proposition}{Proposition}
\newtheorem{corollary}{Corollary}
\newtheorem{claim}{Claim}
\newtheorem{conjecture}{conjecture}
\newtheorem{definition}{Definition}
\newtheorem{construction}{Construction}
\newtheorem*{answer}{Answer}
\newtheorem*{refute}{Refute}
\newtheorem*{example}{Example}
\newtheorem*{counterexample}{Counterexample}

\newenvironment{exercise}[1]{
	\par
	\noindent\textbf{Exercise #1.}\quad
}{
	\par
	\bigskip
}


\DeclareMathAccent{\widehat}{\mathord}{largesymbols}{"62}
\newcommand{\abs}[1]{\left| #1 \right|}
\newcommand{\pbra}[1]{\left( #1 \right)}
\newcommand{\cbra}[1]{\left\{ #1 \right\}}
\newcommand{\sbra}[1]{\left[ #1 \right]}
\newcommand{\bin}{\{0,1\}}


\def\sD{\textsf{D}}
\def\sE{\textsf{E}}
\def\cD{\mathcal{D}}

\title{Exercise Set --- Chapter 13}
\date{}

\begin{document}

    \maketitle

    \begin{exercise}{13.7}
    	Note that $M(f)$ is a random matrix uniformly sampled from $\mathbb F_2^{2^n\times 2^n}$. $\Pr[\text{rank}(M(f)) = 2^n] = \Omega(1)$ has been shown in a previous exercise. Next, we prove that the size of the maximum fooling set is less that $n^3$ w.h.p. Let $\mathcal S \subseteq [2^n]^2$ be the set family gathering all $n^3$-size set $S$ satisfying the element in $S$ are located in distinct rows and columns. It is easy to see that any $n^3$-size fooling set must be contained in $\mathcal S$ and $|\mathcal S| = 2^{O(n^4)}$. Whereas, any $S \in \mathcal S$ is a fooling set on a random communication matrix $M(f)$ with prob. of $2^{-\Theta(n^6)}$. Thus, \emph{Union bound} could show the result we want.
    \end{exercise}

    \begin{exercise}{13.18}
    	The communication matrix would be divided into several submatrices during a protocol. After the $k$th round, $M(f)$ are divide into $M^{(k)}_1, \ldots, M^{(k)}_{2^k}$. Thus, in the $(k + 1)$round, the protocol further divides each submatrix $M_i^{(k)}: S_i^{(k)}\times T_i^{(k)} \to \{0, 1\}$ into $2$ submatrices $M_{i_1}^{(k + 1)}: S_{i_1}^{(k + 1)}\times T_{i_1}^{(k + 1)} \to \{0, 1\}$ and $M_{i_2}^{(k + 1)}: S_{i_2}^{(k + 1)}\times T_{i_2}^{(k + 1)} \to \{0, 1\}$ where $S_i^{(k)}, T_i^{(k)} \subseteq [2^n]$ and, w.l.o.g, $S_{i_1}^{(k + 1)} \sqcup S_{i_2}^{(k + 1)} = S_i^{(k)}$ and $T_{i_1}^{(k + 1)} = T_{i_2}^{(k + 1)} = T_i^{(k)}$ holds. It is easy to see that there are no more than $(2^{2^n})^{2^k}$ different divisions in the $(k+1)$th round. So, there are at most $2^{\sum_{k = 1}^K2^{n+k}}$ ($K \leq n$), which is exponential of the input size $2^{2n}$, different $K$-round protocols. Since that the correctness of a protocol can be checked efficiently, a brute force algorithm is in exponential-time.
    \end{exercise}

\end{document}
