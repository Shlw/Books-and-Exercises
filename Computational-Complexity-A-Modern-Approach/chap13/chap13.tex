% !TeX encoding = UTF-8
% !TeX program = XeLaTeX
% !TeX spellcheck = LaTeX

\documentclass[a4paper]{article}

\usepackage{amsthm,amsmath,amsfonts,amssymb}
\usepackage{mathrsfs}
\usepackage{bm}
\usepackage{geometry}
% \usepackage{ntheorem}
\usepackage{hyperref}
\usepackage[ruled]{algorithm2e}
\usepackage{caption,subcaption}

\geometry{left=2cm,right=2cm,top=2cm,bottom=2cm}

\def\UrlBreaks{\do\A\do\B\do\C\do\D\do\E\do\F\do\G\do\H\do\I\do\J\do\K\do\L\do\M\do\N\do\O\do\P\do\Q\do\R\do\S\do\T\do\U\do\V\do\W\do\X\do\Y\do\Z\do\[\do\\\do\]\do\^\do\_\do\`\do\a\do\b\do\c\do\d\do\e\do\f\do\g\do\h\do\i\do\j\do\k\do\l\do\m\do\n\do\o\do\p\do\q\do\r\do\s\do\t\do\u\do\v\do\w\do\x\do\y\do\z\do\0\do\1\do\2\do\3\do\4\do\5\do\6\do\7\do\8\do\9\do\.\do\@\do\\\do\/\do\!\do\_\do\|\do\;\do\>\do\]\do\)\do\,\do\?\do\'\do+\do\=\do\#}

\newtheorem{theorem}{Theorem}
\newtheorem{lemma}{Lemma}
\newtheorem{proposition}{Proposition}
\newtheorem{corollary}{Corollary}
\newtheorem{claim}{Claim}
\newtheorem{conjecture}{conjecture}
\newtheorem{definition}{Definition}
\newtheorem{construction}{Construction}
\newtheorem*{answer}{Answer}
\newtheorem*{refute}{Refute}
\newtheorem*{example}{Example}
\newtheorem*{counterexample}{Counterexample}

\newenvironment{exercise}[1]{
	\par
	\noindent\textbf{Exercise #1.}\quad
}{
	\par
	\bigskip
}


\DeclareMathAccent{\widehat}{\mathord}{largesymbols}{"62}
\newcommand{\abs}[1]{\left| #1 \right|}
\newcommand{\pbra}[1]{\left( #1 \right)}
\newcommand{\cbra}[1]{\left\{ #1 \right\}}
\newcommand{\sbra}[1]{\left[ #1 \right]}
\newcommand{\bin}{\{0,1\}}


\def\sD{\textsf{D}}
\def\sE{\textsf{E}}
\def\cD{\mathcal{D}}

\title{Exercise Set --- Chapter 13}
\date{}

\begin{document}

    \maketitle


	\begin{exercise}{13.3}
		使用反证法,假设存在 $o(n^2)$ 时间去判断 PAL 的图灵机,根据 H537 的提示,可为相等函数构造出一个通信协议,该通信协议在多余 $2^{n/2}/n$ 个输入上,仅通信 $o(n)$ 位,与 Thm 13.4 矛盾。	
	\end{exercise}

    \begin{exercise}{13.4}
    Just as in the previous question, by creating a "buffer zone" of zeores, the machine has to take $n$ steps just to transmit every message between Alice and Bob. And note that the machine has a space $S(n)$ work tape, which means every message can carry $S(n)$ bits information, it therefore takes at least $\Omega(n^2 / S(n))$ steps to decide the language.   
    \end{exercise}

    \begin{exercise}{13.7}
    	Note that $M(f)$ is a random matrix uniformly sampled from $\mathbb F_2^{2^n\times 2^n}$. $\Pr[\text{rank}(M(f)) = 2^n] = \Omega(1)$ has been shown in a previous exercise. Next, we prove that the size of the maximum fooling set is less that $n^3$ w.h.p. Let $\mathcal S \subseteq [2^n]^2$ be the set family gathering all $n^3$-size set $S$ satisfying the element in $S$ are located in distinct rows and columns. It is easy to see that any $n^3$-size fooling set must be contained in $\mathcal S$ and $|\mathcal S| = 2^{O(n^4)}$. Whereas, any $S \in \mathcal S$ is a fooling set on a random communication matrix $M(f)$ with prob. of $2^{-\Theta(n^6)}$. Thus, \emph{Union bound} could show the result we want.
    \end{exercise}

    \begin{exercise}{13.8}
	    \begin {definition}(entry wise product)
	        A,B is a n*m matrix, then 
	        $$A\circ B=\begin{bmatrix} 
	   	A_{11}B_{11} &\cdots  & A_{1n}B_{1n} \\
	        &\vdots&\\
	   	A_{n1}B_{n1} & \cdots&A_{nn}B_{nn}\\
	   	\end{bmatrix}	$$
	    \end{definition}
	    W.L.O.G, we have a fooling set $S$, $f(S) = 1$. Let $S = \{(x_1,y_1),(x_2,y_2)\cdots,(x_n,y_n)\}$. Let $M_a$ be the submatrix of the $M_f$, choose the row and column in the $S$. Using the definition of  fooling set, $M_a \circ M_a^T = I$. So $|S| = rank(I) = rank(M_a \circ M_a^T) \le M_a \otimes M_a^T = (rank(A))^2 \le (rank(M_f))^2$. So the rank method can get a lower bound at least 1/2$\lceil \log S \rceil$.
	\end{exercise}

	\begin{exercise}{13.14}
		仿照Lemma 13.17的证明。
		
			$\underset{\underset{(k,n)cube}D}E(\underset{a\in D}\pi f(a))\\$
			
			$=\underset{a_1,a_1^{'}}E(\underset{\underset{(k-1,n)cube}{D^{'}}}E(\underset{a\in D^{'}}\pi f(a_1,a)\underset{a\in D^{'}}\pi f(a_1^{'},a)))\\$
			
			$\geq\underset{a_1,a_1^{'}}E(\underset{\underset{(k-1,n)cube}{D^{'}}}E(\underset{a\in D^{'}}\pi f(a_1,a))g_1{(a_1)}^{2^{k-1}}\cdot(\underset{a\in D^{'}}\pi f(a_1^{'},a))g_1{(a_1^{'})}^{2^{k-1}}))\\$
			
			$=\underset{\underset{(k-1,n)cube}{D^{'}}}E(\underset{a_1}E\;\underset{a\in D^{'}}\pi f(a_1,a)g_1{{(a_1)})}^2\\$
			
			$\geq(\underset{\underset{(k-1,n)cube}{D^{'}}}E\underset{a_1}E\;\underset{a\in D^{'}}\pi f(a_1,a)g_1{{(a_1)})}^2\\$
			
			$\geq\cdots\cdots\geqslant(\underset{a_1\dots a_k}Ef(a_1,\dots,a_k)g_1{(a_1)}\dots g_k{{(a_k)})}^{2^k}$
		
		又:
		
			$Disc(f)=\frac1{2^{nk}}\underset T{\;max}\vert\sum_{(a_1,\cdots,a_k)\in T}\;f(a_1,\cdots,a_k)\vert\\$
			
			$=\underset{a_1\dots a_k}Ef(a_1,\dots,a_k)g_1{(a_1)}\dots g_k{(a_k)}$
			
		得证:$Disc(f)\leq(\epsilon(f))^{1/{2^k}}$
		
	\end{exercise}

    \begin{exercise}{13.15}
    	We use the fingerprint technique to obtain a randomized communication protocal. Given two vectors $\mathbf{a}=(a_0,a_1,\cdots,a_{n-1})$,$\mathbf{b}=(b_0,b_1,\cdots,b_{n-1})$, $ a_i,b_i \in \{0,1\}, 0\leq i \leq n-1 $, we construct their corresponding  $(n-1)$-degree polynomials as follows.
      \begin{equation*}
      	 \left\{
      	 	\begin{aligned}
      	 		A_n(x) &= a_{n-1} x^{n-1} + \cdots + a_1 x^1 + a_0 \\
      	 		B_n(x) &= b_{n-1} x^{n-1} + \cdots + b_1 x^1 + b_0 \\
      	 	\end{aligned}
      	 \right .
      \end{equation*}
      And then we have the following proctocol.
      \begin{itemize}
       \item Alice: Pick a random number $k \in \left[ 3n \right]$ and  compute $y_1\equiv A_n(k) \mod (3n) $. Send $k$ and $y_1$ to Bob.
       \item Bob: Compute $y_2 \equiv B_n(k) \mod (3n)$. If $y_2 = y_1$, send the last message 1 to Alice; Otherwise, send 0 to Alice. 
      \end{itemize}
      Since  it requires at most $2\log(3n)$ rounds to send $y_1$ and $k$,  the total communication complexity of the protocol is $O(\log n)$. Next, we'll prove the correctness of the protocol.
      \begin{itemize}
      		\item Compeleteness : if $\mathbf{a}= \mathbf{b}$, the last message of the protocol is always 1.
      		\item Soundness : if $\mathbf{a} \neq \mathbf{b}$, $p(x) := A_n(x) - B_n(x)$ is a univariate nonzero polynomial with degree at most $n-1$, according to Schwartz-Zippel Lemma,
      		\[
      			{\rm Pr}_{x\in_{R} ~[3n]}\left[p(x) \neq 0 \right] \geq 1- \frac{n-1}{3n} > \frac{2}{3} .
      		\]
            Therefore , the protocol output 0 with probability at least $2/3$. 
      \end{itemize} 
  Overall, by giving a $O(\log n)$ randomized communication protocol for equality fucntion, we prove $R(EQ)$ is at most $O(\log n)$.
    \end{exercise}

    \begin{exercise}{13.17}
        Consider the input of Player 1 being a uniform random string among $\bm x\in\bin^n$ and the input of Player 2 being $\bm y=\bm x\oplus\bm e_i$, where $e_i$ is a uniform unit vector. Then condition on Player 1 seeing $\bm x$, $i$ is still of entropy $n$, thus he needs at least $\log n$ bits from Player 2 to determine $i$, vice versa. Thus computing parity in KW game requires at least $2\log n$ bits of communication, which means computing parity requires depth at least $2\log n$.
    \end{exercise}

    \begin{exercise}{13.18}
    	The communication matrix would be divided into several submatrices during a protocol. After the $k$th round, $M(f)$ are divide into $M^{(k)}_1, \ldots, M^{(k)}_{2^k}$. Thus, in the $(k + 1)$round, the protocol further divides each submatrix $M_i^{(k)}: S_i^{(k)}\times T_i^{(k)} \to \{0, 1\}$ into $2$ submatrices $M_{i_1}^{(k + 1)}: S_{i_1}^{(k + 1)}\times T_{i_1}^{(k + 1)} \to \{0, 1\}$ and $M_{i_2}^{(k + 1)}: S_{i_2}^{(k + 1)}\times T_{i_2}^{(k + 1)} \to \{0, 1\}$ where $S_i^{(k)}, T_i^{(k)} \subseteq [2^n]$ and, w.l.o.g, $S_{i_1}^{(k + 1)} \sqcup S_{i_2}^{(k + 1)} = S_i^{(k)}$ and $T_{i_1}^{(k + 1)} = T_{i_2}^{(k + 1)} = T_i^{(k)}$ hold. It is easy to see that there are no more than $(2^{2^n})^{2^k}$ different divisions in the $(k+1)$th round. So, there are at most $2^{\sum_{k = 1}^K2^{n+k}}$ ($K \leq n$), which is exponential of the input size $2^{2n}$, different $K$-round protocols. Since that the correctness of a protocol can be checked efficiently, a brute force algorithm is in exponential-time.
    \end{exercise}
    
    \begin{exercise}{13.19}
        \begin{definition}
             $DIS_n:\{0,1\}^n\times\{0,1\}^n \mapsto \{0,1\}.DIS_n(x,y) = 1$ iff  the subsets of $\{1, 2, . . . , n\}$ whose characteristic vectors are $x$ and $y$ intersect.
        \end{definition}
        \begin{lemma}
            The space complexity of DIS is $\Omega(n)$, and even the approximate of this problem within a $3/4$ factor is also $\Omega(n)$.
            \label{thm:DIS}
        \end{lemma}
        We now assume that we have an $o(n)$ algorithm to calculate the frequency of the most frequent element.  We then use this algorithm to describe a simple protocol for two parties to compute $DIS(x,y)$. The first person just calculate the $DIS(x,y)$ use the algorithm, and then send the memory to another people, the second people just use this memory to calculate the $DIS(x,y)$. Then we get a protocol to calculate the $DIS(x,y)$. However, according to Lemma \ref{thm:DIS}, this is impossible, so there is no space-$o(n)$ streaming algorithm that solves the problem.
    \end{exercise}
\end{document}
