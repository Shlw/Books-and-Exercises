% !TeX encoding = UTF-8
% !TeX program = XeLaTeX
% !TeX spellcheck = LaTeX

\documentclass[a4paper]{article}

\usepackage[UTF8]{ctex}
\usepackage{amsmath,amsfonts,amssymb}
\usepackage{mathrsfs}
\usepackage{bm}
\usepackage{geometry}
\usepackage{ntheorem}
\usepackage{hyperref}
\usepackage[ruled]{algorithm2e}
\usepackage{caption,subcaption}

\geometry{left=2cm,right=2cm,top=2cm,bottom=2cm}

\def\UrlBreaks{\do\A\do\B\do\C\do\D\do\E\do\F\do\G\do\H\do\I\do\J\do\K\do\L\do\M\do\N\do\O\do\P\do\Q\do\R\do\S\do\T\do\U\do\V\do\W\do\X\do\Y\do\Z\do\[\do\\\do\]\do\^\do\_\do\`\do\a\do\b\do\c\do\d\do\e\do\f\do\g\do\h\do\i\do\j\do\k\do\l\do\m\do\n\do\o\do\p\do\q\do\r\do\s\do\t\do\u\do\v\do\w\do\x\do\y\do\z\do\0\do\1\do\2\do\3\do\4\do\5\do\6\do\7\do\8\do\9\do\.\do\@\do\\\do\/\do\!\do\_\do\|\do\;\do\>\do\]\do\)\do\,\do\?\do\'\do+\do\=\do\#}

\newtheorem{theorem}{Theorem}
\newtheorem{lemma}{Lemma}
\newtheorem{proposition}{Proposition}
\newtheorem{corollary}{Corollary}
\newtheorem{claim}{Claim}
\newtheorem{conjecture}{conjecture}
\newtheorem{definition}{Definition}
\newtheorem{construction}{Construction}
\newtheorem*{proof}{Proof}
\newtheorem*{answer}{Answer}
\newtheorem*{refute}{Refute}
\newtheorem*{example}{Example}
\newtheorem*{counterexample}{Counterexample}

\newenvironment{exercise}[1]{
	\par
	\noindent\textbf{Exercise #1.}\quad
}{
	\par
	\bigskip
}


\DeclareMathAccent{\widehat}{\mathord}{largesymbols}{"62}
\newcommand{\abs}[1]{\left| #1 \right|}
\newcommand{\pbra}[1]{\left( #1 \right)}
\newcommand{\cbra}[1]{\left\{ #1 \right\}}
\newcommand{\sbra}[1]{\left[ #1 \right]}
\newcommand{\bin}{\{0,1\}}

\title{Exercise Set --- Chapter $5$}
\date{}

\begin{document}

\maketitle

\begin{exercise}{5.1}
    It is clearly seen that $\Sigma_i SAT$ is in $\Sigma_i P$ from their definition.Therefore, the only thing we need to show is that $\Sigma_i SAT$ is $\Sigma_i P$-hard.We use mathematical induction to prove this result.Suppose the language $\Sigma_i SAT$ is $\Sigma_i P$-hard.
    \begin{itemize}
        \item The basic situation is trivial since the language $SAT$ has been proved to be $NP$-hard.
        \item And the induction follows for $i>1$ if we assume the result holds when $i < n$.
        \item When $i=n$, we need to prove every language $L \in \Sigma_n P$ is polynomial-time Karp reducible to $\Sigma_n SAT$. According to the definition of $\Sigma_n P$,
        \begin{equation}
            x \in L \iff \exists u_1\in\{0,1\}^{q(|x|)} \forall u_2 \in \{0,1\}^{q(|x|)}\cdots Q_iu_i\in\{0,1\}^{q(|x|)} M(x,u_1,\dots,u_i)=1
        \end{equation}
        where $M$ is a polynomial-time TM, $q$ is a polynomial, and $Q_i$ denotes $\forall$ or $\exists$ depending on whetehr $i$ is even or odd respectively.
        
        We define a new language $L^{\prime}$
        \begin{equation}
           x,u_1 \in L^{\prime} \iff \forall u_2 \in \{0,1\}^{q(|x|)}\cdots Q_iu_i\in\{0,1\}^{q(|x|)} M(x,u_1,\dots,u_i)=1
        \end{equation}
        
        Thus, we can rewrite $L$ in another way
        
        \begin{equation}
          x\in L \iff \exists u_1 x,u_1 \in L^{\prime}
        \end{equation}
        
        Note that $L^{\prime}$ is in $\Pi_{n-1} P$ and $\Pi_{n-1} SAT $ is $\Pi_{n-1}$-hard(the proof is analogous to that of TAUTOLOGY).Thus, there is a polynomial time transformation $x \rightarrow \varphi_{x}$ from strings to boolean formulae such that
        \begin{equation}
          x,u_1 \in L^{\prime}  \iff \forall u_2 \in \{0,1\}^{q(|x|)}\cdots Q_i u_i\in\{0,1\}^{q(|x|)}\varphi_{x,u_1}(u_2,\dots,u_i)=1. 
        \end{equation}
        
        Let $\phi_{x}(u_1,u_2,\dots,\u_i)=\varphi_{x,u_1}(u_2,\dots,u_i)$, we have the following equivalent definition of $L$
        \begin{equation}
        x\ in L^{\prime} \iff  \exists u_1\in\{0,1\}^{q(|x|)} \forall u_2 \in \{0,1\}^{q(|x|)}\cdots Q_iu_i\in\{0,1\}^{q(|x|)} \phi_{x}(u_1,\dots,u_i)=1
        \end{equation}
        And obviously, we can transform $x$ to $\phi$ in polynomial time. In a word, we have proved the result in the situation when $i=n$. 
    \end{itemize} 
    
\end{exercise}

\begin{exercise}{5.3}
    Assume $\tt 3SAT$ is polynomial-time reducible to $\overline{\tt 3SAT}$. Denote this reduction as $f$ and the TM checking
    $\tt 3SAT$ as $M$, then
    $$
    \exists y,M(x,y)=1\iff \forall y,M(f(x),y)=0.
    $$
    Therefore
    $$
    \forall y,M(x,y)=0\iff\neg\pbra{\exists y,M(x,y)=1}\iff\neg\pbra{\forall y,M(f(x),y)=0}\iff\exists y,M(f(x),y)=1,
    $$
    which means $f$ is also the polynomial-time reduction from $\overline{\tt 3SAT}$ to $\tt 3SAT$.
    Hence $\Sigma_1^p=\mathbf{NP}=\mathbf{coNP}=\Pi_1^p$ and $\mathbf{PH}$ collapses to $\mathbf{NP}$.
\end{exercise}

\begin{exercise}{5.4}
    By Claim 5.9, $\Sigma_i^p=\bigcup_c\Sigma_i\text{TIME}(n^c),\Pi_i^p=\bigcup_c\Pi_i\text{TIME}(n^c)$.
    Then 
    $$
    \text{PH}=\bigcup_{i,c}\Sigma_i\text{TIME}(n^c)=\bigcup_{i,c}\Pi_i\text{TIME}(n^c).
    $$
\end{exercise}

\begin{exercise}{5.5}
    $\mathbf{PSPACE} \subseteq \mathbf{AP}$ follows since $\mathsf{TQBF}$ is trivially in $\mathbf{AP}$, and every $\mathbf{PSPACE}$ language reduces to $\mathsf{TQBF}$. It suffices to show that $\mathbf{AP} \subseteq \mathbf{PSPACE}$.

    Given $L \in \mathbf{AP}$, there exists an ATM $M$ computing $L$. Note that can a configuration of $M$ can be representation in polynomial bits. Simulate $M$ with a Turing machine. For the non-deterministic choices, the TM simulate each subsequent configuration recursively. Since the ATM halts in polynomial-time, the depth of the recursion is polynomial. Thus, the simulation needs polynomial-space.
\end{exercise}

\begin{exercise}{5.6}
    证明SAT$\notin TISP(n^c,n^d),c(c+d)<2$

    反证法,不妨假设SAT$\in TISP(n^c,n^d)\to NTIME(n)\subseteq TISP(n^{c^{'}}.n^{d^{'}})$$c^{'}>c,d^{'}>d\to NTIME(n^\alpha)\subseteq TISP(n^{\alpha c^{'}}.n^{\alpha d^{'}})\subseteq \sum_2TIME(n^{\frac{\alpha(c^{'}+d^{'})}{2}})$$\subseteq NTIME(n^{\frac{c^{'}\alpha(c^{'}+d^{'})}{2}})\to \alpha \le \frac{c^{'}\alpha(c^{'}+d^{'}) }{2}\to c(c+d)>2$与已知矛盾,因此原命题得证。
\end{exercise}

\begin{exercise}{5.12}
We can prove it inductively. For $k>i$, let $L\in\Sigma_k^p$, there is a polynomial-time Turing machine $M$ and a polynomial $q$ such that
$$
x\in L\Leftrightarrow\exists u_1\in\{0,1\}^{q(x)}\forall u_2\in\{0,1\}^{q(x)}\cdots Q_k u_k\in\{0,1\}^{q(x)}M(x,u_1,\dots,u_k)=1,
$$
define $L'$ as follows:
$$
<x,u_1>\in L'\Leftrightarrow\forall u_2\in\{0,1\}^{q(x)}\cdots Q_k u_k\in\{0,1\}^{q(x)}M(x,u_1,\dots,u_k)=1,
$$
clearly, $L'\in\Pi_{k-1}^p$ so $L'$ is in $\Sigma_{k-1}^p$. Inductively, for any $k>i$ $\Sigma_k^p=\Sigma_i^p$, so $\textbf{PH}=\Sigma_i^p$.
\end{exercise}

\end{document}
