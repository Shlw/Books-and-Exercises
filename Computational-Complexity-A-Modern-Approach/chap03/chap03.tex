% !TeX encoding = UTF-8
% !TeX program = XeLaTeX
% !TeX spellcheck = LaTeX

\documentclass[a4paper]{article}

\usepackage[UTF8]{ctex}
\usepackage{amsmath,amsfonts,amssymb}
\usepackage{mathrsfs}
\usepackage{bm}
\usepackage{geometry}
\usepackage{ntheorem}
\usepackage{hyperref}
\usepackage[ruled]{algorithm2e}
\usepackage{caption,subcaption}

\geometry{left=2cm,right=2cm,top=2cm,bottom=2cm}

\def\UrlBreaks{\do\A\do\B\do\C\do\D\do\E\do\F\do\G\do\H\do\I\do\J\do\K\do\L\do\M\do\N\do\O\do\P\do\Q\do\R\do\S\do\T\do\U\do\V\do\W\do\X\do\Y\do\Z\do\[\do\\\do\]\do\^\do\_\do\`\do\a\do\b\do\c\do\d\do\e\do\f\do\g\do\h\do\i\do\j\do\k\do\l\do\m\do\n\do\o\do\p\do\q\do\r\do\s\do\t\do\u\do\v\do\w\do\x\do\y\do\z\do\0\do\1\do\2\do\3\do\4\do\5\do\6\do\7\do\8\do\9\do\.\do\@\do\\\do\/\do\!\do\_\do\|\do\;\do\>\do\]\do\)\do\,\do\?\do\'\do+\do\=\do\#}

\newtheorem{theorem}{Theorem}
\newtheorem{lemma}{Lemma}
\newtheorem{proposition}{Proposition}
\newtheorem{corollary}{Corollary}
\newtheorem{claim}{Claim}
\newtheorem{conjecture}{conjecture}
\newtheorem{definition}{Definition}
\newtheorem{construction}{Construction}
\newtheorem*{proof}{Proof}
\newtheorem*{answer}{Answer}
\newtheorem*{refute}{Refute}
\newtheorem*{example}{Example}
\newtheorem*{counterexample}{Counterexample}

\newenvironment{exercise}[1]{
	\par
	\noindent\textbf{Exercise #1.}\quad
}{
	\par
	\bigskip
}


\DeclareMathAccent{\widehat}{\mathord}{largesymbols}{"62}
\newcommand{\abs}[1]{\left| #1 \right|}
\newcommand{\pbra}[1]{\left( #1 \right)}
\newcommand{\cbra}[1]{\left\{ #1 \right\}}
\newcommand{\sbra}[1]{\left[ #1 \right]}
\newcommand{\bin}{\{0,1\}}

\title{Exercise Set --- Chapter $3$}
\date{}

\begin{document}

\maketitle


\begin{exercise}{3.1}
Reduction from Halt.

$$M_{1}=\{\lfloor M \rfloor : M \text{ is a machine that runs in } 100n^{2} + 200 \text{ time}, \text{ where } n\text{ is the size of input.}\}$$
Let $$M_{2} = \{\lfloor M \rfloor : M \text{ is a machine that runs in finite time }c.\}$$
 Then $M_{2}\leq_{T} M_{1}$.

Reduction: for a input $x$ of $M_{2}$. If $c< 100n^{2} + 200$, then add some steps to let $c = 100n^{2} + 200$, if $c > 100n^{2} + 200$, then add some additional input to let $x'=xy$, let the size of $x'$ equals to $100n'^{2} + 200$. Then execute $M_{1}$. 

Thus for a Halt instance $M(\alpha, x)$, we can execute $M_{2}$ to find that whether $M_{2}$ can halt in finite step $c$.
\end{exercise}


\begin{exercise}{3.2}
    NP is closed under polynomial-time reduction, but we shall prove $\text{SPACE}(n)$ is not.

    For any $L\in\text{SPACE}(n^2)$, define $L'=\cbra{x01^{|x|^2}\middle| x\in L}$.
    Then $L'\in\text{SPACE}(n)$ and the reduction $x\in L\to x01^{|x|^2}\in L'$ is polynomial-time.
    Assume $\text{NP}=\text{SPACE}(n)$, $L\in\text{SPACE}(n)$, which contradicts the Space Hierarchy Theorem.
\end{exercise}

\begin{exercise}{3.3}
    The language given in the proof of Theorem 3.7 is in $\textbf{EXP}$. To determine the membership for all string, exclude all undetermined string with shorter length than $\ell_i$ from $B$ after each stage-$i$, where $\ell_i$ is the length of the longest determined string in $B$ after stage-$i$. Note that $\ell_i = i + c$ is enough for the proof of Theorem 3.7 with some constant $c$. Thus, stage-$i$ can be done by simulating $M_i$ on $1^{\ell_i}$ within time-$\tilde O(2^i)$. Given string $x$, generate $B$ with $|x|$ stages and check the membership during the generating processing, which is obviously in \textbf{EXP}.
\end{exercise}

\begin{exercise}{3.4}
    \begin{itemize}
        \item [(a)] 给出superior的定义,判断DTIME($n^{1.1}$)是否superior DTIME($n$),根据前面时间层级定理的证明,我们可以证明$\exists L\in DTIME(n^{1.1}), \forall L^{'}\in DTIME(n),\exists x,|x|\in(n,n^2),M_L(x)\ne M_{L^{'}}(x)$因此DTIME($n^{1.1}$)是superior DTIME($n$)的
        \item [(b)] 我们并不能根据非确定性的时间层级定理证得同样的结论,因为在非确定性时间层级定理定理证明的过程中,只取反了一个元素,而不是像确定性时取反了所有元素,因此我们不能够保证取反的部分是在$[n,n^2]$内,因此无法根据前面证明得到相同结论。
    \end{itemize}
\end{exercise}

\begin{exercise}{3.8}
    Recall that given a language $B$, it's corresponding unary language is
    $$
    U_B=\{1^n| \exists x\in \{0,1\}^n, x\in B \}
    $$
   It's obvious that $U_B\in NP^B$. And
$$
Pr(NP^B=P^B)\leq Pr(U_B\in P^B)= \max_{poly(n) M^B} Pr(\forall n, M^B(1^n)=U_B(1^n))
$$
 $\max$ means we select the best possible polynomial time oracle Turing machine.

For a specific polynomial time machine $M^B$ on input $1^n$, it at most uses $B$ oracle for polynomial times and know at most $\frac{poly(n)}{2^n}$ fraction of $n$-bit string whether it belongs to $B$. Thus
\begin{align*}
Pr(M^B(1^n)=U_B(1^n))&=Pr(U_B(1^n)=0)Pr(M^B(1^n)=0|U_B(1^n)=0)+Pr(U_B(1^n)=1)Pr(M^B(1^n)=1|U_B(1^n)=1)\\
&\leq\frac{1}{2}\cdot 1+\frac{1}{2}\cdot \frac{poly(n)}{2^n}\\
&\leq\frac{2}{3};
\end{align*}
Thus
$$
 Pr(\forall n, M^B(1^n)=U_B(1^n))=\lim_{n\rightarrow \infty} c\cdot(\frac{2}{3})^n=0
$$
 Thus $Pr(NP^B=P^B)=0$.

 

\end{exercise}
\end{document}
