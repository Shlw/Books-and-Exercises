% !TeX encoding = UTF-8
% !TeX program = XeLaTeX
% !TeX spellcheck = LaTeX

\documentclass[a4paper]{article}

\usepackage{amsmath,amsfonts,amssymb}
\usepackage{mathrsfs}
\usepackage{bm}
\usepackage{geometry}
\usepackage{ntheorem}
\usepackage{hyperref}
\usepackage[ruled]{algorithm2e}
\usepackage{caption,subcaption}

\geometry{left=2cm,right=2cm,top=2cm,bottom=2cm}

\def\UrlBreaks{\do\A\do\B\do\C\do\D\do\E\do\F\do\G\do\H\do\I\do\J\do\K\do\L\do\M\do\N\do\O\do\P\do\Q\do\R\do\S\do\T\do\U\do\V\do\W\do\X\do\Y\do\Z\do\[\do\\\do\]\do\^\do\_\do\`\do\a\do\b\do\c\do\d\do\e\do\f\do\g\do\h\do\i\do\j\do\k\do\l\do\m\do\n\do\o\do\p\do\q\do\r\do\s\do\t\do\u\do\v\do\w\do\x\do\y\do\z\do\0\do\1\do\2\do\3\do\4\do\5\do\6\do\7\do\8\do\9\do\.\do\@\do\\\do\/\do\!\do\_\do\|\do\;\do\>\do\]\do\)\do\,\do\?\do\'\do+\do\=\do\#}

\newtheorem{theorem}{Theorem}
\newtheorem{lemma}{Lemma}
\newtheorem{proposition}{Proposition}
\newtheorem{corollary}{Corollary}
\newtheorem{claim}{Claim}
\newtheorem{conjecture}{conjecture}
\newtheorem{definition}{Definition}
\newtheorem{construction}{Construction}
\newtheorem*{proof}{Proof}
\newtheorem*{answer}{Answer}
\newtheorem*{refute}{Refute}
\newtheorem*{example}{Example}
\newtheorem*{counterexample}{Counterexample}

\newenvironment{exercise}[1]{
	\par
	\noindent\textbf{Exercise #1.}\quad
}{
	\par
	\bigskip
}


\DeclareMathAccent{\widehat}{\mathord}{largesymbols}{"62}
\newcommand{\abs}[1]{\left| #1 \right|}
\newcommand{\pbra}[1]{\left( #1 \right)}
\newcommand{\cbra}[1]{\left\{ #1 \right\}}
\newcommand{\sbra}[1]{\left[ #1 \right]}
\newcommand{\bin}{\{0,1\}}

\title{Exercise Set --- Chapter $3$}
\date{}

\begin{document}

\maketitle

\begin{exercise}{3.1}
    hhh
\end{exercise}

\begin{exercise}{3.8}
Similary to proof in Theorem 3.7, we use the property of unary language $U_B$. Recall that given a language $B$, it's corresponding unary language is
$$
U_B=\{1^n: \text{some string of length $n$ is in $B$}\}
$$
It'a easy to see $U_B\in NP^B$ by regarding the execution of nondeterministic Turing machine as enumerating all input of length $n$. So we next we prove with high probability, $U_B\not\in P^B$. It suffice to show that in polynomial time, any DTM $M$ on input $1^n$ can check at most $\frac{poly(n)}{2^n}$ strings thus output the right anwser with probability $\frac{poly(n)}{2^n}=o(1)$. Note that the condition ``with probability $1/2$, $B$ has no string of length $n$`` is necessary, otherwise simply $U_B=\{1^n, \forall n\}$.

 

\end{exercise}
\end{document}
