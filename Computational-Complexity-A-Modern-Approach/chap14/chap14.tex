
% !TeX encoding = UTF-8
% !TeX program = XeLaTeX
% !TeX spellcheck = LaTeX

\documentclass[a4paper]{article}

\usepackage{amsthm,amsmath,amsfonts,amssymb}
\usepackage{mathrsfs}
\usepackage{bm}
\usepackage{geometry}
% \usepackage{ntheorem}
\usepackage{hyperref}
\usepackage[ruled]{algorithm2e}
\usepackage{caption,subcaption}

\geometry{left=2cm,right=2cm,top=2cm,bottom=2cm}

\def\UrlBreaks{\do\A\do\B\do\C\do\D\do\E\do\F\do\G\do\H\do\I\do\J\do\K\do\L\do\M\do\N\do\O\do\P\do\Q\do\R\do\S\do\T\do\U\do\V\do\W\do\X\do\Y\do\Z\do\[\do\\\do\]\do\^\do\_\do\`\do\a\do\b\do\c\do\d\do\e\do\f\do\g\do\h\do\i\do\j\do\k\do\l\do\m\do\n\do\o\do\p\do\q\do\r\do\s\do\t\do\u\do\v\do\w\do\x\do\y\do\z\do\0\do\1\do\2\do\3\do\4\do\5\do\6\do\7\do\8\do\9\do\.\do\@\do\\\do\/\do\!\do\_\do\|\do\;\do\>\do\]\do\)\do\,\do\?\do\'\do+\do\=\do\#}

\newtheorem{theorem}{Theorem}
\newtheorem{lemma}{Lemma}
\newtheorem{proposition}{Proposition}
\newtheorem{corollary}{Corollary}
\newtheorem{claim}{Claim}
\newtheorem{conjecture}{conjecture}
\newtheorem{definition}{Definition}
\newtheorem{construction}{Construction}
\newtheorem*{answer}{Answer}
\newtheorem*{refute}{Refute}
\newtheorem*{example}{Example}
\newtheorem*{counterexample}{Counterexample}

\newenvironment{exercise}[1]{
	\par
	\noindent\textbf{Exercise #1.}\quad
}{
	\par
	\bigskip
}


\DeclareMathAccent{\widehat}{\mathord}{largesymbols}{"62}
\newcommand{\abs}[1]{\left| #1 \right|}
\newcommand{\pbra}[1]{\left( #1 \right)}
\newcommand{\cbra}[1]{\left\{ #1 \right\}}
\newcommand{\sbra}[1]{\left[ #1 \right]}
\newcommand{\bin}{\{0,1\}}


\def\sD{\textsf{D}}
\def\sE{\textsf{E}}
\def\cD{\mathcal{D}}

\title{Exercise Set --- Chapter 14}
\date{}

\begin{document}
    \maketitle

    \begin{exercise}{14.9}
        Assume $L$ can be computed by a width-$c$ ($c=O(1)$) depth-$d$ branching program. 
        Assume the $i$-th layer queries the $a_i$-th bit. Then the $i$-th layer can be seen as two functions 
        $f_i,g_i:[c]\to[c]$, where we go along $f_i$ iff $x_{a_i}=0$.

        Now we construct an $O(\log d)$ circuit $\mathcal C$ for $L$. In the first $O(1)$ layers, $\mathcal C$ concurrently evaluates
        $h_i^1:[c]\to[c]$, where 
        $$
        h_i^1=\begin{cases}
            f_i&x_{a_i}=0\\
            g_i&x_{a_i}=1.
        \end{cases}
        $$
        Then with $O(\log d)$ layers, $\mathcal C$ computes the composition of $h_i$'s, i.e., compute
        $h_i^k=h_{2i}^{k-1}\circ h_{2i-1}^{k-1}$ for $k\in[\log d]$.
        At last, $\mathcal C$ simply evaluates its value as we have computed $h_d^1\circ\cdots\circ h_2^1\circ h_1^1$ like the evaluation of the original branching program.
    \end{exercise}

    \begin{exercise}{14.10}
        Given a DAG $G = (V, A)$ with $|V| = n, |A| = m$ and the depth of $d$, select an arbitrary mapping $f: V \to [d]$ s.t., for each $(u \to v) \in A$, $c(u) < c(v)$ holds. Label the edges as 
        \[
           (u \to v) \mapsto \lfloor \log(c(u)\oplus c(v))\rfloor.
        \]
        Suppose the path containing at least $2^\ell$ vertices contains at least $\ell$ distinct edge-labels. Assume a path $P$ contains $2^{\ell + 1}$ vertices whose labels are gathered by $T = c(S) = \{a_1, \ldots, a_{2^{\ell + 1}}\}$ where $a_1 < \ldots < a_{2^{\ell + 1}}$. W.l.o.g, assume there exist $i$ s.t. $\lfloor\log a_i\rfloor \neq \lfloor\log a_{i + 1}\rfloor$ where $i$ is the maximum one. Thus, $\lfloor \log a_{i + 1}\rfloor$ appears in an edges of the path and all the others are less than it. Note that either the subpath containing the first $i$ vertices of $P$, or the one containing the last $2^{\ell + 1} - i$ has $2^{\ell}$ vertices, which means there are $\ell$ distinct and less than $\lfloor \log a_{i + 1}\rfloor$ edge-labels in either the first- or the second-part of $P$ due to the induction assumption. Remove all the edges labeled by the $k$ ``least popular'' numbers, which will be no more than $km/\lceil\log d\rceil$. After that, the longest path will be shorter than $2^{\ell - k} \leq d/2^{k - 1}$ as we want.
    \end{exercise}
\end{document}
