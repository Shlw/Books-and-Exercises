% !TeX encoding = UTF-8
% !TeX program = XeLaTeX
% !TeX spellcheck = LaTeX

\documentclass[a4paper]{article}

\usepackage{amsthm,amsmath,amsfonts,amssymb}
\usepackage{mathrsfs}
\usepackage{bm}
\usepackage{geometry}
% \usepackage{ntheorem}
\usepackage{hyperref}
\usepackage[ruled]{algorithm2e}
\usepackage{caption,subcaption}

\geometry{left=2cm,right=2cm,top=2cm,bottom=2cm}

\def\UrlBreaks{\do\A\do\B\do\C\do\D\do\E\do\F\do\G\do\H\do\I\do\J\do\K\do\L\do\M\do\N\do\O\do\P\do\Q\do\R\do\S\do\T\do\U\do\V\do\W\do\X\do\Y\do\Z\do\[\do\\\do\]\do\^\do\_\do\`\do\a\do\b\do\c\do\d\do\e\do\f\do\g\do\h\do\i\do\j\do\k\do\l\do\m\do\n\do\o\do\p\do\q\do\r\do\s\do\t\do\u\do\v\do\w\do\x\do\y\do\z\do\0\do\1\do\2\do\3\do\4\do\5\do\6\do\7\do\8\do\9\do\.\do\@\do\\\do\/\do\!\do\_\do\|\do\;\do\>\do\]\do\)\do\,\do\?\do\'\do+\do\=\do\#}

\newtheorem{theorem}{Theorem}
\newtheorem{lemma}{Lemma}
\newtheorem{proposition}{Proposition}
\newtheorem{corollary}{Corollary}
\newtheorem{claim}{Claim}
\newtheorem{conjecture}{conjecture}
\newtheorem{definition}{Definition}
\newtheorem{construction}{Construction}
\newtheorem*{answer}{Answer}
\newtheorem*{refute}{Refute}
\newtheorem*{example}{Example}
\newtheorem*{counterexample}{Counterexample}

\newenvironment{exercise}[1]{
	\par
	\noindent\textbf{Exercise #1.}\quad
}{
	\par
	\bigskip
}


\DeclareMathAccent{\widehat}{\mathord}{largesymbols}{"62}
\newcommand{\abs}[1]{\left| #1 \right|}
\newcommand{\pbra}[1]{\left( #1 \right)}
\newcommand{\cbra}[1]{\left\{ #1 \right\}}
\newcommand{\sbra}[1]{\left[ #1 \right]}
\newcommand{\bin}{\{0,1\}}

\title{Exercise Set --- Chapter 8}
\date{}

\begin{document}

    \maketitle

	\begin{exercise}{8.1}
	(a) It is obvious that $IP\subseteq IP'$. To prove the other hand. We first observe that if $L \in IP'[2]$, then we have iteration process:

	Give $x$ and $r,r'$, if $\exists a_{1}, a_{2}$ subject to
	\begin{itemize}
	\item $f(x, r) = a_{1}$
	\item $g(x, a_{1}, r') = a_{2}$
	\item $a_{2} = 1\Rightarrow $ ACC.
	\item $a_{2} = 0 \Rightarrow $ REJ.
	\end{itemize}

	Firstly, when prover knows the answer, then $r, r'$ can be omitted. In the other hand, when prover don't know answer, it has a strategy $r'$ or without strategy would have the same result since verifier can check whether the prover is honest. That is, there exists $u(r, r')$ such that

	$\exists a_{1}, a_{2}$ subject to
	\begin{itemize}
	\item $f(x, u(r,r')) = a_{1}$
	\item $g(x) = a_{2}$
	\item $a_{2} = 1\Rightarrow $ ACC.
	\item $a_{2} = 0 \Rightarrow $ REJ.
	\end{itemize}
	  Thus $IP'\subseteq IP$.
	%In other hand,$\exists  f$ can be computed polynomially, for any $g$ and $x$,
	%\begin{itemize}
	%\item $x \in L \Rightarrow $ $\Pr(ACC)\geq 1 - \frac{1}{2^{n}}$. 
	%\item $x \not\in L \Rightarrow $ for any $r'$, $\Pr(REJ)\geq 1 - \frac{1}{2^{n}}$.  
	%\end{itemize}

	(2) Prove $IP\subseteq PSPACE$.

	Review the definition of PSPACE, if $L\in PSPACE$, we need to find the prover in $poly(|x|)$ space. $L\in IP$, given any verifier, we can compute the optimal prover in $poly(|x|)$ space, i.e., $L\in PSPACE$.

	(3) IP' : changing $2/3$ in $(8.2)$ to 1.

	Review the proof of Theorem 8.19: $IP = PSPACE$. We prove that there is a protocol for TQBF that uses public coins, if $x\in TQBF$, then there is a prover that makes the verifier accept with probability 1. Thus we can change the successful probability to $1$ in IP.

	(4) Changing $1/3$ in $(8.3)$ to 1.

	That is, $\exists V (f), r$, for any $P (g)$
	\begin{itemize}
	\item $x \in L \Rightarrow \exists P\Pr[\text{out}_{V}\langle{V,P}\rangle(x) = 1] = 1$. 
	\item $x \not\in L \Rightarrow \forall P \Pr[\text{out}_{V}\langle{V,P}\rangle(x) = 1]  =0$.  
	\end{itemize}
	$\Rightarrow IP' = NP$.
	\end{exercise}

	\begin{exercise}{8.2}
	    Since $\textbf{IP}'\subseteq\textbf{IP}=\textbf{PSPACE}$, it suffices to show $\textbf{PSPACE}\subseteq\textbf{IP}'$.
	    The protocol designed for TQBF naturally satisfies the requirement.
	\end{exercise}

	\begin{exercise}{8.9}
	Assuming the time complexity is $O(n^c)$, oracles for the same input size will not be used at the same time, so the space complexity is $O(\sum_{k=1}^n k^c)=O(n^{c+1})$.
	\end{exercise}
	
	\begin{exercise}{8.11}
		Give the definition of $MIP$ and probabilistic oracle machine:\\
		
		$P_1,...,P_k$ and $V$ form a multi-prover interactive protocol for a language $L$ if :

		\begin{enumerate}
			\item[(1)] If $x\in L$, $\exists P_1,\ldots,P_k$, have $Pr[\text{$P_1,\ldots,P_k$ and $V$ on $x$ accecpt}]>1-2^{-n}$;
			\item[(2)] If $x\notin L$, $\forall P_1,\ldots,P_k$, have $Pr[\text{$P_1,\ldots,P_k$ and $V$ on $x$ accecpt}]<2^{-n}$
		\end{enumerate}
		
		$MIP$ is the class of all languages which have multi-prover interactive protocols.\\
		
		Let $M$ be a probabilistic ploy-time Turing machine with access to an oracle $O$,a language L is accepted by M iff 

		\begin{enumerate}
			\item[(1)] For all $x\in L$, $\exists O$,have $Pr[M^O(x)=1]>1-2^{-n}$;
			\item[(2)] For all $x\notin L$, $\forall O$,have $Pr[M^O(x)=1]]<2^{-n}$
		\end{enumerate}

		By Lemma 1 and Lemma 2, we can proof $MIP \subseteq NEXP $.\\

		\begin{lemma}
			If a language $L$ is accepted by a  a multi-prover interactive protocol , then $L$ can be accepted by a probabilistic oracle machine.
		\end{lemma}

		\begin{proof}
			Suppose now that $L$ is accepted by a multi-prover interactive protocol. Then define $M$ as follows: Have $M$ simulate $V$ with $M$ remembering all messages. When $V$ sends the $j$th message to the $i$th prover, $M$ asks the oracle the question$(i,j,l,\beta_i1,...,\beta_ij)$ properly encoded and uses the response as the $l$th bit of the $j$th response from prover $i$ where $\beta_i1,...,\beta_ij$ is everything prover $i$ can see in the point.

			\begin{itemize}
				\item[(a)] If $x\in L$, then the oracle $O$ could convince $M$ to accept by just encoding each prover's answer to each question.
				\item[(b)] If an oracle $O$ could convince $M$ to accept a string x then the provers could convince the verifier to accept by just using that $O$ to create their responses.
			\end{itemize}
		\end{proof}

		\begin{lemma}
			If a language $L$ is accepted by  a probabilistic oracle machine, then $L$ can be computed in nondeterministic exponential time.
		\end{lemma}

		\begin{proof}
			By lemma 1,$M$ accept $L$ with $M$ using time $n^k$ on inputs length $n$ for some $k>0$.We create a nondeterministic exponential time machine to accept $L$ as follow:
		
			On input $x$($|x|=n$),guess the value of the oracle $O$ on all questions of length at most $n^k$.Note $M(k)$ can only ask oracle questions of lendth no longer than $n^k$,there are $2^{n^k +1} -1)$ such questions.For $r$ is a string of length $n^k$, let $f(x,O,r)=1$ if $M$ on input $x$ accepts using random coin tosses $r$ and getting the oracle answers from $O$ and $f(x,O,r)=0$ otherwise. Compute
			
					$S= \sum_{r \in \{0,1\}^{n^k}}f(x,O,r)$

			Accept if $S>2^{n^k}/2$.
			
			By the definition of probabilistic oracle machine,for $x\in L$ there exist a setting of the oracle such that $S\ge(1-2^{-n})2^{n^k}$. If $x \notin L$ then for any setting of the oracle ,$S\leq 2^{-n}2^{n^k}$.
		\end{proof}
	\end{exercise}

\end{document}
