% !TeX encoding = UTF-8
% !TeX program = XeLaTeX
% !TeX spellcheck = LaTeX

\documentclass[a4paper]{article}

\usepackage{amsmath,amsfonts,amssymb}
\usepackage{mathrsfs}
\usepackage{bm}
\usepackage{geometry}
\usepackage{ntheorem}
\usepackage{hyperref}
\usepackage[ruled]{algorithm2e}
\usepackage{caption,subcaption}

\geometry{left=2cm,right=2cm,top=2cm,bottom=2cm}

\def\UrlBreaks{\do\A\do\B\do\C\do\D\do\E\do\F\do\G\do\H\do\I\do\J\do\K\do\L\do\M\do\N\do\O\do\P\do\Q\do\R\do\S\do\T\do\U\do\V\do\W\do\X\do\Y\do\Z\do\[\do\\\do\]\do\^\do\_\do\`\do\a\do\b\do\c\do\d\do\e\do\f\do\g\do\h\do\i\do\j\do\k\do\l\do\m\do\n\do\o\do\p\do\q\do\r\do\s\do\t\do\u\do\v\do\w\do\x\do\y\do\z\do\0\do\1\do\2\do\3\do\4\do\5\do\6\do\7\do\8\do\9\do\.\do\@\do\\\do\/\do\!\do\_\do\|\do\;\do\>\do\]\do\)\do\,\do\?\do\'\do+\do\=\do\#}

\newtheorem{theorem}{Theorem}
\newtheorem{lemma}{Lemma}
\newtheorem{proposition}{Proposition}
\newtheorem{corollary}{Corollary}
\newtheorem{claim}{Claim}
\newtheorem{conjecture}{conjecture}
\newtheorem{definition}{Definition}
\newtheorem{construction}{Construction}
\newtheorem*{proof}{Proof}
\newtheorem*{answer}{Answer}
\newtheorem*{refute}{Refute}
\newtheorem*{example}{Example}
\newtheorem*{counterexample}{Counterexample}

\newenvironment{exercise}[1]{
	\par
	\noindent\textbf{Exercise #1.}\quad
}{
	\par
	\bigskip
}


\DeclareMathAccent{\widehat}{\mathord}{largesymbols}{"62}
\newcommand{\abs}[1]{\left| #1 \right|}
\newcommand{\pbra}[1]{\left( #1 \right)}
\newcommand{\cbra}[1]{\left\{ #1 \right\}}
\newcommand{\sbra}[1]{\left[ #1 \right]}
\newcommand{\bin}{\{0,1\}}

\title{Exercise Set --- Chapter $8$}
\date{}

\begin{document}

\maketitle

\begin{exercise}{8.1}
(a) It is obvious that $IP\subseteq IP'$. To prove the other hand. We first observe that if $L \in IP'[2]$, then we have iteration process:

Give $x$ and $r,r'$, if $\exists a_{1}, a_{2}$ subject to
\begin{itemize}
\item $f(x, r) = a_{1}$
\item $g(x, a_{1}, r') = a_{2}$
\item $a_{2} = 1\Rightarrow $ ACC.
\item $a_{2} = 0 \Rightarrow $ REJ.
\end{itemize}

Firstly, when prover knows the answer, then $r, r'$ can be omitted. In the other hand, when prover don't know answer, it has a strategy $r'$ or without strategy would have the same result since verifier can check whether the prover is honest. That is, there exists $u(r, r')$ such that

$\exists a_{1}, a_{2}$ subject to
\begin{itemize}
\item $f(x, u(r,r')) = a_{1}$
\item $g(x) = a_{2}$
\item $a_{2} = 1\Rightarrow $ ACC.
\item $a_{2} = 0 \Rightarrow $ REJ.
\end{itemize}
  Thus $IP'\subseteq IP$.
%In other hand,$\exists  f$ can be computed polynomially, for any $g$ and $x$,
%\begin{itemize}
%\item $x \in L \Rightarrow $ $\Pr(ACC)\geq 1 - \frac{1}{2^{n}}$. 
%\item $x \not\in L \Rightarrow $ for any $r'$, $\Pr(REJ)\geq 1 - \frac{1}{2^{n}}$.  
%\end{itemize}

(2) Prove $IP\subseteq PSPACE$.

Review the definition of PSPACE, if $L\in PSPACE$, we need to find the prover in $poly(|x|)$ space. $L\in IP$, given any verifier, we can compute the optimal prover in $poly(|x|)$ space, i.e., $L\in PSPACE$.

(3) IP' : changing $2/3$ in $(8.2)$ to 1.

Review the proof of Theorem 8.19: $IP = PSPACE$. We prove that there is a protocol for TQBF that uses public coins, if $x\in TQBF$, then there is a prover that makes the verifier accept with probability 1. Thus we can change the successful probability to $1$ in IP.

(4) Changing $1/3$ in $(8.3)$ to 1.

That is, $\exists V (f), r$, for any $P (g)$
\begin{itemize}
\item $x \in L \Rightarrow \exists P\Pr[\text{out}_{V}\ket{V,P}(x) = 1] = 1$. 
\item $x \not\in L \Rightarrow \forall P \Pr[\text{out}_{V}\ket{V,P}(x) = 1]  =0$.  
\end{itemize}
$\Rightarrow IP' = NP$.
\end{exercise}

\begin{exercise}{8.2}
    Since $\textbf{IP}'\subseteq\textbf{IP}=\textbf{PSPACE}$, it suffices to show $\textbf{PSPACE}\subseteq\textbf{IP}'$.
    The protocol designed for TQBF naturally satisfies the requirement.
\end{exercise}

\end{document}
