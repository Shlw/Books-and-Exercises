% !TeX encoding = UTF-8
% !TeX program = XeLaTeX
% !TeX spellcheck = LaTeX

\documentclass[a4paper]{article}

\usepackage{amsmath,amsfonts,amssymb}
\usepackage{mathrsfs}
\usepackage{bm}
\usepackage{geometry}
\usepackage{ntheorem}
\usepackage{hyperref}
\usepackage{caption,subcaption}

\geometry{left=2cm,right=2cm,top=2cm,bottom=2cm}

\def\UrlBreaks{\do\A\do\B\do\C\do\D\do\E\do\F\do\G\do\H\do\I\do\J\do\K\do\L\do\M\do\N\do\O\do\P\do\Q\do\R\do\S\do\T\do\U\do\V\do\W\do\X\do\Y\do\Z\do\[\do\\\do\]\do\^\do\_\do\`\do\a\do\b\do\c\do\d\do\e\do\f\do\g\do\h\do\i\do\j\do\k\do\l\do\m\do\n\do\o\do\p\do\q\do\r\do\s\do\t\do\u\do\v\do\w\do\x\do\y\do\z\do\0\do\1\do\2\do\3\do\4\do\5\do\6\do\7\do\8\do\9\do\.\do\@\do\\\do\/\do\!\do\_\do\|\do\;\do\>\do\]\do\)\do\,\do\?\do\'\do+\do\=\do\#}

\newtheorem{theorem}{Theorem}
\newtheorem{lemma}{Lemma}
\newtheorem{proposition}{Proposition}
\newtheorem{corollary}{Corollary}
\newtheorem{claim}{Claim}
\newtheorem{conjecture}{conjecture}
\newtheorem{definition}{Definition}
\newtheorem{construction}{Construction}
\newtheorem*{exercise}{Exercise}
\newtheorem*{proof}{Proof}
\newtheorem*{answer}{Answer}
\newtheorem*{refute}{Refute}
\newtheorem*{example}{Example}
\newtheorem*{counterexample}{Counterexample}

\DeclareMathAccent{\widehat}{\mathord}{largesymbols}{"62}
\newcommand{\abs}[1]{\left| #1 \right|}
\newcommand{\pbra}[1]{\left( #1 \right)}
\newcommand{\cbra}[1]{\left\{ #1 \right\}}
\newcommand{\sbra}[1]{\left[ #1 \right]}
\newcommand{\bin}{\{0,1\}}

\title{Exercise Set --- Chapter $1$}
\date{}

\begin{document}

\maketitle

\begin{exercise}[1.3]
	in \textbf{Claim 1.6}, we want to proof that for every k-tape TM $M$,there always exist a single-tape TM $\widetilde{M}$ can simulate $M$.The way to map the k-tape to a single tape:
		$$
        \begin{matrix}		
			 1-tape&\quad location\quad 1, k+1, 2k+1\\			
			 2-tape&\quad location\quad 2, k+2, 2k+2\\			
			 \dots&\dots\\			
			 k-tape&location\quad k, 2k, 3k\\			
		\end{matrix}
        $$
	Because $M$ can compute $f$ in $T(n)$ ,so the tape in $M$ must be short than $T(n)$, trivial show the tape in $\widetilde{M}$ is no longer than $n+1+kT(n)$ (the first n+1 locations locate the input).
	For every symbol a in $M's$ alphabet, $\widetilde{M}$ will contain both the symbol $a$ and $\widehat{a}$, indicating the corresponding heads of $M$. 
	In the beginning, $\widetilde{M}$ would take $O(n^2)$ steps to cope the input bit by bit into the rest of the tape.
	To simulate one step of $M$ ($T(n)$ steps together):
    \begin{enumerate}
        \item move the head to the left (the location 1),then sweeps the tape left-to-right direction to get the k symbols that are marked by $\widehat{ }$. \quad ($2kT(n)$)
        \item use $M's$ transition function to determine the new state ,symbols and head moves.
        \item sweep the tape back in the right- to -left direction to update the encode, and move the head to the right place (the location of the next symbol to be read). \quad ($2kT(n)$)
    \end{enumerate}
	In the 3 step, there some additional steps maybe needed for updating head movement. 
	So for the at most $T(n)$ steps of $M$,$\widetilde{M}$ perform at most $5kT(n)^2$. 
\end{exercise}

\begin{exercise}[1.7]
    By \textbf{Claim 1.6}, it suffices to show $f$ can be computed in $O(T(n))$ with a three-tape TM $M$.
    Define bijection $\sigma:\mathbb Z^2\to\mathbb N$, where
    $$
    \begin{matrix}
        \sigma(0,0)=0 & \sigma(1,0)=1 & \sigma(0,1)=2 & \sigma(-1,0)=3 & \sigma(0,-1)=4\\
        \sigma(2,0)=5 & \sigma(1,1)=6 & \sigma(0,2)=7 & \sigma(-1,1)=8 & \sigma(-2,0)=9\\
        \sigma(-1,-1)=10 & \sigma(0,-2)=11 & \sigma(1,-1)=12 & \sigma(3,0)=13 & \dots\dots\\
    \end{matrix}
    $$
    Therefore $\sigma$ is time-constructible in $O(T(n))$.
    $M$ is constructed to maintain current coordinate $(x,y)$ in the second tape 
    and calculate $\sigma(x,y)$ in the third tape.
    Then it simulates the action of two-dimensional TM in the first tape.
\end{exercise}

\begin{exercise}[1.13]
	\begin{itemize}
		\item[(a)] According to Section 1.5.2, 
$$DIVIDES(x,y)=\exists_k: y=x\times k$$
$$PRIME(y)=\forall_x (x=1)\vee(x=y)\vee(\neg DIVIDES(x,y))$$

$$
BIT(n,i)=\exists_p: DIVIDES(n,p)\wedge ((C+i)^3<p<(C+i+1)^3) \wedge (\forall_{p'}: p'\leq (C+i)^3\vee p'\geq p \vee \neg PRIME(p'))
$$

		\item[(b)] $COMPARE(n,m,i,j)=\forall_i (BIT(n,i)\wedge BIT(m,i)\vee \neg BIT(n,i)\wedge \neg BIT(n,i))$
	\item[(c)] We need to restore the information of the string in the tape, the current state, the position of the read-write head. We encode those information into a $0,1$ string. We use $111$ to represent the seperator and use $01$ to represent character $0$ and $10$ for character $1$. For example, suppose the original alphabet is $\{0,1\}$, and at a specific time, the string in the tape is $001$, then we encode this into $010110$. The encoding for the left two elements are similar. And notice that $3$ consecutive $1$ will only emerge as a separator.
\item[(d)] According to (c), we can encode the inicial configuration into a string. By using method in the description of Excercise 13, we can encode this string into an integer $m_{M,x}$. Thus 
$$INIT_{M,x}(n)=\forall_i COMPARE(n,m,i,i)$$.
\item[(e)] Suppose $\{n_i\}$ encoded all the string $<b,state>$ such that $M$ will halt at this configuration. Then 
$$
HALT_M(n)=\exists_i \forall_x COMPARE(n,n_i,x,x)
$$   
\item[(f)] Precompute all the string $x,y$ where $x$ represent a configuration of $M$ and $y$ is the configuration obtained by a single computation step of $M$. Suppose $\{n_i\}$,$\{n_i\}$ are the corresposing integers edcoded $x,y$.
Then 
$$
NEXT(n,m)=\exists_i\forall_x COMPARE(n,n_i,x,x)\wedge COMPARE(m,m_i,x,x)
$$
\item[(g)] 
$$
VALID_M(m,t)=\exists_i\forall_j (j>t-1)\vee(j<1)\vee NEXT(m_j,m_{j+1})
$$
\item[(h)] 
$$
HALT_{M,x}=\exists_m VALID(m,t)\wedge INIT_{M,x}(m[1])\wedge HALT_M(m[t])
$$
\item[(i)] 
$$HALT(M,x)=\exists_t HALT_{M,x}(t)$$.



		\item[(b)] $COMPARE(n,m,i,j)=(BIT(n,i)\wedge BIT(m,i)\vee \neg BIT(n,i)\wedge \neg BIT(n,i))\wedge(BIT(n,j)\wedge BIT(m,j)\vee \neg BIT(n,j)\wedge \neg BIT(n,j))$
	\item[(c)] We need to restore the information of the string in the tape, the current state, the position of the read-write head. We encode those information into a $0,1$ string. We use $111$ to represent the separator and use $01$ to represent character $0$ and $10$ for character $1$. For example, suppose the original alphabet is $\{0,1\}$, and at a specific time, the string in the tape is $001$, then we encode this into $010110$. The encoding for the left two elements are similar. And notice that $3$ consecutive $1$ will only emerge as a separator.
\item[(d)] 

	\end{itemize}
\end{exercise}

\begin{exercise}[1.14]
\begin{enumerate}		
	\item[\textbf{(a)}] 
		The language is in P.
		Choose one vertex and start DFS in the graph, get the DFS tree,judge whether the scale of the tree is $|V|$ . If equal, then the graph is connected.It's trivial that can be computed in ploy-time.
		\item[\textbf{(b)}] 
		The language is in P.
		Choose one edge$(u,v)$ in G.Then visit all $u's$ adjacent points $t$,judge whether $(t,v)\in E$ . Repeat until all the edges in G have been chosen. The time is $O(n^3)$.
		\item[\textbf{(c)}] 
		The language is P.
		First conducting DFS in G, get the DFS tree and do 2-vertex coloring in the tree.After coloring , check it .
		Then add the other vertices, which is not in the tree ,to the 2-vertex coloring.If success, G is a bipartite.The algorithm is poly-time.
		\item[\textbf{(d)}] 
		The language is P.Choose one vertex and start DFS in the graph, get the DFS tree,judge whether the scale of the tree is $|V|$ and there is not residual edge.It is poly-time.
\end{enumerate}
\end{exercise}

\end{document}
