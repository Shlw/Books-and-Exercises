% !TeX encoding = UTF-8
% !TeX program = XeLaTeX
% !TeX spellcheck = LaTeX

\documentclass[a4paper]{article}

\usepackage{amsmath,amsfonts,amssymb}
\usepackage{mathrsfs}
\usepackage{bm}
\usepackage{geometry}
\usepackage{ntheorem}
\usepackage{hyperref}
\usepackage{caption,subcaption}

\geometry{left=2cm,right=2cm,top=2cm,bottom=2cm}

\def\UrlBreaks{\do\A\do\B\do\C\do\D\do\E\do\F\do\G\do\H\do\I\do\J\do\K\do\L\do\M\do\N\do\O\do\P\do\Q\do\R\do\S\do\T\do\U\do\V\do\W\do\X\do\Y\do\Z\do\[\do\\\do\]\do\^\do\_\do\`\do\a\do\b\do\c\do\d\do\e\do\f\do\g\do\h\do\i\do\j\do\k\do\l\do\m\do\n\do\o\do\p\do\q\do\r\do\s\do\t\do\u\do\v\do\w\do\x\do\y\do\z\do\0\do\1\do\2\do\3\do\4\do\5\do\6\do\7\do\8\do\9\do\.\do\@\do\\\do\/\do\!\do\_\do\|\do\;\do\>\do\]\do\)\do\,\do\?\do\'\do+\do\=\do\#}

\newtheorem{theorem}{Theorem}
\newtheorem{lemma}{Lemma}
\newtheorem{proposition}{Proposition}
\newtheorem{corollary}{Corollary}
\newtheorem{claim}{Claim}
\newtheorem{conjecture}{conjecture}
\newtheorem{definition}{Definition}
\newtheorem{construction}{Construction}
\newtheorem*{exercise}{Exercise}
\newtheorem*{proof}{Proof}
\newtheorem*{answer}{Answer}
\newtheorem*{refute}{Refute}
\newtheorem*{example}{Example}
\newtheorem*{counterexample}{Counterexample}

\DeclareMathAccent{\widehat}{\mathord}{largesymbols}{"62}
\newcommand{\abs}[1]{\left| #1 \right|}
\newcommand{\pbra}[1]{\left( #1 \right)}
\newcommand{\cbra}[1]{\left\{ #1 \right\}}
\newcommand{\sbra}[1]{\left[ #1 \right]}
\newcommand{\bin}{\{0,1\}}

\title{Exercise Set --- Chapter $1$}
\date{}

\begin{document}

\maketitle

\begin{exercise}[1.7]
    By \textbf{Claim 1.6}, it suffices to show $f$ can be computed in $O(T(n))$ with a three-tape TM $M$.
    Define bijection $\sigma:\mathbb Z^2\to\mathbb N$, where
    $$
    \begin{matrix}
        \sigma(0,0)=0 & \sigma(1,0)=1 & \sigma(0,1)=2 & \sigma(-1,0)=3 & \sigma(0,-1)=4\\
        \sigma(2,0)=5 & \sigma(1,1)=6 & \sigma(0,2)=7 & \sigma(-1,1)=8 & \sigma(-2,0)=9\\
        \sigma(-1,-1)=10 & \sigma(0,-2)=11 & \sigma(1,-1)=12 & \sigma(3,0)=13 & \dots\dots\\
    \end{matrix}
    $$
    Therefore $\sigma$ is time-constructible in $O(T(n))$.
    $M$ is constructed to maintain current coordinate $(x,y)$ in the second tape 
    and calculate $\sigma(x,y)$ in the third tape.
    Then it simulates the action of two-dimensional TM in the first tape.
\end{exercise}

\begin{exercise}[1.8]
	hhh
\end{exercise}

\begin{exercise}[1.13]
	hhh+2
\end{exercise}

\end{document}
