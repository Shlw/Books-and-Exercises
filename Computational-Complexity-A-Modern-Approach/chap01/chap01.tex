% !TeX encoding = UTF-8
% !TeX program = XeLaTeX
% !TeX spellcheck = LaTeX

\documentclass[a4paper]{article}

\usepackage{amsmath,amsfonts,amssymb}
\usepackage{mathrsfs}
\usepackage{bm}
\usepackage{geometry}
\usepackage{ntheorem}
\usepackage{hyperref}
\usepackage{caption,subcaption}

\geometry{left=2cm,right=2cm,top=2cm,bottom=2cm}

\def\UrlBreaks{\do\A\do\B\do\C\do\D\do\E\do\F\do\G\do\H\do\I\do\J\do\K\do\L\do\M\do\N\do\O\do\P\do\Q\do\R\do\S\do\T\do\U\do\V\do\W\do\X\do\Y\do\Z\do\[\do\\\do\]\do\^\do\_\do\`\do\a\do\b\do\c\do\d\do\e\do\f\do\g\do\h\do\i\do\j\do\k\do\l\do\m\do\n\do\o\do\p\do\q\do\r\do\s\do\t\do\u\do\v\do\w\do\x\do\y\do\z\do\0\do\1\do\2\do\3\do\4\do\5\do\6\do\7\do\8\do\9\do\.\do\@\do\\\do\/\do\!\do\_\do\|\do\;\do\>\do\]\do\)\do\,\do\?\do\'\do+\do\=\do\#}

\newtheorem{theorem}{Theorem}
\newtheorem{lemma}{Lemma}
\newtheorem{proposition}{Proposition}
\newtheorem{corollary}{Corollary}
\newtheorem{claim}{Claim}
\newtheorem{conjecture}{conjecture}
\newtheorem{definition}{Definition}
\newtheorem{construction}{Construction}
\newtheorem*{exercise}{Exercise}
\newtheorem*{proof}{Proof}
\newtheorem*{answer}{Answer}
\newtheorem*{refute}{Refute}
\newtheorem*{example}{Example}
\newtheorem*{counterexample}{Counterexample}

\DeclareMathAccent{\widehat}{\mathord}{largesymbols}{"62}
\newcommand{\abs}[1]{\left| #1 \right|}
\newcommand{\pbra}[1]{\left( #1 \right)}
\newcommand{\cbra}[1]{\left\{ #1 \right\}}
\newcommand{\sbra}[1]{\left[ #1 \right]}
\newcommand{\bin}{\{0,1\}}

\title{Exercise Set --- Chapter $1$}
\date{}

\begin{document}

\maketitle

\begin{exercise}[1.3]
	In \textbf{Claim 1.6}, we want to prove that for every $k$-tape TM $M$,
    there always exist a single-tape TM $\widetilde{M}$ can simulate $M$.
    The way to map the $k$-tape to a single tape is as follows:
		$$
        \begin{matrix}		
            1\text{-tape}&\quad \text{location}\quad 1, k+1, 2k+1\\			
             2\text{-tape}&\quad \text{location}\quad 2, k+2, 2k+2\\			
			 \dots&\dots\\			
             k\text{-tape}&\text{location}\quad k, 2k, 3k\\			
		\end{matrix}
        $$
	Because $M$ can compute $f$ in $T(n)$ steps, the tape in $M$ must be shorter than $T(n)$.
    Hence the tape in $\widetilde{M}$ is no longer than $n+1+kT(n)$ (the first $n+1$ locations is the input).

	For every symbol $a$ in $M$'s alphabet, $\widetilde{M}$ contains both $a$ and $\widehat{a}$, 
    indicating the corresponding heads of $M$. 
	To simulate one step of $M$, procedure goes as follows:
    \begin{enumerate}
        \item Now the head locates at the leftmost.
            Then sweep to the rightmost and record the $k$ original heads, i.e., symbols with $\widehat{ }$. 
            This takes $2kT(n)$ steps.
        \item Use $M$'s transition function to determine the new state, symbol, and head.
        \item Sweep the tape back from right to left to update the tape. 
            This takes another $2kT(n)$ steps.
    \end{enumerate}
	During the three steps, there some minor steps maybe needed for updating head movement. 
	Therefore, $\widetilde{M}$ perform at most $5kT(n)^2$. 
\end{exercise}

\begin{exercise}[1.7]
    By \textbf{Claim 1.6}, it suffices to show $f$ can be computed in $O(T(n))$ with a three-tape TM $M$.
    Define bijection $\sigma:\mathbb Z^2\to\mathbb N$, where
    $$
    \begin{matrix}
        \sigma(0,0)=0 & \sigma(1,0)=1 & \sigma(0,1)=2 & \sigma(-1,0)=3 & \sigma(0,-1)=4\\
        \sigma(2,0)=5 & \sigma(1,1)=6 & \sigma(0,2)=7 & \sigma(-1,1)=8 & \sigma(-2,0)=9\\
        \sigma(-1,-1)=10 & \sigma(0,-2)=11 & \sigma(1,-1)=12 & \sigma(3,0)=13 & \dots\dots\\
    \end{matrix}
    $$
    Therefore $\sigma$ is time-constructible in $O(T(n))$.
    $M$ is constructed to maintain current coordinate $(x,y)$ in the second tape 
    and calculate $\sigma(x,y)$ in the third tape.
    Then it simulates the action of two-dimensional TM in the first tape.
\end{exercise}

\begin{exercise}[1.13]
	\begin{itemize}
		\item[(a)] According to Section 1.5.2, 
            \begin{gather*}
                \text{DIVIDES}(x,y)=\exists_k: y=x\times k\\
                \text{PRIME}(y)=\forall_x (x=1)\vee(x=y)\vee(\neg\text{DIVIDES}(x,y))\\
                \text{BIT}(n,i)=\exists_p: \text{DIVIDES}(n,p)\wedge ((C+i)^3<p<(C+i+1)^3) 
                \wedge (\forall_{p'}: p'\geq p\vee\neg\text{PRIME}(p'))
            \end{gather*}
		\item[(b)] 
            \begin{align*}
                \text{COMPARE}(n,m,i,j)=&(\text{BIT}(n,i)\wedge\text{BIT}(m,i)\vee\neg\text{BIT}(n,i)\wedge\neg\text{BIT}(n,i))\\
                &\wedge(\text{BIT}(n,j)\wedge\text{BIT}(m,j)\vee\neg\text{BIT}(n,j)\wedge\neg\text{BIT}(n,j))
            \end{align*}
	    \item[(c)] In general, we need to store the string in the tape, the state,
            the position of the read-write head. 
            Use $111$ to be the separator, $01$ for character $0$, and $10$ for character $1$. 
            For example, if the original alphabet is $\bin$, and at a specific time the string in the tape is $001$, 
            then we encode this as $010110$. The encoding for the left two elements are similar. 
            Notice that $3$ consecutive $1$ will only show as a separator, which is unambiguous.
        \item[(d)] 
	\end{itemize}
\end{exercise}

\begin{exercise}[1.14]
\begin{enumerate}		
	\item[\textbf{(a)}] 
		The language is in P.
		Choose any node and DFS the graph, then check whether the number of visited nodes is $|V|$. 
        If so, the graph is connected.
		\item[\textbf{(b)}] 
		The language is in P.
        Choose any edge $(u,v)\in G=(V,E)$. Then enumerate all $u$'s adjacent nodes $t$ and check whether $(t,v)\in E$. 
        Repeat until all the edges in $G$ have been visited.
		\item[\textbf{(c)}] 
		The language is P.
		DFS in a connected component of $G$ and do $2$-coloring simultaneously.
        Then check the validity of the coloring in this component. 
        Repeat the process for other connected components until all nodes are colored.
        If all are valid, $G$ is bipartite. 
		\item[\textbf{(d)}] 
		The language is P.
        Choose any node and start DFS in the graph, then
        check whether the number of visited nodes is $|V|$ and $|E|=|V|-1$.
\end{enumerate}
\end{exercise}

\end{document}
