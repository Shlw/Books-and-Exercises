% !TeX encoding = UTF-8
% !TeX program = XeLaTeX
% !TeX spellcheck = LaTeX

\documentclass[a4paper]{article}

\usepackage{amsmath,amsfonts,amssymb}
\usepackage{mathrsfs}
\usepackage{bm}
\usepackage{geometry}
\usepackage{ntheorem}
\usepackage{hyperref}
\usepackage[ruled]{algorithm2e}
\usepackage{caption,subcaption}

\geometry{left=2cm,right=2cm,top=2cm,bottom=2cm}

\def\UrlBreaks{\do\A\do\B\do\C\do\D\do\E\do\F\do\G\do\H\do\I\do\J\do\K\do\L\do\M\do\N\do\O\do\P\do\Q\do\R\do\S\do\T\do\U\do\V\do\W\do\X\do\Y\do\Z\do\[\do\\\do\]\do\^\do\_\do\`\do\a\do\b\do\c\do\d\do\e\do\f\do\g\do\h\do\i\do\j\do\k\do\l\do\m\do\n\do\o\do\p\do\q\do\r\do\s\do\t\do\u\do\v\do\w\do\x\do\y\do\z\do\0\do\1\do\2\do\3\do\4\do\5\do\6\do\7\do\8\do\9\do\.\do\@\do\\\do\/\do\!\do\_\do\|\do\;\do\>\do\]\do\)\do\,\do\?\do\'\do+\do\=\do\#}

\newtheorem{theorem}{Theorem}
\newtheorem{lemma}{Lemma}
\newtheorem{proposition}{Proposition}
\newtheorem{corollary}{Corollary}
\newtheorem{claim}{Claim}
\newtheorem{conjecture}{conjecture}
\newtheorem{definition}{Definition}
\newtheorem{construction}{Construction}
\newtheorem*{proof}{Proof}
\newtheorem*{answer}{Answer}
\newtheorem*{refute}{Refute}
\newtheorem*{example}{Example}
\newtheorem*{counterexample}{Counterexample}

\newenvironment{exercise}[1]{
	\par
	\noindent\textbf{Exercise #1.}\quad
}{
	\par
	\bigskip
}


\DeclareMathAccent{\widehat}{\mathord}{largesymbols}{"62}
\newcommand{\abs}[1]{\left| #1 \right|}
\newcommand{\pbra}[1]{\left( #1 \right)}
\newcommand{\cbra}[1]{\left\{ #1 \right\}}
\newcommand{\sbra}[1]{\left[ #1 \right]}
\newcommand{\bin}{\{0,1\}}

\title{Exercise Set --- Chapter $7$}
\date{}

\begin{document}

    \maketitle

    \begin{exercise}{7.1}
        Toss a coin for $T$ times where $T = \lceil\log_2(N) + \log_2(1/\delta)\rceil$. Naturally, the $T$ result forms a 0/1 sequence as the binary representation of $M$. Note that $2^T \leq N$ and $M < 2^T$. Let $k$ be the maximal integer s.t. $kN \leq 2^T$. If $M < kN$, output $\lfloor M/k\rfloor + 1$. Otherwise, output ``?''. Note that conditioned on $M < kN$, $M$ follows a uniform distribution, which implies the output is uniformly sampled from $[N]$. Whereas, $2^T - kN < N$ holds. Thus, the probability to output ``?'' is less that $(2^T - kN)/2^T < N/2^T \leq \delta$.
    \end{exercise}

    \begin{exercise}{7.6(a)}
		
		\paragraph{If.} Suppose $L$ is a language satisfying the settings mentioned. Construct a $\mathbf{ZPP}$ machine $M'$ as follows: $M'(x)$ keeps invokes $M(x)$ until $M(x)\neq ?$, and then output it. Apparently $M'$ will not err since $M$ does not. Suppose the running time of $M$ is bounded by a polynomial $T(n)$, then the expected running time of $M'$ is
		$$\mathbb{E}[T'(n)]\leq\frac{T(n)}{2^0}+\frac{2T(n)}{2^1}+\frac{3T(n)}{2^2}+...\leq 4T(n)$$
		Therefore $L\in\mathbf{ZPP}$.
		
		\paragraph{Only if.} Suppose $L\in\mathbf{ZPP}$. Then there exists $c>0$ satisfying, there is a machine $M'$ that runs in an expected-time $T'(n)=O(n^c)$ such that for every input $x$, if $M'(x)$ halts then $M'(x)=L(x)$. Construct a new machine $M$ as follows. $M(x)$ simulates $M'(x)$ on $T(n)=2T'(n)=O(n^c)$ steps. If in this duration $M'(x)$ outputs something, then $M(x)$ outputs it; otherwise, output $?$.
		
		By definition of $\mathbf{ZTIME}$, $M'(x)$ is always correct, and thus $M(x)\in\{L(x),?\}$. By Markov's inequality, 
		$$\Pr[\text{Runtime of }M'(x)\geq 2T'(n)]\leq\frac{T'(n)}{2T'(n)}=\frac{1}{2}$$
		And thus $\Pr[M(x)=?]\leq\frac{1}{2}$. Actually the constant of the bound can be changed to $\frac{1}{3}$ or less. 
    \end{exercise}

    \begin{exercise}{7.6(b)}
		\paragraph{$\subseteq$.} We are now going to prove $\mathbf{ZPP}\subseteq\mathbf{RP}$ and $\mathbf{ZPP}\subseteq\mathbf{coRP}$. The exercise 7.6(a) tells us there is a machine $M$ such that $M(x)\in\{L(x),?\}$ and $\Pr[M(x)=?]\leq\frac{1}{3}$. Construct $M''(x)$ as follows: if $M(x)=?$ then output $0$ otherwise output $M(x)$. Then
		$$L(x)=0\Rightarrow\Pr[M''(x)=0]=1$$
		and
		$$L(x)=1\Rightarrow\Pr[M''(x)=1]\geq\frac{2}{3}$$
		This gives $\mathbf{ZPP}\subseteq\mathbf{RP}$. Similar arguments can be applied to $\mathbf{ZPP}\subseteq\mathbf{coRP}$. 
		
		\paragraph{$\supseteq$.} We are now going to prove that, if $L\in\mathbf{RP}$ and $L\in\mathbf{coRP}$ then $L\in\mathbf{ZPP}$. Since $L\in\mathbf{RP}$, there is a polynomial PTM $M'$ such that $M'(x)=1\Rightarrow L(x)=1$. Similarly, since $L\in\mathbf{coRP}$, there is a polynomial PTM $M''$ such that $M''(x)=0\Rightarrow L(x)=0$. Then construct a PTM $M$ as follows:
		\begin{enumerate}
			\item $M(x)$ invokes $M'(x)$. If $M'(x)=1$ then output $1$.
			\item $M(x)$ invokes $M''(x)$. If $M''(x)=0$ then output $0$. 
			\item Goto the first step.
		\end{enumerate}
	
		The loop continues only if $M'(x)=0$ and $M''(x)=1$ happens simutaneously. By definition, this happens with probability no more than $\frac{1}{9}$, and the expected running time of $M$ is
		$$\mathbb{E}[T(n)]\leq\frac{T'(n)+T''(n)}{9^0}+\frac{2(T'(n)+T''(n))}{9^1}+...< 2(T'(n)+T''(n))$$
		which is a polynomial. This yields $L\in\mathbf{ZPP}$. 
		
    \end{exercise}

\begin{exercise}{7.7}
Assuming $L\in \textbf{BP}\cdot\textbf{NP}$, by encoding the randomness and certification of $L$ to $y$, let the circuit work as a verifier, then $L\in\textbf{NP}/poly$.
\end{exercise}

\begin{exercise}{7.8}
To prove $PH$ collapses to $\Sigma_3^p$, it suffices to show that $\Pi_3 SAT\subset \Sigma_3^p$.

$$
\Pi_3 SAT=\{\varphi(\cdot,\cdot,\cdot)| \forall x_1, \exists x_2, \forall x_3, \varphi(x_1,x_2,x_3)=0\}.
$$

$\varphi\in \Pi_3SAT$ iff $\forall x_1, \exists x_2$ such that $\varphi(x_1,x_2,\cdot)\in \overline{3SAT}$.

Since $\overline{3SAT}\leq_r 3SAT$, there exists deterministic polynomial time Turing Machine $M$,  $n=\max(|x_1|,|x_2|,|\varphi|)$, integer $m$, such that $\forall \varphi(x_1,x_2,\cdot)$
$$
P_{r\in\{0,1\}^m}(  \varphi(x_1,x_2,\cdot)\in \overline{3SAT} \text{ iff } M(r,\varphi(x_1,x_2,\cdot))\in 3SAT)\geq 1- 2^{-3n-1}.
$$ 
Say that a random bit $r$ is bad if the event in the $P(\cdot)$ does not happen. Add over all  $\varphi(x_1,x_2,\cdot)$, the bad random bit is at most $2^{-3n-1}*2^{m}*2^{3n}<2^{m}$. Thus there exists a $r_0$ which is good for all inputs, that is

$$
 \varphi(x_1,x_2,\cdot)\in \overline{3SAT} \text{ iff } M(r_0,\varphi(x_1,x_2,\cdot))\in 3SAT
$$

$$
\Rightarrow 
\varphi\in \Pi_3SAT \text{ iff } \exists r_0, \forall x_1, \exists x_2, \exists u,   \langle M(r_0,\varphi(x_1,x_2,\cdot)), u\rangle =1
$$.

%\Leftarrow \varphi\not\in \Pi_3SAT, then \forall r, there exists
 


Thus we conclude if $\overline{3SAT}\leq_r 3SAT$, then $\Pi_3 SAT\subset \Sigma_3^p$.
\end{exercise}

\begin{exercise}{7.9}
	
	From the definition , $BPL\subseteq NL$, and because $NL\subseteq P$,so easily get that $BPL\subseteq P$. 
	
\end{exercise}

\end{document}
