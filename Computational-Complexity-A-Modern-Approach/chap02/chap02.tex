\documentclass[a4paper]{article}

\usepackage[UTF8]{ctex}
\usepackage{amsmath,amsfonts,amssymb}
\usepackage{mathrsfs}
\usepackage{bm}
\usepackage{geometry}
\usepackage{ntheorem}
\usepackage{hyperref}
\usepackage{caption,subcaption}

\geometry{left=2cm,right=2cm,top=2cm,bottom=2cm}

\def\UrlBreaks{\do\A\do\B\do\C\do\D\do\E\do\F\do\G\do\H\do\I\do\J\do\K\do\L\do\M\do\N\do\O\do\P\do\Q\do\R\do\S\do\T\do\U\do\V\do\W\do\X\do\Y\do\Z\do\[\do\\\do\]\do\^\do\_\do\`\do\a\do\b\do\c\do\d\do\e\do\f\do\g\do\h\do\i\do\j\do\k\do\l\do\m\do\n\do\o\do\p\do\q\do\r\do\s\do\t\do\u\do\v\do\w\do\x\do\y\do\z\do\0\do\1\do\2\do\3\do\4\do\5\do\6\do\7\do\8\do\9\do\.\do\@\do\\\do\/\do\!\do\_\do\|\do\;\do\>\do\]\do\)\do\,\do\?\do\'\do+\do\=\do\#}

\newtheorem{theorem}{Theorem}
\newtheorem{lemma}{Lemma}
\newtheorem{proposition}{Proposition}
\newtheorem{corollary}{Corollary}
\newtheorem{claim}{Claim}
\newtheorem{conjecture}{conjecture}
\newtheorem{definition}{Definition}
\newtheorem{construction}{Construction}
\newtheorem*{proof}{Proof}
\newtheorem*{answer}{Answer}
\newtheorem*{refute}{Refute}
\newtheorem*{example}{Example}
\newtheorem*{counterexample}{Counterexample}

\newenvironment{exercise}[1]{
	\par
	\noindent\textbf{Exercise #1.}\quad
}{
	\par
	\bigskip
}


\DeclareMathAccent{\widehat}{\mathord}{largesymbols}{"62}
\newcommand{\abs}[1]{\left| #1 \right|}
\newcommand{\pbra}[1]{\left( #1 \right)}
\newcommand{\cbra}[1]{\left\{ #1 \right\}}
\newcommand{\sbra}[1]{\left[ #1 \right]}
\newcommand{\bin}{\{0,1\}}

\title{Exercise Set --- Chapter $2$}
\date{}

\begin{document}

\maketitle

	\begin{exercise}{2.1}
	    hhh
	\end{exercise}

	\begin{exercise}{2.2}
		\textbf{Solution.}
		\begin{itemize}
			\item 2COL is in NP.

			Given a colouring scheme $C$ for graph $G(V,E)$, where $|V| = n, |E| = m$, we can verify whether $C$ in $O(m)$ time by checking each of edges $e = (u,v)$ whether their color are the same, if there exists $u,v$ whose color is the same, then the colouring scheme $C$ is illegal, otherwise accept $C$.

			2COL is the same as bipartite problem. To find whether a graph can be 2COL. That is, we need only prove there are not exists odd cycle in graph G.
			\item 3COL is in NP. The verify method is the same as 2COL problem.
			\item Connectivity is in NP. 
			Given a graph G, we can decide whether which is connected by a nondeterministic Turing machine in $O(n)$ time.
			\begin{itemize}
				\item Random select a vertex $u\in G(V,E)$, if degree$(u) = 0$, reject. Otherwise:
				\item Let $S = \{u\}$, do:
				%\item \quad Add $u$ to vertex set $S$.
				\item \quad Nondeterministic goto all of neighbour of $u$, and add all of its neighbour to $S$, for any neighbour of $u$, recursively and nondeterministic execute step 3 until for two adjacent step $|S|$ is fixed. 
				\item If $|S| = n$, accept, otherwise reject. 
			\end{itemize}
			The process can be executed at most with depth $n$, and thus the time if $O(n)$ to check whether $G$ is connected by a nondeterministic Turing machine.
		\end{itemize}

		\begin{itemize}
			\item 2COL is in P. %A graph can be 2COL iff it has no odd cycle. 
			\begin{itemize}
				\item For every connected sub-graph of $G$, random select a vertex $v\in V$, color $v$ to red, let step$(v) = 0$, push $v$ into queue $Q$\;
				\item \quad Pop $v$ from $Q$, for any $u \in N(v)$, if $v$ has been coloured, and step$(u)=$ step$(v)$, then reject. If $v$ has not been coloured, do:
				\item \quad \quad Push $u$ to $Q$, step$(u)=step(v) + 1$ (mod $2$);
				\item \quad \quad If step$(u)=0$, color $u$ to red, otherwise color $u$ to blue.
				\item \quad go to step (2) until $Q$ is empty.
			\end{itemize}
			Time complexity: $O(m)$.
			\item Connectivity is in P.
			\begin{itemize}
				\item Random select a vertex $v\in P$, if degree($v$) = 0, reject. Otherwise add $S = \{v\}$.
				\item Using depth first method adding all of neighbours of $u$ into $S$.
				\item Execute step 2 until $|S| = n$ or $|S|$ fixed.
				\item If $|S|< n$ reject, otherwise accept.
			\end{itemize}
			Time complexity: $O(m)$.
		\end{itemize}
	\end{exercise}

	\begin{exercise}{2.3}
	    LINEQ can be solved with Gaussian elimination method, which is in P.
	\end{exercise}

	\begin{exercise}{2.4}
		The problem linear programming is corresponding to the following language:
		\[
			\mathsf{LP} := \{\langle A,b \rangle \mid A \in \mathbb{R}^{m\times n}, b \in \mathbb{R}^m, \exists v \in \mathbb{R}^n: Av \leq b\}.
		\]
		Consider the following polynomial-time algorithm $M$: Given a pair $\langle A,b \rangle$ and a string $u \in \{0,1\}^*$, output $1$ if and only if $u$ encodes a non-empty subset $S \subseteq [m]$ such that for all $v$ in the affine subspace
		\[
			V = \{v \in \mathbb{R}^n \mid \forall i \in S: A_i v = b_i\},
		\]
		$Av \leq b$ holds. We claim it can be verify in polynomial time. By Gauss algorithm, the basis $\{v_1,\ldots,v_k\}$ and bias $b'$ of $V$ can be computed. It just need to be checked that for all $i\in [k]$, $A v_i = 0$ and $Ab' \leq b$. Clearly, $\langle A,b \rangle$ is in $\mathsf{LP}$ if and only if there exists a string $u$ such that $M(\langle A,b \rangle, u) = 1$. Note that a subset of $[m]$, as the certificate, can be encoded with $m$ bits. Thus, $\mathsf{LP}$ is in $\mathbf{NP}$
	\end{exercise}

	\begin{exercise}{2.9}
		To show that the relation $\leq_p$ is not symmetric, we let $L=\{0,1\}^*$ and $L'=SAT$. It is easy to show that $L\leq_p L'$ by transform every $x\in L$ to a tautology such as $x\vee \neq x$. However, $L'\not\leq_p L$ iff we can solve $SAT$ in polynomial time.
	\end{exercise}
	
	\begin{exercise}{2.5}
		证明所有素数二进制编码构成的语言属于 NP。​

		根据 NP 的定义:$x\in L \iff\exists u\in (0,1)^{p|x|},M(u,x)=1$。对于任意一个质数 $n$,我们可以找到一个验证,对于任意的小于 $n$ 的素数 $q$ 都能找到一个 $a$ 满足 $a^{n-1}=1 \pmod n$ 且 $a^{\frac{n-1}{q}}\ne 1 \pmod n$。那么 $u=q_0a_0\dots q_ka_k$ 且 $u=O(\phi (n)\cdot\log n)=poly (n)$,因此所有素数的二进制编码构成的语言属于 NP。
	\end{exercise}

    \begin{exercise}{2.13}
	\begin{itemize}
\item [(a)]
Parsimonious (Page 50, 2.3.6): 
%The size of certificate of $x$ and certificate of $f(x)$ is the same. $u$ is a certificate for $x$ if $x\in L$ and $u\in \{0,1\}^{p(x)}$ satisfy $M(x,u) = 1$, for some polynomial $p$ and polynomial Turing machine $M$.\\
%where $v = g(u)$ satisfy the above two relationships.
To make the number of solutions for a instance in $L$ is the same as its corresponding SAT instance, we just need to modify the Turing machine $M$ so that it clears up its work tape before outputting a 1 and moves both heads to one end of the tape. Then the final snapshot and head locations are unique. 
\item[(b)]Show a parsimonious reduction from SAT to 3SAT.\\
For a SAT instance $\phi$, which has $n$ variables $x_{1}, \cdots, x_{n}$, $m$ clauses $c_{1},\cdots, c_{m}$, and the maximal size of clause is $k$. We construct a 3SAT instance $\phi'$, which has at most $n + k^{2}$ variables $x_{1},\cdots, x_{n}$ and $y_{1},\cdots , y_{s}$, where $s\leq k^{2}$.

Firstly, for a clause $c_{j} = v_{1} \vee v_{2} \vee \cdots \vee v_{k}$, which equals to 
\begin{equation}(v_{1} \vee v_{2} \vee y_{1} )\wedge (\bar{y}_{1} \vee v_{3} \vee y_{2})
\wedge \cdots \wedge(\bar{y}_{k-1}\vee v_{k - 1} \vee v_{k})
\label{EqSAT}\end{equation}

Execute this partition process for every clauses, this gives $\phi'$. There will be at most  $n + k(k - 1)$ variables and at most $m + k(k - 1)$ clauses. Obviously, the process of the reduction is polynomial time.

To make the reduction parsimonious, we revise the reduction a little.
Change Eq. \ref{EqSAT} to
\begin{equation}
(F \vee v_{1} \vee y_{1} )\wedge (\bar{y}_{1} \vee v_{2} \vee y_{2})\vee \cdots \vee
\wedge (\bar{y}_{k}\vee v_{k} \vee F) \wedge (\bar{v}_{1} \vee \bar{y}_{1} \vee F) \wedge ( y_{1} \vee \bar{y}_{2}\vee F) \cdots \wedge (\bar{v}_{k-1} \vee \bar{y}_{k} \vee F)\wedge 
(  \bar{v}_{k}\vee \bar{y}_{k}\vee F)
\label{EqSAT1}\end{equation}
Thus $\phi'$ has at most $m(3k - 2)$ clauses and $n + k^{2}$ variables.

Now we prove the number of solutions for $\phi$ is the same as solutions for $\phi'$.

For a solution $v_{1}, v_{2}, \cdots, v_{n}$ for $\phi$. Let the corresponding $x_{1} = v_{1},\cdots, x_{n} = v_{n}$. Since the clauses in $\phi'$ is consisted of $m$ clause string like Eq. \eqref{EqSAT1}. Then for any clause string $j$ (for which $1
\leq j \leq m$), we assigned $y$ from left to right. For a clause $(\bar{y}_{i}\vee v_{i} \vee y_{i+1})$, if the first two variables are false, then assign $y_{i+1}$ to be true, otherwise assign $y_{i}$ to be false, since clause $(y_{i}\vee \bar{v}_{i} \vee F)$ and $({y}_{i}\vee \bar{y}_{i+1} \vee F)$ need to be satisfied. Obviously, in this assignments, the assignment of $y_{j}$ is fixed. Then we will get a satisfying assignment $x_{1},\cdots, x_{n}, y_{1},\cdots, y_{s}$ for $\phi'$.

For a solution $x_{1} = v_{1},\cdots, x_{n}= v_{n}, y_{1}=u_{1},\cdots, y_{s}= u_{s}$ for $\phi'$, we just need to let $x_{1} = v_{1},\cdots, x_{n}= v_{n}$ for $\phi$.

Since the mapping is monomorphism. Thus the number of solutions for $\phi$ and $\phi'$ is the same.
\end{itemize} 
\end{exercise}

    \begin{exercise}{2.14}
        If $M_1$ shows a polynomial-time Cook reduction from $L$ to $L'$ and $M_2$ shows a Cook reduction from $L'$ to $L''$,
        then consider $M_3$ which simulates $M_2$ and calls $M_1$ every time $M_2$ uses its oracle. Thus $M_3$ shows 
        $L$ is polynomial-time Cook reducible to $L''$, which indicates the transitivity.

        Construct a polynomial-time TM $M$. Assume $\phi$ is an instance of 3SAT. 
        $M$ runs TAUTOLOGY oracle on $\neg\phi$; if the output of the oracle is $b\in\bin$, $M$ outputs $b\oplus 1$.
        Thus 3SAT is Cook reducible to TAUTOLOGY.
    \end{exercise}
	
	
	\begin{exercise}{2.15}
		\begin{itemize}
			\item Note that the \textsf{CLIQUE} problem on the input $\langle G,k\rangle$ is equivalent to the \textsf{INDSET} problem on the input $\langle \bar G, k\rangle$. The complement graph of a given graph can be constructed in polynomial-times. I.e., \textsf{CLIQUE} and \textsf{INDSET} are polynomial-time reducible to one anther, which implies \textsf{CLIQUE} is \textbf{NP}-complete.

			\item Given a 3-CNF formula $\phi$ with $m$ clauses $C_1, \ldots, C_m$ on variable set $\{x_1,\ldots,x_n\}$, construc a graph $G = (V,E)$ where $V = \{v_{x_i}, v_{\overline{x_i}} \mid i \in [n]\} \cup \{v_{C_{i,1}}, v_{C_{i,2}}, v_{C_{i,3}} \mid i\in [m]\}$. Connect $v_{x_i}$ and $v_{\overline{x_i}}$ for all $i\in [n]$. Connect $v_{C_{i,1}}$, $v_{C_{i,2}}$ and $v_{C_{i,3}}$ pairwise for all $i \in [m]$. For $i \in [m]$ and $j \in [3]$, connect $v_{C_{i,j}}$ and $v_{x_k}$ if $j$-th literal in $C_i$ is $x_k$ and connect $v_{C_{i,j}}$ and $v_{\overline{x_k}}$ if it is $\overline{x_k}$. If $\phi$ is satisfiable with an assignment $y$, a $2m+n$-size vertex cover can be constructed. For all $i\in [n]$, $S$ gathers $v_{x_i}$ if $y_i$ is true, or gathers $v_{\overline{x_i}}$ otherwise. For each $i \in [m]$, $S$ gathers two of $v_{C_{i,1}}$, $v_{C_{i,2}}$ and $v_{C_{i,3}}$ with a vertex, which represents the satisfied literal in the assignment $y$ not gathered (such vertex must exist since each clause is satisfied). Clearly, $S$ covers all the edge in $G$. If there exists a vertex cover $S$ with size of $n+2m$, exact one vertex in $\{v_{x_i}, v_{\overline{x_i}}\}$ and exact two vertices in $\{v_{C_{j,1}}, v_{C_{j,2}}, v_{C_{j,3}}\}$ are gathered for all $i \in [n]$ and $j\in [m]$ due to the structure of $G$. Construct an assignment $y$ satisfying $\phi$. For all $i\in [n]$, if $v_{x_i}$ is gathered by $S$, set $y_i$ as true, or as false otherwise. Consider each $i\in [m]$, the not gathered vertex in $\{v_{C_{i,j}}\mid j\in [3]\}$ is incident to a edge which must be covered by anther vertex $v_{x_k}$ (or $v_{\overline{x_k}}$). $x_k$ (or $\overline{x_k}$) can satisfy $C_i$ and its value is true in $y$. Thus, $y$ satisfies $\phi$. Combining the fact that the construction of $G$ from $\phi$ can be done efficiently, \textsf{3SAT} can be reduced to \textsf{VERTEX COVER} polynomially, which means \textsf{VERTEX COVER} is \textbf{NP}-hard. Furthermore, it is easy to see the vertex covering can be checked within polynomial times. Thus, \textsf{VERTEX COVER} is \textbf{NP}-complete.		
		\end{itemize}
	\end{exercise}

	\begin{exercise}{2.16}
		证明最大割是 NP 完全问题。最大割是 NP 问题的证明比较显然,我们可以非确定性的枚举所有点集划分的可能性,然后去判断是否割能达到 $k$。

		证明最大割是 NP 难问题,将 partition 规约到 MAX-CUT 问题是一个比较简单的方法,按照课本中提到的用顶点覆盖去规约的方法如下:因为极小顶点覆盖的补集一定是极大顶点独立问题,因此我们可以将IS规约到MAX CUT。图构造过程:$V^{'}=V\cup\{x\}\cup V^{''},V^{''}=\{[e,u],[e,v]|e=uv\in E\},E^{'}=\{xv|v\in V\}\cup E^{''}$ $E^{''}=\{x[e,u],x[e,v],[e,u][e,v],[e,u]u,[e,v]v|e=uv\},k^{'}=k+4|E|$。这个结构的好处是,求完最大割后,如果$x$与$u,v$中任何一点在一边的话,在$x,u,v$这里的割一定是4,否则为3先证明,如果原问题中存在至少为k的独立集,那么一定有至少为$k+4|E|$的割。因为$S$是一个独立集,那么当$S$与$x$都不在同一侧的时候,这里有$k$条边,又因为$S$为独立集,所以不存在边,因此每一个边$uv$可以保证有一个和$x$同边,因此这里最多产生$4|E|$条边。

		再证明,如果有一个大于$k+4|E|$的割,那么一定有一个至少为$k$的独立集假设现在的割为$S^{'}$与$V^{'}-S^{'} $  假设$E_c$为所有端点在两段的集合,$|E_c| >=k+4|E|$因为新加进的$[e,u]$这些边一组最多贡献4个割边,所以至少有$k$个边是不含$[e,u]$这个端点的,不含新加端点的边一定是$xv$,我们把和$x$在异端的点集合设为$S_c$ 因为$Sc$大于等于$k$,不妨令$|S_c|=k+l$

		因为$|E_c|>= k+ 4|E| $联立可以有至多$l$组新加点造成的割为3,这些割不足3的代表边的两段都与$x$异端,因此我们只需要对着$l$组,任选一个点丢出$S_c$,那么剩下的点,一定没有公共边(对于原来可能存在边的都丢了一个)因此剩下的一定是一个独立集,并且点的个数为$k+l $减去小于$l$的点,因此大于等于$k$
	\end{exercise}

	\begin{exercise}{2.20}
		To show that $REALQUADEQ$ is \textbf{NP}-complete, we make reduction from $3SAT$ to $REALQUADEQ$.
		Given a SAT instance, which has $n$ variables and $m$ clauses, we construct $REALQUADEQ$ which has $n+m$ variables and $n+m$ quadratic equations. For example, given
		\begin{align*}
			& x_1\vee x_2\vee x_3\\
			&\neg x_2\vee x_4\vee x_3\\
		\end{align*}
		for $m$ clause, we add $m$ new variables as $y_i$ and construct $m$ equation as 
		\begin{align*}
			&x_1^2+x_2^2+x^3-y_1^2=1\\
			&(1-x_2)^2+x_4^2+x_3^2-y_2^2=1\\
		\end{align*}
		To constrain orignal $x_i$ to have value in $\{0,1\}$, we construct extra $n$ equations as 
		$$x_i(1-x_i)=0$$.
	\end{exercise}
        
	\begin{exercise}{2.24}
	Since $\bar{L} \in NP$, then by the definition of NP, exists $p: \mathbb{N}\rightarrow \mathbb{N}$ and a polynomial TM $M$ s.t. for every $x \in \{0,1\}^{*}$,
$$x \in \bar{L} \Leftrightarrow \exists u \in \{0,1\}^{p(|x|)} \text{ s.t. } M(x,u) = 1$$
which means
$$x \in L \Leftrightarrow \forall u \in \{0,1\}^{p(|x|)} \text{ s.t. } M(x,u) = 0$$
Let $M'$ output $1$ if $M$ output $0$, and output $0$ if $M$ output $1$.
Then $M'$ satisfies
$$x \in L \Leftrightarrow \forall u \in \{0,1\}^{p(|x|)} \text{ s.t. } M'(x,u) = 1$$
That's implies definition 2.20.
\end{exercise}

    \begin{exercise}{2.25}
        For any $L\in NP=P$, $\overline{L}\in P=NP$, which means $L\in coNP$.
        For any $L\in coNP$, $\overline{L}\in NP=P$ and $L=\overline{\overline{L}}\in P=NP$, which means $L\in NP$.
        Therefore $NP=coNP$.
    \end{exercise}
	
	\begin{exercise}{2.26}
		Suppose $\textsf{3SAT} \leq_p \textsf{TAUTOLOGY}$ and $ \leq_p \textsf{TAUTOLOGY} \leq_p \textsf{3SAT}$. For any $L \in \textbf{NP}$, that $L \leq_q \in \textsf{3SAT}$ implies $\bar L \leq_q \in \overline{\textsf{3SAT}}$. Thus, we have $\bar L \leq_q \in \overline{\textsf{3SAT}} \leq_p \textsf{TAUTOLOGY} \leq_p \textsf{3SAT}$, which means $L \in \textbf{coNP}$. Besides, for any $L \in \textbf{coNP}$, $L \leq_p \textsf{TAUTOLOGY} \leq_p \textsf{3SAT}$. Combining them, $\textbf{NP} = \textbf{coNP}$ can be shown.

		Note that $\textsf{3SAT}$ is \textbf{NP}-complete and $\textsf{TAUTOLOGY}$ i s \textbf{coNP}-complete. If $\textbf{NP} = \textbf{coNP}$, they will be polynomial-time reducible to one another.
	\end{exercise}

	\begin{exercise}{2.27}
		仿照NP的2.1定义形式给NEXP定义,并证明两个定义等价
		\begin{itemize}
			\item [1.] NEXP定义$x\in L \iff\exists u\in (0,1)^{\exp|x|},M(u,x)=1$其中M是一台指数时间的确定性图灵机。
			\item [2.] 两个定义等价,NDTIME定义$\to$本题定义,$L\in $NEXP$\iff L$可以由非确定性图灵机在指数时间内求解,那么NTM每一步选择的$\delta$函数序列为$u$,则存在确定性图灵机可以利用这个序列$u$和原输入$x$在指数时间内可以验证,且$u$是指数范围内的。

				本题定义$\to$NDTIME定义,非确定性产生验证$u$,新的非确定性图灵机停机当且仅当有一个验证$u$,验证$x$。又$\exp(n) =O(2^{n^c})\to L\in NTIME(2^{n^c})$。综上,两个定义等价。
		\end{itemize}
	\end{exercise}

\begin{exercise}{2.31}
The difference between SUBSET SUM and UNARY SUBSET SUM is the input size. In UNARY SUBSET SUM problem, given numbers $n_1,n_2,...,n_k$, the input size is $\sum_i n_i$. To show that UNARY is in P, we give an algorithm whose running time is $k*(\sum_i n_i)<(\sum_i n_i)^2$. The algorihtm uses dynamic programming and is
similar to those for knapsack problem.

Specifically, the algorithm takes a list $n_1,...,n_k$ as input, and output $0$ or $1$ depends on whether this exists a subset $\{n_{i_{m}}\}$ such that $\sum_m n_{i_m}=(\sum_i n_i)/2$. For convinience, we write $M$ for $(\sum_i n_i)/2$. We use a table $T(s,i)$, $1\leq s\leq M, 1\leq i\leq k$, to record whether there exists a subset of the first $i$ integers whose total sum is $s$. Then
$$
T(s,i+1)=max(T(s-n_i,i), T(s,i))
$$
\end{exercise}
\end{document}
