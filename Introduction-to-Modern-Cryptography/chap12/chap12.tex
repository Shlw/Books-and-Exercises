% !TeX encoding = UTF-8
% !TeX program = XeLaTeX
% !TeX spellcheck = LaTeX

% Author : Shlw

\documentclass[a4paper]{article}

\usepackage{amsmath,amsfonts,amssymb}
\usepackage{mathrsfs}
\usepackage{bm}
\usepackage{geometry}
\usepackage{ntheorem}
\usepackage{hyperref}
\usepackage[ruled]{algorithm2e}
\usepackage{caption,subcaption}

\geometry{left=2cm,right=2cm,top=2cm,bottom=2cm}

\def\UrlBreaks{\do\A\do\B\do\C\do\D\do\E\do\F\do\G\do\H\do\I\do\J\do\K\do\L\do\M\do\N\do\O\do\P\do\Q\do\R\do\S\do\T\do\U\do\V\do\W\do\X\do\Y\do\Z\do\[\do\\\do\]\do\^\do\_\do\`\do\a\do\b\do\c\do\d\do\e\do\f\do\g\do\h\do\i\do\j\do\k\do\l\do\m\do\n\do\o\do\p\do\q\do\r\do\s\do\t\do\u\do\v\do\w\do\x\do\y\do\z\do\0\do\1\do\2\do\3\do\4\do\5\do\6\do\7\do\8\do\9\do\.\do\@\do\\\do\/\do\!\do\_\do\|\do\;\do\>\do\]\do\)\do\,\do\?\do\'\do+\do\=\do\#}

\newtheorem{theorem}{Theorem}
\newtheorem{lemma}{Lemma}
\newtheorem{proposition}{Proposition}
\newtheorem{corollary}{Corollary}
\newtheorem{claim}{Claim}
\newtheorem{conjecture}{conjecture}
\newtheorem{definition}{Definition}
\newtheorem{construction}{Construction}
\newtheorem*{proof}{Proof}
\newtheorem*{answer}{Answer}
\newtheorem*{refute}{Refute}
\newtheorem*{example}{Example}
\newtheorem*{counterexample}{Counterexample}
\newenvironment{exercise}[1]{
	\par
	\noindent\textbf{Exercise #1.}\quad
}{
	\par
	\bigskip
}

\DeclareMathAccent{\widehat}{\mathord}{largesymbols}{"62}
\DeclareMathOperator{\lequiv}{\ \Leftrightarrow\ }
\DeclareMathOperator{\Image}{\mathop{Im}}
\newcommand{\rawE}{\mathop{\mathbb E}}
\newcommand{\E}[1]{\mathop{\mathbb E}_{#1}}
\newcommand{\abs}[1]{\left| #1 \right|}
\newcommand{\pbra}[1]{\left( #1 \right)}
\newcommand{\cbra}[1]{\left\{ #1 \right\}}
\newcommand{\sbra}[1]{\left[ #1 \right]}
\newcommand{\bin}{\{0,1\}}
\newcommand{\Enc}{\mathrm{Enc}}
\newcommand{\hc}{\mathrm{hc}}
\newcommand{\half}{\mathrm{half}}
\newcommand{\lsb}{\mathrm{lsb}}
\newcommand{\Gen}{\mathrm{Gen}}
\newcommand{\Ext}{\mathrm{Ext}}
\newcommand{\Dec}{\mathrm{Dec}}
\newcommand{\Mac}{\mathrm{Mac}}
\newcommand{\Sign}{\mathrm{Sign}}
\newcommand{\Vrfy}{\mathrm{Vrfy}}
\newcommand{\PrivK}{\mathrm{PrivK}}
\newcommand{\PubK}{\mathrm{PubK}}
\newcommand{\KEM}{\mathrm{KEM}}
\newcommand{\Encaps}{\mathrm{Encaps}}
\newcommand{\Trans}{\mathrm{Trans}}
\newcommand{\GenRSA}{\mathrm{GenRSA}}
\newcommand{\Decaps}{\mathrm{Decaps}}
\newcommand{\Macforge}{{\mathrm{Mac}\text{-}\mathrm{forge}}}
\newcommand{\Macsforge}{{\mathrm{Mac}\text{-}\mathrm{sforge}}}
\newcommand{\Encforge}{{\mathrm{Enc}\text{-}\mathrm{Forge}}}
\newcommand{\Sigforge}{{\mathrm{Sig}\text{-}\mathrm{forge}}}
\newcommand{\Invert}{\mathrm{Invert}}
\newcommand{\Ident}{\mathrm{Ident}}
\newcommand{\RSAinv}{{\mathrm{RSA}\text{-}\mathrm{inv}}}
\newcommand{\Feistel}{\mathrm{Feistel}}
\newcommand{\Hashcoll}{{\mathrm{Hash}\text{-}\mathrm{coll}}}
\newcommand{\negl}{\mathrm{negl}}
\newcommand{\ppt}{{\sc ppt}~}
\newcommand{\eav}{\mathrm{eav}}
\newcommand{\out}{\mathrm{out}}
\newcommand{\mult}{\mathrm{mult}}
\newcommand{\cpa}{\mathrm{cpa}}
\newcommand{\cca}{\mathrm{cca}}
\newcommand{\onetime}{{\mathrm{1}\text{-}\mathrm{time}}}
\newcommand{\Used}{\mathrm{Used}}
\newcommand{\Asked}{\mathrm{Asked}}
\newcommand{\Acal}{\mathcal{A}}
\newcommand{\Bcal}{\mathcal{B}}
\newcommand{\Dcal}{\mathcal{D}}
\newcommand{\Gcal}{\mathcal{G}}
\newcommand{\Ocal}{\mathcal{O}}
\newcommand{\Pcal}{\mathcal{P}}
\newcommand{\Vcal}{\mathcal{V}}
\newcommand{\Xcal}{\mathcal{X}}
\newcommand{\Ycal}{\mathcal{Y}}
\newcommand{\Zcal}{\mathcal{Z}}
\newcommand{\Gset}{\mathbb{G}}
\newcommand{\Nset}{\mathbb{N}}
\newcommand{\Zset}{\mathbb{Z}}
\newcommand{\Hset}{\mathbb{H}}

\title{Exercise Set --- Chapter $12$}
\date{}

\begin{document}

\maketitle

\begin{exercise}{12.2}
    Assume $\Pi=(\Gen,\Sign,\Vrfy)$ is a one-time-secure signature scheme and $\Gen(1^n)$ uses at most $\ell(n)$ random bits.
    Define $f(r)=pk_r,r\in\bin^{\ell(n)}$, where $pk_r$ is the public key generated by $\Gen(1^n;r)$. 
    For any \ppt adversary $\Acal$ of $f$, construct an adversary $\Bcal$ of $\Pi$:
    \begin{enumerate}
        \item $\Bcal$ is given $pk$.
        \item Run $\Acal(pk)$ and get $r'$.
        \item Run $\Gen(1^n;r')$ and get secret key $(pk',sk')$.
        \item If $pk'=pk$ output $(0,\Sign_{sk'}(0))$; otherwise abort.
    \end{enumerate}
    Therefore, there exists a negligible function $\negl$ such that
    \begin{align*}
        \negl(n)&\leq\Pr\sbra{\Sigforge_{\Bcal,\Pi}(n)=1}\\
        &=\Pr\sbra{r'\gets\Acal(pk),(pk',sk')\gets\Gen(1^n;r'),pk'=pk,\Vrfy_{pk}(0,\Sign_{sk'}(0))=1}\\
        &=\Pr\sbra{r'\gets\Acal(pk),pk'=pk,(pk',sk')\in\Gen(1^n)}\\
        &=\Pr\sbra{\Acal(pk)\in\cbra{r\in\bin^{\ell(n)}\middle|\Gen(1^n;r)=(pk,\cdot)}}\\
        &=\Pr\sbra{f(\Acal(f(r)))=f(r)},
    \end{align*}
    which gives the one-wayness of $f$.
\end{exercise}

\begin{exercise}{12.3}
    If the adversary wants to compute $m^d$, it can simply ask for $(-m)^d$ then multiply it by $-1$.
    Since $N=pq$, $d$ is odd and $m\not\equiv -m\mod N$ if $m\neq 0$.
\end{exercise}
    
\begin{exercise}{12.4}
    For any \ppt adversary $\Acal$ of plain RSA signature scheme under the experiment of the weak security,
    construct an adversary $\Bcal$ of RSA experiment, which simply outputs $\Acal(N,e,y)$ given input $(N,e,y)$.

    Let $y=m^e,m\in\Zset_N^*$. Since in RSA experiment $m$ is chosen uniformly in $\Zset_N^*$, from the perspective of $\Acal$ 
    message $y$ is also uniform in $\Zset_N^*$. Therefore,
    $$
        \Pr\sbra{\RSAinv_{\Bcal,\GenRSA}(n)=1}
        =\Pr\sbra{\Acal(N,e,m^e)=m}
        =\Pr\sbra{\Acal(N,e,m^e)=(m^e)^d}
        =\Pr\sbra{\text{valid signature}}.
    $$
    Then the hardness of RSA problem gives the weak security of plain RSA signature scheme immediately.
\end{exercise}

\begin{exercise}{12.5}
    \begin{itemize}
        \item[(a)] Check if $\Enc(m)$ equals $\sigma^e\mod N$.
        \item[(b)] Because $\Dec(u^e\mod N)$ may not has small length.
        \item[(c)] $\Sign_{pk}(1)=2^{d\kappa/10}$ and $\Sign_{pk}(10^{\kappa/10})=2^{2\cdot d\kappa/10}$.
        \item[(d)] $\Sign_{pk}(1)=\pbra{2^{\ell+1}+1}^d,\Sign_{pk}(10)=2^d\pbra{2^{\ell+1}+1}^d$ and 
            $\Sign_{pk}(100)=4^d\pbra{2^{\ell+1}+1}^d$.
    \end{itemize}
\end{exercise}

\begin{exercise}{12.6}
    Assume $\widehat\Pi=(\widehat\Gen,\Pcal_1,\Pcal_2,\Vcal)$ is a secure identification scheme.
    Then in the variant $\Pi=(\Gen,\Sign,\Vrfy)$,
    \begin{itemize}
        \item $\Gen(1^n)$ runs $\widehat\Gen(1^n)$, gets $(pk,sk)$, and generate $H:\bin^*\to\Omega_{pk}$. 
            (assume $H$ is random oracle)
        \item $\Sign_{sk}(m)=(I,s)$, where $(I,st)\gets\Pcal_1(sk),r:=H(I,m),s:=\Pcal_2(sk,st,r)$.
        \item $\Vrfy_{pk}(m,I,s)=1$ iff $I=\Vcal(pk,r,s)=\Vcal(pk,H(I,m),s)$.
    \end{itemize}
    For any \ppt adversary $\Acal$ of $\Pi$, we may always assume $\Acal$ knows the value of $H(I,m)$ 
    before outputting forgery on $m$; and $\Acal$ never queries $H(I,m)$ twice or ask $H(I,m)$ after receive signature
    $(I,r,s)$ of $m$.
    Let $q(n)$ be the upper bound of $\Acal$'s query number on $H$.
    Then construct an adversary $\Bcal$ of $\widehat\Pi$:
    \begin{enumerate}
        \item $\Bcal$ is given $pk,\Trans_{sk}$. Choose uniform $j\sim[q(n)]$.
        \item Run $\Acal(pk)$. 
            \begin{itemize}
            \item Whenever $\Acal$ asks $H(I_i,m_i)$, 
                \begin{itemize}
                    \item if $i=j$, output $I_i$ and get $r_i$ from the outer experiment; then return $r_i$.
                    \item if $i\neq j$, return a uniform $r_i$.
                \end{itemize}
            \item Whenever $\Acal$ asks for the signature of $m_i$, let $(I_i,r_i,s_i)\gets\Trans_{sk}$
                and return $s_i$.
            \end{itemize}
        \item When $\Acal$ gives forgery $(m,I,s)$, output $s$.
    \end{enumerate}
    Therefore, there exists a negligible function $\negl$ such that
    \begin{align*}
        \negl(n)&\geq\Pr\sbra{\Ident_{\Bcal,\widehat\Pi}(n)=1}\\
        &\geq\Pr\sbra{\Sigforge_{\Acal,\Pi}(n)=1,I_j=I, m_j=m,
        H\text{ query is consistent with signature query}}\\
        &=\frac1{q(n)}\Pr\sbra{\Sigforge_{\Acal,\Pi}(n)=1,
        H\text{ query is consistent with signature query}}\\
        &\geq\frac1{q(n)}\pbra{\Pr\sbra{\Sigforge_{\Acal,\Pi}(n)=1}
        -\Pr\sbra{H\text{ query is not consistent with signature query}}}\\
        &\geq\frac1{q(n)}\pbra{\Pr\sbra{\Sigforge_{\Acal,\Pi}(n)=1}-\negl'(n)},
    \end{align*}
    where $\negl'(n)$ comes from the negligible possibility of $\Trans_{sk}$ having the same $I,r$ as
    $\Bcal$ answers $H$ queries.
\end{exercise}

\begin{exercise}{12.7}
    Assume we have $\pbra{g^k,\sbra{k^{-1}\cdot(m+xr)\mod q}}$, then $F(g^k)=r$.
    Let $t$ be an arbitrary number and $r'=F(g^{tk})$.
    Now we have a signature of message $m'=\frac{r'm}{r}$ as
    $$
    \pbra{g^{tk},\sbra{(tk)^{-1}\cdot(m'+xr')\mod p}}=\pbra{g^{tk},\sbra{\frac{r'}{rt}\cdot k^{-1}\cdot(m+xr)\mod p}}.
    $$
\end{exercise}

\begin{exercise}{12.8}
\begin{itemize}
    \item[(a)] 
        \begin{answer}
            Given signature $(i,f^{(n-i)}(x))$, one can forge signatures
            $(j,f^{(n-j)}(x))=(j,f^{(i-j)}(f^{(n-i)}(x))),\forall j<i$.
            Thus the above is not a one-time-secure signature scheme.
        \end{answer}
    \item[(b)] 
        \begin{proof}
            For any \ppt adversary $\Acal$, which gives a signature on $1\leq i<j\leq n$, 
            construct an adversary $\Bcal$ to invert $f$:
            \begin{enumerate}
                \item $\Bcal$ is given $y$. Pick a uniform $j^*\in\cbra{i+1,i+2,\cdots,n}$.
                \item Run $\Acal\pbra{f^{(j^*-1)}(y),(i,f^{(j^*-i-1)}(y))}$ and get $(j,\sigma),j>i$.
                \item If $j=j*$ output $\sigma$; otherwise abort.
            \end{enumerate}
            Therefore, there exists a negligible function $\negl$ such that
            \begin{align*}
                \negl(n)&\geq\Pr_x\sbra{f(\Bcal(f(x)))=f(x)}=\Pr_x\sbra{\Bcal(f(x))=x}\\
                &=\Pr_{x,j^*}\sbra{\Acal\pbra{f^{(j^*-1)}(y),(i,f^{(j^*-i-1)}(y))}=(j^*,f^{(-1)}(y)),y=f(x)}\\
                &=\frac1{n-i}\sum_{j^*=i+1}^n
                \Pr_x\sbra{\Acal\pbra{f^{(j^*)}(x),(i,f^{(j^*-i)}(x))}=(j^*,x)}\\
                &=\frac1{n-i}\sum_{j^*=i+1}^n
                \Pr_x\sbra{\Acal\pbra{f^{(n)}(t),(i,f^{(n-i)}(t))}=(j^*,f^{(n-j^*)}(t)),t=f^{(j^*-n)}(x)}\\
                &=\frac1{n-i}\sum_{j^*=i+1}^n
                \Pr_t\sbra{\Acal\pbra{f^{(n)}(t),(i,f^{(n-i)}(t))}=(j^*,f^{(n-j^*)}(t))}\\
                &=\frac1{n-i}\Pr\sbra{\Acal\text{ succeeds}}.
            \end{align*}
            Thus the one-wayness of $f$ implies the negligible possibility of 
            $\Acal$ producing signature on any message $j>i$.
        \end{proof}
    \item[(c)] 
        \begin{construction}
            Assume $f$ is a one-way permutation. Define $\Pi=(\Gen,\Sign,\Vrfy)$, where
            \begin{itemize}
                \item $\Gen(1^n)$ : choose uniform $x,x'\in\bin^n$ and set $y=f^{(n)}(x),y'=f^{(n)}(x')$. 
                    The public key is $pk=(y,y')$ and private key is $sk=(x,x')$.
                \item $\Sign_{sk}(i)=\pbra{f^{(n-i)}(x),f^{(i)}(x')}$.
                \item $\Vrfy_{pk}(i,\sigma,\sigma')=1$ iff $y=f^{(i)}(\sigma),y'=f^{(n-i)}(\sigma')$.
            \end{itemize}
        \end{construction}
        \begin{proof}
            For any \ppt adversary $\Acal$ of $\Pi$, construct an adversary $\Bcal$ to invert $f$:
            \begin{enumerate}
                \item $\Bcal$ is given $y$. Choose uniform $j^*\sim[n],b\sim\bin,x'\sim\bin^n$.
                \item If $b=0$, run $\Acal\pbra{f^{(j^*-1)}(y),f^{(n)}(x')}$. When $\Acal$ queries the signature on $i$,
                    if $j^*\geq i+1$ return $\pbra{f^{(j^*-i-1)}(y),f^{(i)}(x')}$; otherwise abort.
                    When $\Acal$ gives a valid forgery $(j,\sigma,\sigma')$.
                    If $j=j^*$ output $\sigma$; otherwise abort.
                \item If $b=1$, run $\Acal\pbra{f^{(n)}(x'),f^{(n-j^*-1)}(y)}$. When $\Acal$ queries the signature on $i$,
                    if $i\geq j^*+1$ return $\pbra{f^{(n-i)}(x'),f^{(i-j^*-1)}(y)}$; otherwise abort.
                    When $\Acal$ gives a valid forgery $(j,\sigma,\sigma')$.
                    If $j=j^*$ output $\sigma'$; otherwise abort.
            \end{enumerate}
            Thus there exists a negligible function $\negl$ such that
            \begin{align*}
                \negl(n)&\geq\Pr\sbra{f(\Bcal(f(x)))=f(x)}\\
                &=\frac12\left(
                \Pr\sbra{j^*\geq i+1,\Acal\pbra{f^{(j^*-1)}(y),f^{(n)}(x')}=\pbra{j^*,f^{(-1)}(y),\cdot}}\right.\\
                &\quad
                \left.+\Pr\sbra{j^*\leq i-1,\Acal\pbra{f^{(n)}(x'),f^{(n-j^*-1)}(y)}=\pbra{j^*,\cdot,f^{(-1)}(y)}}\right)\\
                &=\frac1{2n}\left(
                \Pr\sbra{\Acal\pbra{f^{(j-1)}(y),f^{(n)}(x')}=\pbra{j,f^{(-1)}(y),\cdot},j>i}\right.\\
                &\quad
                \left.+\Pr\sbra{\Acal\pbra{f^{(n)}(x'),f^{(n-j-1)}(y)}=\pbra{j,\cdot,f^{(-1)}(y)},j<i}\right)\\
                &=\frac1{2n}\left(
                \Pr\sbra{\Acal\pbra{f^{(n)}(x''),f^{(n)}(x')}=\pbra{j,f^{(n-j)}(x''),\cdot},j>i}\right.\\
                &\quad
                \left.+\Pr\sbra{\Acal\pbra{f^{(n)}(x'),f^{(n)}(x'')}=\pbra{j,\cdot,f^{(j)}(x'')},j<i}\right)\\
                &=\frac1{2n}
                \Pr\sbra{\Acal\pbra{f^{(n)}(x'),f^{(n)}(x'')}=\pbra{j,u,v},u=f^{(n-j)}(x')\lor v=f^{(j)}(x'),j\neq i}\\
                &=\frac1{2n}\Pr\sbra{\Sigforge^\onetime_{\Acal,\Pi}(n)=1}.
            \end{align*}
        \end{proof}
\end{itemize}
\end{exercise}

\begin{exercise}{12.9}
    \begin{itemize}
        \item[(b)]
            Assume $f$ is an OWF, then $f'$ is also an OWF where
            $$
            f'(x)=\begin{cases}
                f(x) & x\text{ is even}\\
                f(x-1) & x\text{ is odd}.
            \end{cases}
            $$
            But the Lamport's scheme using $f'$ is apparently not strongly one-time-secure.
        \item[(c)]
            Use RSA (or any other OWP).
    \end{itemize}
\end{exercise}

\begin{exercise}{12.10}
    The adversary can forge signature of $x_1x_2\cdots x_n$ if given signature of 
    $\bar x_1x_2\cdots x_n$ and $x_1\overline{x_2\cdots x_n}$.
\end{exercise}

\begin{exercise}{12.11}
\begin{proof} 
    Assume $y_i=H(x_i)$, where $H$ is a OWF.
    For any \ppt adversary $\Acal$ of this Lamport scheme, we may always assume it queries one signature, since
    it can query an irrelevant signature right before outputting forgery.
    Then construct an adversary $\Bcal$ to invert $H$:
    \begin{enumerate}
        \item $\Bcal$ is given $y$. Choose uniform $k\sim\cbra{1,\cdots,2\ell}$.
        \item For any $i\in[2\ell]\backslash\{k\}$, choose uniform $x_i\sim\bin^n$.
            Let $y_i=H(x_i),i\neq k$ and $y_k=y$.
        \item Run $\Acal(y_1,\cdots,y_{2\ell})$. When $\Acal$ queries signature of $m$, if $k\notin S_m$ then
            reveal $\{x_i\}_{i\in S_m}$; otherwise abort.
        \item When $\Acal$ gives forgery $(m',\cbra{x_i'}_{i\in S_{m'}})$, if $k\in S_{m'}$ then output $x_k'$;
            otherwise abort.
    \end{enumerate}
    Therefore, there exists a negligible function $\negl$ such that
    \begin{align*}
        \negl(n)&\geq\Pr\sbra{H(\Bcal(H(x)))=H(x)}\\
        &=\Pr\sbra{k\notin S_m,k\in S_{m'},H(x_k')=H(x)}\\
        &\geq\Pr\sbra{k\notin S_m,k\in S_{m'},\cbra{x_i'}_{i\in S_{m'}}\text{ is valid forgery}}\\
        &\geq\frac1{2\ell}\sum_{k=1}^{2\ell}
            \Pr\sbra{k\in S_{m'}\backslash S_m,\cbra{x_i'}_{i\in S_{m'}}\text{ is valid forgery}}\\
        &=\frac1{2\ell}\Pr\sbra{\Acal\text{ succeeds}},
    \end{align*}
    which proves the one-time-security since $\ell$ is polynomial in $n$.
\end{proof}
\begin{answer}
    Since the mapping should be one-to-one, we have $2^{\ell'}\leq\binom{2\ell}{\ell}$.
    Thus 
    $$
    \ell'\leq\log\binom{2\ell}{\ell}\approx
    \log\pbra{\frac{\sqrt{4\pi\ell}\pbra{\frac{2\ell}{e}}^{2\ell}}
    {\sqrt{2\pi\ell}\pbra{\frac{\ell}{e}}^\ell\cdot\sqrt{2\pi\ell}\pbra{\frac{\ell}{e}}^\ell}}
    =2\ell-\frac12\log\pbra{\pi\ell}.
    $$
\end{answer}
\end{exercise}

\begin{exercise}{12.14}
    Since the statement is signed with respect to $pk_B$ and $pk_B$ is known to public, 
    anyone can forge such signature. Thus Bob's identity can not be verified in this case.
\end{exercise}


\end{document}
