% !TeX encoding = UTF-8
% !TeX program = XeLaTeX
% !TeX spellcheck = LaTeX

% Author : Shlw

\documentclass[a4paper]{article}

\usepackage{amsmath,amsfonts,amssymb}
\usepackage{mathrsfs}
\usepackage{bm}
\usepackage{geometry}
\usepackage{ntheorem}
\usepackage{hyperref}
\usepackage[ruled]{algorithm2e}
\usepackage{caption,subcaption}

\geometry{left=2cm,right=2cm,top=2cm,bottom=2cm}

\def\UrlBreaks{\do\A\do\B\do\C\do\D\do\E\do\F\do\G\do\H\do\I\do\J\do\K\do\L\do\M\do\N\do\O\do\P\do\Q\do\R\do\S\do\T\do\U\do\V\do\W\do\X\do\Y\do\Z\do\[\do\\\do\]\do\^\do\_\do\`\do\a\do\b\do\c\do\d\do\e\do\f\do\g\do\h\do\i\do\j\do\k\do\l\do\m\do\n\do\o\do\p\do\q\do\r\do\s\do\t\do\u\do\v\do\w\do\x\do\y\do\z\do\0\do\1\do\2\do\3\do\4\do\5\do\6\do\7\do\8\do\9\do\.\do\@\do\\\do\/\do\!\do\_\do\|\do\;\do\>\do\]\do\)\do\,\do\?\do\'\do+\do\=\do\#}

\newtheorem{theorem}{Theorem}
\newtheorem{lemma}{Lemma}
\newtheorem{proposition}{Proposition}
\newtheorem{corollary}{Corollary}
\newtheorem{claim}{Claim}
\newtheorem{conjecture}{conjecture}
\newtheorem{definition}{Definition}
\newtheorem{construction}{Construction}
\newtheorem*{proof}{Proof}
\newtheorem*{answer}{Answer}
\newtheorem*{refute}{Refute}
\newtheorem*{example}{Example}
\newtheorem*{counterexample}{Counterexample}

\newenvironment{exercise}[1]{
	\par
	\noindent\textbf{Exercise #1.}\quad
}{
	\par
	\bigskip
}

\DeclareMathAccent{\widehat}{\mathord}{largesymbols}{"62}
\DeclareMathOperator{\lequiv}{\ \Leftrightarrow\ }
\newcommand{\rawE}{\mathop{\mathbb E}}
\newcommand{\E}[1]{\mathop{\mathbb E}_{#1}}
\newcommand{\abs}[1]{\left| #1 \right|}
\newcommand{\pbra}[1]{\left( #1 \right)}
\newcommand{\cbra}[1]{\left\{ #1 \right\}}
\newcommand{\sbra}[1]{\left[ #1 \right]}
\newcommand{\bin}{\{0,1\}}
\newcommand{\Enc}{\mathrm{Enc}}
\newcommand{\Gen}{\mathrm{Gen}}
\newcommand{\Dec}{\mathrm{Dec}}
\newcommand{\Mac}{\mathrm{Mac}}
\newcommand{\Vrfy}{\mathrm{Vrfy}}
\newcommand{\PrivK}{\mathrm{PrivK}}
\newcommand{\Macforge}{\mathrm{Mac}\text{-}\mathrm{forge}}
\newcommand{\Macsforge}{\mathrm{Mac}\text{-}\mathrm{sforge}}
\newcommand{\Encforge}{\mathrm{Enc}\text{-}\mathrm{Forge}}
\newcommand{\negl}{\mathrm{negl}}
\newcommand{\ppt}{{\sc ppt} }
\newcommand{\eav}{\mathrm{eav}}
\newcommand{\out}{\mathrm{out}}
\newcommand{\mult}{\mathrm{mult}}
\newcommand{\cpa}{\mathrm{cpa}}
\newcommand{\cca}{\mathrm{cca}}
\newcommand{\Used}{\mathrm{Used}}
\newcommand{\Asked}{\mathrm{Asked}}

\title{Homework Set $2$}
\date{}

\begin{document}

\maketitle

\begin{exercise}{4.1}
    Let $\mathcal M$ be the message space.
    For any message $m$ and tag $u$, 
    consider a \ppt adversary $\mathcal A_{m,u}$ of $\Pi$, which simply outputs $(m,u)$.
    Therefore,
    \begin{align*}
        \E{m,u}\Pr\sbra{\Macforge_{\mathcal A_{m,u},\Pi}(n)=1}
        &=\E{m,u}\Pr\sbra{\Mac_k(m)=u}\\
        &=2^{-t(n)}\times|\mathcal M|^{-1}\times\sum_{u,m}\Pr\sbra{\Mac_k(m)=u}\\
        &=2^{-t(n)}.
    \end{align*}
    Since $\Pi$ is a secure MAC and $\mathcal M$ is finite, for any fixed $c\in\mathbb N$
    $$
    2^{-t(n)}\leq\E{m,u}\Pr\sbra{\Macforge_{\mathcal A_{m,u},\Pi}(n)=1}\leq\frac{1}{n^c},
    $$
    which gives $t(n)=\omega(\log n)$.
\end{exercise}

\begin{exercise}{4.6}
    $\Pi=(\Gen,\Mac,\Vrfy)$ is not secure.

    Consider the following \ppt adversary $\mathcal A$:
    \begin{enumerate}
        \item Ask the tag of $0^{n-1}\|0^{n-1}$, thus obtaining $c_0=F_k(0^n)$.
        \item Ask the tag of $1^{n-1}\|1^{n-1}$, thus obtaining $c_1=F_k(1^n)$.
        \item Output $(0^{n-1}\|1^{n-1},c_0\|c_1)$.
    \end{enumerate}
    Therefore, $\Pr\sbra{\Macforge_{\mathcal A,\Pi}(n)=1}=1$.
\end{exercise}

\begin{exercise}{4.12}
The advantage is 
\begin{itemize}
    \item to process while reading message;
    \item to have fewer blocks thus accelerating the procedure;
    \item to be able to authenticate potentially longer message.
\end{itemize}
\begin{proof}[New Construction $4.7$ is secure]
    Consider an arbitrary \ppt adversary $\mathcal A$, which makes at most $q(n)$ queries.
    Let $(m_1,\cdots,m_d)$ be the output message and $(r,t_1,\cdots,t_d)$ be the output tag.
    $\mathrm{Repeat}$ denotes $r$ is used more than once during query;
    $\mathrm{NewBlock}$ denotes there exists $m_i\in[d-1]$ such that $r\|0\|i\|m_i$ is new
    or $r\|1\|d\|m_d$ is new.
    Therefore, 
    \begin{align*}
        &\Pr\sbra{\Macforge_{\mathcal A,\Pi}(n)=1}\\
        =&\Pr\sbra{\Macforge_{\mathcal A,\Pi}(n)=1,\mathrm{Repeat}}
        +\Pr\sbra{\Macforge_{\mathcal A,\Pi}(n)=1,\overline{\mathrm{Repeat}},\mathrm{NewBlock}}\\
        &+\Pr\sbra{\Macforge_{\mathcal A,\Pi}(n)=1,\overline{\mathrm{Repeat}},\overline{\mathrm{NewBlock}}}\\
        \leq&\Pr\sbra{\mathrm{Repeat}}
        +\Pr\sbra{\Macforge_{\mathcal A,\Pi}(n)=1,\mathrm{NewBlock}}
        +\Pr\sbra{\Macforge_{\mathcal A,\Pi}(n)=1,\overline{\mathrm{Repeat}},\overline{\mathrm{NewBlock}}}.
    \end{align*}
    It is easy to see
    $$
        \Pr\sbra{\mathrm{Repeat}}\leq\sum_{i=1}^{q(n)}\sum_{j\neq i}2^{-n/4}\leq\frac{q^2(n)}{2^{n/4}}.
    $$
    On the other hand, if $\mathrm{NewBlock}$ and $\mathrm{Repeat}$ do not occur but the forgery is valid,
    there exists a query identical with $(m_1,\cdots,m_d)$, which is a contradiction. Thus
    $$
    \Pr\sbra{\Macforge_{\mathcal A,\Pi}(n)=1,\overline{\mathrm{Repeat}},\overline{\mathrm{NewBlock}}}=0. 
    $$
    
    Let $\widetilde\Pi$ be the MAC same as $\Pi$ except that a FRF $f$ is used in place of $F$.
    Let $\ell$ be the expansion factor of $F$.
    Develop a \ppt algorithm $\mathcal D$ as follows:
    \begin{enumerate}
        \item $\mathcal D$ is given $1^n$ and oracle $\mathcal O$.
        \item Run $\mathcal A(1^n)$. Whenever $\mathcal A$ asks to authenticate message $m'=(m'_1,\cdots,m'_l)$,
            generate $r'\in\{0,1\}^{n/4}$ uniformly and return $(r',\mathcal O(r'\|0\|1\|m'_1),\cdots,
            \mathcal O(r'\|1\|l\|m'_l))$. Then store $m'$ in memory.
        \item When $\mathcal A$ outputs $m=(m_1,\cdots,m_d)$ and $(r,t_1,\cdots,t_d)$,
            \begin{itemize}
                \item if $m$ is in memory, 
                    or $(r,\mathcal O(r\|0\|1\|m_1),\cdots,\mathcal O(r\|1\|d\|m_d))\neq(r,t_1,\cdots,t_d)$,
                    or $\mathrm{NewBlock}$ does not occur,
                    output $0$;
                \item otherwise,
                    output $1$.
            \end{itemize}
    \end{enumerate}
    Hence, 
    \begin{gather*}
    \Pr\sbra{\mathcal D^{F_k(\cdot)}(1^n)=1}=\Pr\sbra{\Macforge_{\mathcal A,\Pi}(n)=1,\mathrm{NewBlock}}\\
    \Pr\sbra{\mathcal D^{f(\cdot)}(1^n)=1}=\Pr\sbra{\Macforge_{\mathcal A,\widetilde\Pi}(n)=1,\mathrm{NewBlock}}.
    \end{gather*}
    Since in new block, the successful rate is extremely low under FRF, which gives
    \begin{gather*}
    \Pr\sbra{\mathcal D^{f(\cdot)}(1^n)=1}=\Pr\sbra{\Macforge_{\mathcal A,\widetilde\Pi}(n)=1,\mathrm{NewBlock}}\leq 
        2^{-\ell(n)}\leq 2^{-n}.
    \end{gather*}
    Combining the definition of PRF, there exists a negligible function $\negl$ such that
    \begin{align*}
        \Pr\sbra{\Macforge_{\mathcal A,\Pi}(n)=1,\mathrm{NewBlock}}\leq 2^{-n}+\Pr\sbra{\mathcal D^{F_k(\cdot)}(1^n)=1}
        -\Pr\sbra{\mathcal D^{f(\cdot)}(1^n)=1}\leq 2^{-n}+\negl(n).
    \end{align*}
    As a result, 
    \begin{align*}
        &\Pr\sbra{\Macforge_{\mathcal A,\Pi}(n)=1}\\
        \leq&\Pr\sbra{\mathrm{Repeat}}
        +\Pr\sbra{\Macforge_{\mathcal A,\Pi}(n)=1,\mathrm{NewBlock}}
        +\Pr\sbra{\Macforge_{\mathcal A,\Pi}(n)=1,\overline{\mathrm{Repeat}},\overline{\mathrm{NewBlock}}}\\
        \leq&\frac{q^2(n)}{2^{n/4}}+2^{-n}+\negl(n),
    \end{align*}
    which proves the security of new Construction $4.7$.
\end{proof}
\end{exercise}

\begin{exercise}{4.25}

\begin{proof}[Not authenticated]
    Since $F_k$ is permutation, every ciphertext $c\in\bin^n$ is valid.
    Therefore, a \ppt adversary which simply outputs ciphertext $0^n$ will pass the $\Encforge$ experiment.
\end{proof}

\begin{proof}[CCA-secure]
    Let $\Pi$ be the proposed scheme and $\widetilde\Pi$
    be same as $\Pi$ except that FRP $f$ is used in place of $F$.
    For any \ppt adversary $\mathcal A$ of $\Pi$, develop a \ppt algorithm $\mathcal D$ as follows:
    \begin{enumerate}
        \item $\mathcal D$ is given input $1^n$ and access to oracle $\mathcal O:\bin^n\to\bin^n$ and 
            $\mathcal O^{-1}$.
        \item Run $\mathcal A(1^n)$. Whenever $\mathcal A$ queries the 
            \begin{itemize}
                \item encryption of $m\in\bin^{n/2}$, 
                    choose a uniform $r\in\bin^{n/2}$ and return $\mathcal O(m\|r)$ to $\mathcal A$.
                \item decryption of $c\in\bin^{n}$, 
                    return the second half of $\mathcal O^{-1}(c)$ to $\mathcal A$.
            \end{itemize}
        \item When $\mathcal A$ outputs $m_0,m_1\in\bin^{n/2}$, choose a uniform bit $b\in\bin$ and
            uniform $r^*\in\bin^{n/2}$, then return $c^*=\mathcal O(m_b\|r^*)$ to $\mathcal A$.
        \item Continue answering the encryption (decryption) query from $\mathcal A$ as long as it does not ask to
            decrypt $c^*$ until $\mathcal A$ gives a bit $b'$.
            Output $1$ if $b'=b$, and $0$ otherwise.
    \end{enumerate}
    When $\mathcal O$ is a FRP, we have
        $$
        \Pr\sbra{\mathcal D^{f(\cdot),f^{-1}(\cdot)}(1^n)=1}=\Pr\sbra{\PrivK_{\mathcal A,\widetilde\Pi}^\cca(n)=1}.
        $$
    When $\mathcal O$ is a SPRP, we have
        $$
        \Pr\sbra{\mathcal D^{F_k(\cdot),F_k^{-1}(\cdot)}(1^n)=1}=\Pr\sbra{\PrivK_{\mathcal A,\Pi}^\cca(n)=1}.
        $$
    Combining the definition of SPRP, there exists a negligible function $\negl$ such that
    $$
        \Pr\sbra{\PrivK_{\mathcal A,\Pi}^\cca(n)=1}\leq\negl(n)+\Pr\sbra{\PrivK_{\mathcal A,\widetilde\Pi}^\cca(n)=1}.
    $$

    Now, write the encryption and decryption oracle of $\PrivK_{\mathcal A,\widetilde\Pi}^\cca$ explicitly 
    in Experiment $1$.

    \begin{minipage}{\textwidth}
    \centering
    \begin{minipage}{0.46\textwidth}
        \centering
        \scriptsize
        \begin{algorithm}[H]
            \caption{Experiment $1$}
            \DontPrintSemicolon
            \SetKwFunction{enc}{Encrypt}
            \SetKwFunction{dec}{Decrypt}
            \SetKwProg{func}{Function}{}{}

            $M,C,S,T\gets\varnothing$\;
            \func{\enc{$m\in\bin^{n/2}$}}{
                $r\sim\bin^{n/2},u\gets m\|r$\;
                \If{$u\notin M$}{
                    $c\sim\bin^n\backslash C$\;
                    $M\gets M\cup\{u\},C\gets C\cup\{c\}$\;
                    $S[u]\gets c,T[c]\gets u$\;
                }
                \KwRet $S[u]$\;
            }
            \func{\dec{$c\in\bin^{n}$}}{
                \If{$c\notin C$}{
                    $u\sim\bin^n\backslash M$\;
                    $M\gets M\cup\{u\},C\gets C\cup\{c\}$\;
                    $S[u]\gets c,T[c]\gets u$\;
                }
                \KwRet first half of $T[c]$\;
            }
        \end{algorithm}
    \end{minipage}
    \begin{minipage}{0.46\textwidth}
        \centering
        \scriptsize
        \begin{algorithm}[H]
            \caption{Experiment $2$}
            \DontPrintSemicolon
            \SetKwFunction{enc}{Encrypt}
            \SetKwFunction{dec}{Decrypt}
            \SetKwProg{func}{Function}{}{}

            $M,C,S,T,R\gets\varnothing$\;
            \func{\enc{$m\in\bin^{n/2}$}}{
                $r\sim\bin^{n/2}\backslash R,u\gets m\|r$\;
                \If{$u\notin M$}{
                    $c\sim\bin^n$\;
                    $M\gets M\cup\{u\},C\gets C\cup\{c\}$\;
                    $S[u]\gets c,T[c]\gets u,R\gets R\cup\{r\}$\;
                }
                \KwRet $S[u]$\;
            }
            \func{\dec{$c\in\bin^{n}$}}{
                \If{$c\notin C$}{
                    $u\gets m\|r\sim\bin^{n/2}\times(\bin^{n/2}\backslash R)$\;
                    $M\gets M\cup\{u\},C\gets C\cup\{c\}$\;
                    $S[u]\gets c,T[c]\gets u,R\gets R\cup\{r\}$\;
                }
                \KwRet first half of $T[c]$\;
            }
        \end{algorithm}
    \end{minipage}
\end{minipage}


    Applying Lemma \ref{ap} in Appendix for multiple times and introducing randomness recorder $R$, 
    Experiment $1$ is indistinguishable from Experiment $2$.
    Then the $\mathtt{if}$ condition is always satisfied in $\mathtt{Encrypt}$, which gives rise to 
    Experiment $3$. Then using Lemma \ref{ap} to remove the dependence of $R$, we have Experiment $4$.

    \begin{minipage}{\textwidth}
    \centering
    \begin{minipage}{0.46\textwidth}
        \centering
        \scriptsize
        \begin{algorithm}[H]
            \caption{Experiment $3$}
            \DontPrintSemicolon
            \SetKwFunction{enc}{Encrypt}
            \SetKwFunction{dec}{Decrypt}
            \SetKwProg{func}{Function}{}{}

            $M,C,S,T,R\gets\varnothing$\;
            \func{\enc{$m\in\bin^{n/2}$}}{
                $r\sim\bin^{n/2}\backslash R,u\gets m\|r$\;
                $c\sim\bin^n$\;
                $M\gets M\cup\{u\},C\gets C\cup\{c\}$\;
                $S[u]\gets c,T[c]\gets u,R\gets R\cup\{r\}$\;
                \KwRet $S[u]$\;
            }
            \func{\dec{$c\in\bin^{n}$}}{
                \If{$c\notin C$}{
                    $u\gets m\|r\sim\bin^{n/2}\times(\bin^{n/2}\backslash R)$\;
                    $M\gets M\cup\{u\},C\gets C\cup\{c\}$\;
                    $S[u]\gets c,T[c]\gets u,R\gets R\cup\{r\}$\;
                }
                \KwRet first half of $T[c]$\;
            }
        \end{algorithm}
    \end{minipage}
    \begin{minipage}{0.46\textwidth}
        \centering
        \scriptsize
        \begin{algorithm}[H]
            \caption{Experiment $4$}
            \DontPrintSemicolon
            \SetKwFunction{enc}{Encrypt}
            \SetKwFunction{dec}{Decrypt}
            \SetKwProg{func}{Function}{}{}

            $M,C,S,T\gets\varnothing$\;
            \func{\enc{$m\in\bin^{n/2}$}}{
                $r\sim\bin^{n/2},u\gets m\|r$\;
                $c\sim\bin^n$\;
                $M\gets M\cup\{u\},C\gets C\cup\{c\}$\;
                $S[u]\gets c,T[c]\gets u$\;
                \KwRet $S[u]$\;
            }
            \func{\dec{$c\in\bin^{n}$}}{
                \If{$c\notin C$}{
                    $u\gets m\|r\sim\bin^{n}$\;
                    $M\gets M\cup\{u\},C\gets C\cup\{c\}$\;
                    $S[u]\gets c,T[c]\gets u$\;
                }
                \KwRet first half of $T[c]$\;
            }
        \end{algorithm}
    \end{minipage}
\end{minipage}


    Now, eliminate redundant variables and calls to obtain Experiment $5$.
    We may always assume $\mathcal A$ never asks the decryption of ciphertext which $\mathcal A$ has obtained,
    which means the $\mathtt{if}$ condition is always satisfied in $\mathtt{Decrypt}$.
    Hence we gets Experiment $6$.

    \begin{minipage}{\textwidth}
    \centering
    \begin{minipage}{0.46\textwidth}
        \centering
        \scriptsize
        \begin{algorithm}[H]
            \caption{Experiment $5$}
            \DontPrintSemicolon
            \SetKwFunction{enc}{Encrypt}
            \SetKwFunction{dec}{Decrypt}
            \SetKwProg{func}{Function}{}{}

            $C,T\gets\varnothing$\;
            \func{\enc{$m\in\bin^{n/2}$}}{
                $r\sim\bin^{n/2},u\gets m\|r$\;
                $c\sim\bin^n$\;
                $T[c]\gets u,C\gets C\cup\{c\}$\;
                \KwRet $c$\;
            }
            \func{\dec{$c\in\bin^{n}$}}{
                \If{$c\notin C$}{
                    $u\sim\bin^{n}$\;
                    $T[c]\gets u,C\gets C\cup\{c\}$\;
                }
                \KwRet first half of $T[c]$\;
            }
        \end{algorithm}
    \end{minipage}
    \begin{minipage}{0.46\textwidth}
        \centering
        \scriptsize
        \begin{algorithm}[H]
            \caption{Experiment $6$}
            \DontPrintSemicolon
            \SetKwFunction{enc}{Encrypt}
            \SetKwFunction{dec}{Decrypt}
            \SetKwProg{func}{Function}{}{}

            \func{\enc{$m\in\bin^{n/2}$}}{
                $c\sim\bin^n$\;
                \KwRet $c$\;
            }
            \func{\dec{$c\in\bin^{n}$}}{
                $m\sim\bin^{n/2}$\;
                \KwRet $m$\;
            }
        \end{algorithm}
    \end{minipage}
\end{minipage}


    Note that by Lemma \ref{ap}, Experiment $6$ and $1$ are indistinguishable, which means there is a negligible function
    $\negl'$ such that
    \begin{align*}
        \Pr\sbra{\PrivK_{\mathcal A,\widetilde\Pi}^\cca(n)=1}&=\Pr\sbra{\text{Experiment $1$ outputs $1$}}\\
        &\leq\Pr\sbra{\text{Experiment $6$ outputs $1$}}+\negl'(n)\\
        &\leq\frac12+\negl'(n).
    \end{align*}
    Therefore, the CCA-security of $\Pi$ naturally follows from
    \begin{align*}
        \Pr\sbra{\PrivK_{\mathcal A,\Pi}^\cca(n)=1}
        &\leq\negl(n)+\Pr\sbra{\PrivK_{\mathcal A,\widetilde\Pi}^\cca(n)=1}\\
        &=\frac12+\negl'(n)+\negl(n).
    \end{align*}
\end{proof}
\end{exercise}

\newpage
\section*{Appendix}

\begin{lemma}\label{ap}
    Assume $\mathcal A$ is a \ppt algorithm and $x\sim S$ is an assignment in $\mathcal A$, where $S$ is a constant.
    Then change this assignment into $x\sim S\backslash X$, where $X$ is maintained during $\mathcal A$ 
    and always of size negligible to $S$.
    The new algorithm $\mathcal A'$ is indistinguishable from $\mathcal A$.
\end{lemma}
\begin{proof}
    Since $\mathcal A$ is \ppt algorthm, assume it calls the assignment at most $q$ times.
    Hence
    $$
        \Pr\sbra{\mathcal A'\text{ runs like }\mathcal A}\geq
            1-q\times \Pr\sbra{\text{$x\sim S$ falls into $x\in X$}}=1-\text{negligible possibility},
    $$
    which indicates the indistinguishability between $\mathcal A$ and $\mathcal A'$. 
\end{proof}

\end{document}
