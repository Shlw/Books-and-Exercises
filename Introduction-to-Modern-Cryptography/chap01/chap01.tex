% !TeX encoding = UTF-8
% !TeX program = XeLaTeX
% !TeX spellcheck = LaTeX

\documentclass[a4paper]{article}

\usepackage{amsmath,amsfonts,amssymb}
\usepackage{mathrsfs}
\usepackage{bm}
\usepackage{geometry}
\usepackage{ntheorem}
\usepackage{hyperref}
\usepackage[ruled]{algorithm2e}
\usepackage{caption,subcaption}

\geometry{left=2cm,right=2cm,top=2cm,bottom=2cm}

\def\UrlBreaks{\do\A\do\B\do\C\do\D\do\E\do\F\do\G\do\H\do\I\do\J\do\K\do\L\do\M\do\N\do\O\do\P\do\Q\do\R\do\S\do\T\do\U\do\V\do\W\do\X\do\Y\do\Z\do\[\do\\\do\]\do\^\do\_\do\`\do\a\do\b\do\c\do\d\do\e\do\f\do\g\do\h\do\i\do\j\do\k\do\l\do\m\do\n\do\o\do\p\do\q\do\r\do\s\do\t\do\u\do\v\do\w\do\x\do\y\do\z\do\0\do\1\do\2\do\3\do\4\do\5\do\6\do\7\do\8\do\9\do\.\do\@\do\\\do\/\do\!\do\_\do\|\do\;\do\>\do\]\do\)\do\,\do\?\do\'\do+\do\=\do\#}

\newtheorem{theorem}{Theorem}
\newtheorem{lemma}{Lemma}
\newtheorem{proposition}{Proposition}
\newtheorem{corollary}{Corollary}
\newtheorem{claim}{Claim}
\newtheorem{conjecture}{conjecture}
\newtheorem{definition}{Definition}
\newtheorem{construction}{Construction}
\newtheorem*{proof}{Proof}
\newtheorem*{answer}{Answer}
\newtheorem*{refute}{Refute}
\newtheorem*{example}{Example}
\newtheorem*{counterexample}{Counterexample}

\newenvironment{exercise}[1]{
	\par
	\noindent\textbf{Exercise #1.}\quad
}{
	\par
	\bigskip
}


\DeclareMathAccent{\widehat}{\mathord}{largesymbols}{"62}
\newcommand{\abs}[1]{\left| #1 \right|}
\newcommand{\pbra}[1]{\left( #1 \right)}
\newcommand{\cbra}[1]{\left\{ #1 \right\}}
\newcommand{\sbra}[1]{\left[ #1 \right]}
\newcommand{\bin}{\{0,1\}}

\title{Exercise Set --- Chapter $1$}
\date{}

\begin{document}

\maketitle

\begin{exercise}{1.3}
    Let $\Sigma=\{a,b,\dots,z\}$ and $\Gamma=\{A,B,\dots,Z\}$.
    Denote $[n]$ as $\{1,2,\dots,n\}$.
    \begin{definition}[$\mathrm{Num}$]
        $\mathrm{Num}:\Sigma\to\{0,1,\dots,25\}$ is $\mathrm{Num}(a)=0,\mathrm{Num}(b)=1,\cdots,\mathrm{Num}(z)=25.$
    \end{definition}
    \begin{definition}[$\mathrm{Cap}$]
        $\mathrm{Cap}:\Sigma\to\Gamma$ is $\mathrm{Cap}(a)=A,\mathrm{Cap}(b)=B,\cdots,\mathrm{Cap}(z)=Z.$
    \end{definition}
    \begin{definition}[$\mathrm{RShift}$]
        $\mathrm{RShift}:\Sigma\times\Sigma\to\Gamma$ is
        $$
        \mathrm{RShift}(\alpha,\beta)=\mathrm{Num}^{-1}(\mathrm{Num}(\alpha)+\mathrm{Num}(\beta)\mod 26).
        $$
    \end{definition}
    \begin{definition}[$\mathrm{LShift}$]
        $\mathrm{LShift}:\Sigma\times\Sigma\to\Gamma$ is
        $$
        \mathrm{LShift}(\alpha,\beta)=\mathrm{Num}^{-1}(\mathrm{Num}(\alpha)-\mathrm{Num}(\beta)\mod 26).
        $$
    \end{definition}

    Assume the maximum possible length of plaintext is $\ell$.
    Thus $\mathcal K=\bigcup_{i=1}^\ell\Sigma^i,\mathcal M=\bigcup_{i=1}^\ell\Sigma^i,\mathcal C=\bigcup_{i=1}^\ell\Gamma^i$.
    \begin{definition}[$\mathrm{Gen}$]
        $\mathrm{Gen}:\varnothing\to\mathcal K$ is a randomized algorithm,
        which randomly outputs an element from $\mathcal K$ with uniform distribution.
    \end{definition}
    \begin{definition}[$\mathrm{Enc}$]
        $\mathrm{Enc}:\mathcal K\times\mathcal M\to\mathcal C$ is a deterministic algorithm.
        Assume $k\in\Sigma^i,m\in\Sigma^j,i,j\in[\ell]$. Let $\hat k=k^\ell$, then
        $\mathrm{Enc}(k,m)=c$, where $c\in\Gamma^j$ and $c_t=\mathrm{Cap}(\mathrm{RShift}(m_t,\hat k_t)),t\in[j]$.
    \end{definition}
    \begin{definition}[$\mathrm{Dec}$]
        $\mathrm{Dec}:\mathcal K\times\mathcal C\to\mathcal M$ is a deterministic algorithm.
        Assume $k\in\Sigma^i,c\in\Gamma^j,i,j\in[\ell]$. Let $\hat k=k^\ell$, then
        $\mathrm{Dec}(k,c)=m$, where $m\in\Sigma^j$ and $m_t=\mathrm{LShift}(\mathrm{Cap}^{-1}(c_t),\hat k_t),t\in[j]$.
    \end{definition}
\end{exercise}

\end{document}
