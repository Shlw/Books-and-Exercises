% !TeX encoding = UTF-8
% !TeX program = XeLaTeX
% !TeX spellcheck = LaTeX

% Author : Shlw

\documentclass[a4paper]{article}

\usepackage{amsmath,amsfonts,amssymb}
\usepackage{mathrsfs}
\usepackage{bm}
\usepackage{geometry}
\usepackage{ntheorem}
\usepackage{hyperref}
\usepackage[ruled]{algorithm2e}
\usepackage{caption,subcaption}

\geometry{left=2cm,right=2cm,top=2cm,bottom=2cm}

\def\UrlBreaks{\do\A\do\B\do\C\do\D\do\E\do\F\do\G\do\H\do\I\do\J\do\K\do\L\do\M\do\N\do\O\do\P\do\Q\do\R\do\S\do\T\do\U\do\V\do\W\do\X\do\Y\do\Z\do\[\do\\\do\]\do\^\do\_\do\`\do\a\do\b\do\c\do\d\do\e\do\f\do\g\do\h\do\i\do\j\do\k\do\l\do\m\do\n\do\o\do\p\do\q\do\r\do\s\do\t\do\u\do\v\do\w\do\x\do\y\do\z\do\0\do\1\do\2\do\3\do\4\do\5\do\6\do\7\do\8\do\9\do\.\do\@\do\\\do\/\do\!\do\_\do\|\do\;\do\>\do\]\do\)\do\,\do\?\do\'\do+\do\=\do\#}

\newtheorem{theorem}{Theorem}
\newtheorem{lemma}{Lemma}
\newtheorem{proposition}{Proposition}
\newtheorem{corollary}{Corollary}
\newtheorem{claim}{Claim}
\newtheorem{conjecture}{conjecture}
\newtheorem{definition}{Definition}
\newtheorem{construction}{Construction}
\newtheorem*{proof}{Proof}
\newtheorem*{answer}{Answer}
\newtheorem*{refute}{Refute}
\newtheorem*{example}{Example}
\newtheorem*{counterexample}{Counterexample}

\newenvironment{exercise}[1]{
	\par
	\noindent\textbf{Exercise #1.}\quad
}{
	\par
	\bigskip
}

\DeclareMathAccent{\widehat}{\mathord}{largesymbols}{"62}
\DeclareMathOperator{\lequiv}{\ \Leftrightarrow\ }
\newcommand{\rawE}{\mathop{\mathbb E}}
\newcommand{\E}[1]{\mathop{\mathbb E}_{#1}}
\newcommand{\abs}[1]{\left| #1 \right|}
\newcommand{\pbra}[1]{\left( #1 \right)}
\newcommand{\cbra}[1]{\left\{ #1 \right\}}
\newcommand{\sbra}[1]{\left[ #1 \right]}
\newcommand{\bin}{\{0,1\}}
\newcommand{\Enc}{\mathrm{Enc}}
\newcommand{\Gen}{\mathrm{Gen}}
\newcommand{\Dec}{\mathrm{Dec}}
\newcommand{\Mac}{\mathrm{Mac}}
\newcommand{\Vrfy}{\mathrm{Vrfy}}
\newcommand{\PrivK}{\mathrm{PrivK}}
\newcommand{\Macforge}{\mathrm{Mac}\text{-}\mathrm{forge}}
\newcommand{\Macsforge}{\mathrm{Mac}\text{-}\mathrm{sforge}}
\newcommand{\Encforge}{\mathrm{Enc}\text{-}\mathrm{Forge}}
\newcommand{\negl}{\mathrm{negl}}
\newcommand{\ppt}{{\sc ppt} }
\newcommand{\eav}{\mathrm{eav}}
\newcommand{\out}{\mathrm{out}}
\newcommand{\mult}{\mathrm{mult}}
\newcommand{\cpa}{\mathrm{cpa}}
\newcommand{\cca}{\mathrm{cca}}
\newcommand{\Used}{\mathrm{Used}}
\newcommand{\Asked}{\mathrm{Asked}}

\title{Exercise Set --- Chapter $3$}
\date{}

\begin{document}

\maketitle

\begin{exercise}{3.4}
    Let $\Pi=(\Gen,\Enc,\Dec)$ be a private-key encryption scheme and $\mathcal A$ be an arbitrary \ppt adversary.
    Then
    \begin{align*}
        &\Pr\sbra{\PrivK_{\mathcal A,\Pi}^{\eav}(n)=1}\leq\frac12+\negl_{\mathcal A}(n)\\
        \lequiv &
        \frac12\pbra{\Pr\sbra{\PrivK_{\mathcal A,\Pi}^{\eav}(n,0)=1}+\Pr\sbra{\PrivK_{\mathcal A,\Pi}^{\eav}(n,1)=1}}
        \leq\frac12+\negl_{\mathcal A}(n)\\
        \lequiv &
        \frac12\pbra{1-\Pr\sbra{\out_{\mathcal A}\pbra{\PrivK_{\mathcal A,\Pi}^\eav(n,0)}=1}
        +\Pr\sbra{\out_{\mathcal A}\pbra{\PrivK_{\mathcal A,\Pi}^\eav(n,1)}=1}}
        \leq\frac12+\negl_{\mathcal A}(n)\\
        \lequiv &
        \Pr\sbra{\out_{\mathcal A}\pbra{\PrivK_{\mathcal A,\Pi}^\eav(n,1)}=1}
        -\Pr\sbra{\out_{\mathcal A}\pbra{\PrivK_{\mathcal A,\Pi}^\eav(n,0)}=1}
        \leq 2\negl_{\mathcal A}(n).
    \end{align*}
    Let $\overline{\mathcal A}$ be the \ppt adversary identical with $\mathcal A$ except the output is flipped.
    Thus
    \begin{align*}
        &\Pr\sbra{\PrivK_{\overline{\mathcal A},\Pi}^{\eav}(n)=1}\leq\frac12+\negl_{\overline{\mathcal A}}(n)\\
        \lequiv &
        \Pr\sbra{\out_{\overline{\mathcal A}}\pbra{\PrivK_{\overline{\mathcal A},\Pi}^\eav(n,1)}=1}
        -\Pr\sbra{\out_{\overline{\mathcal A}}\pbra{\PrivK_{\overline{\mathcal A},\Pi}^\eav(n,0)}=1}
        \leq 2\negl_{\overline{\mathcal A}}(n)\\
        \lequiv &
        \Pr\sbra{\out_{\mathcal A}\pbra{\PrivK_{\mathcal A,\Pi}^\eav(n,1)}=0}
        -\Pr\sbra{\out_{\mathcal A}\pbra{\PrivK_{\mathcal A,\Pi}^\eav(n,0)}=0}
        \leq 2\negl_{\overline{\mathcal A}}(n)\\
        \lequiv &
        \Pr\sbra{\out_{\mathcal A}\pbra{\PrivK_{\mathcal A,\Pi}^\eav(n,0)}=1}
        -\Pr\sbra{\out_{\mathcal A}\pbra{\PrivK_{\mathcal A,\Pi}^\eav(n,1)}=1}
        \leq 2\negl_{\overline{\mathcal A}}(n).
    \end{align*}
    Therefore
    \begin{align*}
        &\Big|
        \Pr\sbra{\out_{\mathcal A}\pbra{\PrivK_{\mathcal A,\Pi}^\eav(n,0)}=1}
        -\Pr\sbra{\out_{\mathcal A}\pbra{\PrivK_{\mathcal A,\Pi}^\eav(n,1)}=1}
        \Big|\leq\negl(n)\\
        \lequiv & 
        \max\cbra{\Pr\sbra{\PrivK_{\mathcal A,\Pi}^{\eav}(n)=1},
        \Pr\sbra{\PrivK_{\overline{\mathcal A},\Pi}^{\eav}(n)=1}}
        \leq\frac12+\frac12\negl(n),\\
    \end{align*}
    which indicates the equivalence of Definition $3.8$ and $3.9$.
\end{exercise}

\begin{exercise}{3.6}
Let $\ell'$ be the expansion factor of $G'$.
\begin{enumerate}
\item[(a)] $G'(s):=G(s_1\cdots s_{\lfloor n/2\rfloor})$, where $s=s_1\cdots s_n$, is not necessarily a PRG.
\begin{counterexample}
    Assume the expansion factor of $G$ is $\ell(n)=2n+1$, then if $|s|=2k+1$, $|G'(s)|=2k+1$ as well, 
    which violates the requirement $\ell'(n)>n$.

    But if $\ell'(n)>n$, $G'$ is indeed a PRG. See Theorem \ref{ap3} in Appendix.
\end{counterexample}
\item[(b)] $G'(s):=G(0^{|s|}\|s)$ is not necessarily a PRG.
\begin{counterexample}
    Assume $G''$ is a PRG with expansion factor $\ell''(n)>2n$ and $G$ is defined as
    $$
        G(s):=G''(s_1\cdots s_{\lfloor n/2\rfloor})\|G''(s_{\lfloor n/2\rfloor+1}\cdots s_n),
    $$
    where $s=s_1\cdots s_n$. 
    Then $\ell(n)=\ell''(\lfloor n/2\rfloor)+\ell''(n-\lfloor n/2\rfloor)>2n$ and $G$ is a PRG by Theorem \ref{ap1} in Appendix.

    Design a \ppt algorithm $\mathcal D$ which outputs $1$ if and only if the first half is 
    $G''(0^{|s|})$. Therefore, assume $|s|=n$, we have a significant separation
    $$
        \Big|\Pr\sbra{\mathcal D(G(s))=1}-\Pr\sbra{\mathcal D(r)=1}\Big|=1-2^{-\ell''(n)}.
    $$
\end{counterexample}
\item[(c)] $G'(s):=G(s)\|G(s+1)$ is not necessarily a PRG.
\begin{proof}
    Assume $G''$ is a PRG with expansion factor $\ell''(n)>2n$ and $G$ is defined as
    $$
        G(s):=G''(s_1\cdots s_{\lfloor n/2\rfloor})\|G''(s_{\lfloor n/2\rfloor+1}\cdots s_n),
    $$
    where $s=s_1\cdots s_n$. 
    Then $\ell(n)=\ell''(\lfloor n/2\rfloor)+\ell''(n-\lfloor n/2\rfloor)>2n$ and $G$ is a PRG by Theorem \ref{ap1} in Appendix.
    Let $s'=s+1$, hence
    $$
    G'(s)=G''(s_1\cdots s_{\lfloor n/2\rfloor})\|G''(s_{\lfloor n/2\rfloor+1}\cdots s_n)
    \|G''(s'_1\cdots s_{\lfloor n/2\rfloor}')\|G''(s'_{\lfloor n/2\rfloor+1}\cdots s'_n).
    $$

    Design a \ppt algorithm $\mathcal D$ which outputs $1$ if and only if the first part equals the third part.
    Therefore, assume $|s|=n$, we have a significant separation
    \begin{align*}
        \Pr\sbra{\mathcal D(G(s))=1}-\Pr\sbra{\mathcal D(r)=1}\geq1-2^{-\lceil n/2\rceil}-2^{-\ell''(\lfloor n/2\rfloor)}.
    \end{align*}
\end{proof}
\end{enumerate}
\end{exercise}

\begin{exercise}{3.11}
    Assume $F$ is a length-preserving PRF.
    Let $\Pi=(\Gen,\Enc,\Dec)$ be a private-key encryption scheme, where
    \begin{itemize}
        \item $\Gen$ generates a random key from $\bin^n$ given input $1^n$.
        \item $\Enc$ is a deterministic algorithm, where if $k,m_i\in\bin^{n},\forall i\in[t]$,
            $$
            \Enc_k(m_1,m_2,\cdots,m_t)=(F_k(1)\oplus m_1,\cdots,F_k(i)\oplus m_i,\cdots,F_k(t)\oplus m_t).
            $$
        \item $\Dec$ is a deterministic algorithm, where if $k,c_i\in\bin^{n},\forall i\in[t]$,
            $$
            \Dec_k(c_1,c_2,\cdots,c_t)=(F_k(1)\oplus c_1,\cdots,F_k(i)\oplus m_i,\cdots,F_k(t)\oplus m_t).
            $$
    \end{itemize}
    Then $\Pi$ is an indistinguishable multiple encryption in the presence of an eavesdropper by Lemma \ref{ap2} in Appendix.

    On the other hand, given access to $\Enc_k$, the adversary may ask $\Enc_k(0,0,\cdots,0)$ which gives
    $\bm c^*=(F_k(1),\cdots,F_k(t))$. Then it can easily distinguish $\bm c_0=\bm c^*\oplus(m_{0,1},\cdots m_{0,t})$ 
    and $\bm c_1=\bm c^*\oplus(m_{1,1},\cdots,m_{1,t})$, which indicates $\Pi$ is not CPA-secure.
\end{exercise}

\begin{exercise}{3.17}
    Without loss of generality, assume $F$ is a length-preserving PRP.
    Define $F'$ as
    $$
        F'_k(r)=\begin{cases}
            0^n & r=k\\
            F_k(k) & r=F^{-1}_k(0^n)\\
            F_k(r) & \text{otherwise}.
        \end{cases}
    $$

    Firstly, $F'$ is a permutation, since
    $$
        F'^{-1}_k(r)=\begin{cases}
            k & r=0^n\\
            F^{-1}_k(0^n) & r=F_k(k)\\
            F^{-1}_k(r) & \text{otherwise}.
        \end{cases}
    $$

    Secondly, to show that $F'$ is not a SPRP, construct a \ppt algorithm $\mathcal D_1$ as follows:
    \begin{enumerate}
        \item $\mathcal D_1$ is given $1^n$ and oracles $\mathcal O,\mathcal O^{-1}:\bin^n\to\bin^n$.
        \item Query $k:=\mathcal O^{-1}(0^n)$. Then uniformly select $m\in\bin^n$ and query $c:=\mathcal O(m)$.
        \item Output $1$ if $c=F_k(m)$, and $0$ otherwise.
    \end{enumerate}
    Therefore, if $\mathcal O$ is $F'$, we have
    \begin{align*}
        \Pr\sbra{\mathcal D_1^{F'_k(\cdot),F'^{-1}_k(\cdot)}(1^n)=1}\geq 1-2^{n-1}.
    \end{align*}
    If $\mathcal O$ is FRP $f$, we have
    \begin{align*}
        \Pr\sbra{\mathcal D_1^{f(\cdot),f^{-1}(\cdot)}(1^n)=1}=
        \sum_{k,m}2^{-2n}\Pr\sbra{f(m)=F_k(m)\middle|f^{-1}(0^n)=k}\leq \frac{1}{2^n-1}.
    \end{align*}

    Thirdly, we shall prove $F'$ is pseudorandom.
    For any \ppt algorithm $\mathcal D$, let $\mathcal D_R$ be algorithm $\mathcal D$ when given random tape $R$;
    and $q(n)$ be the upper bound of potential queries from $\mathcal D$.
    Then
    \begin{align*}
        &\abs{\Pr\sbra{\mathcal D^{F'_k(\cdot)}(1^n)=1}-\Pr\sbra{\mathcal D^{F_k(\cdot)}(1^n)=1}}\\
        \leq&\sum_R
        \Pr\sbra{R}\sum_v
        2^{-n}
        \abs{\Pr\sbra{\mathcal D_R^{F'_v(\cdot)}(1^n)=1}-\Pr\sbra{\mathcal D_R^{F_v(\cdot)}(1^n)=1}}
    \end{align*}
    For any fixed $R$ and $v$, $\mathcal D_R^{F_v(\cdot)},\mathcal D_R^{F'_v(\cdot)}$ are deterministic algorithms.
    Therefore, let
    $$
    S_R=\cbra{v\middle|\mathcal D_R^{F_v(\cdot)}(1^n)\text{ does not query $v$ and $F_v^{-1}(0^n)$}}.
    $$
    When $v\in S_R$, $\mathcal D_R^{F'_v(\cdot)}(1^n)$ does not query $v$ and $F_v^{-1}(0^n)$ as well;
    and $\mathcal D_R^{F'_v(\cdot)}(1^n)=\mathcal D_R^{F_v(\cdot)}(1^n)$. Hence
    \begin{align*}
        &\abs{\Pr\sbra{\mathcal D^{F'_k(\cdot)}(1^n)=1}-\Pr\sbra{\mathcal D^{F_k(\cdot)}(1^n)=1}}\\
        \leq&\sum_R
        \Pr\sbra{R}\sum_{v\notin S_R}
        2^{-n}
        \abs{\Pr\sbra{\mathcal D_R^{F'_v(\cdot)}(1^n)=1}-\Pr\sbra{\mathcal D_R^{F_v(\cdot)}(1^n)=1}}\\
        \leq&\sum_R\Pr\sbra{R}\times 2^{-n}\times\pbra{2^n-\abs{S_R}}.
    \end{align*}

    Now for any $\mathcal D_R$, develop a new \ppt algorithm $\mathcal B$ as follows:
    \begin{enumerate}
        \item $\mathcal B$ is given $1^n$ and oracle $\mathcal O:\bin^n\to\bin^n$.
        \item Run $\mathcal D_R(1^n)$. Whenever $\mathcal D_R$ queries $r$,
            \begin{itemize}
                \item if $\mathcal O(r)=0^n$, output $1$ and terminate;
                \item otherwise, uniformly select $m\in\bin^n$ and query $c:=\mathcal O(m)$.
                    \begin{itemize}
                        \item If $c=F_r(m)$, output $1$ and terminate;
                        \item otherwise, return $\mathcal O(r)$ to $\mathcal D_R$.
                    \end{itemize}
            \end{itemize}
        \item When $\mathcal D_R$ gives $b\in\bin$, always output $0$.
    \end{enumerate}
    When $\mathcal O$ is PRP $F$,
    $$
    \Pr\sbra{\mathcal B^{F_k(\cdot)}(1^n)=1}=\Pr\sbra{k\notin S_R}
    =2^{-n}\times\pbra{2^n-\abs{S_R}}.
    $$
    When $\mathcal O$ is FRP $f$, applying union bound, we have
    $$
    \Pr\sbra{\mathcal B^{f(\cdot)}(1^n)=1}\leq q(n)\times\pbra{2^{-n}+2^{-n}}=q(n)2^{-n+1}.
    $$
    On the other hand, since $F$ is PRP, there exists a negligible function $\negl$ such that
    \begin{align*}
        2^{-n}\times\pbra{2^n-\abs{S_R}}\leq
        q(n)2^{-n+1}+\Pr\sbra{\mathcal B^{F_k(\cdot)}(1^n)=1}-\Pr\sbra{\mathcal B^{f(\cdot)}(1^n)=1}
        \leq q(n)2^{-n+1}+\negl(n).
    \end{align*}

    Thus,
    \begin{align*}
        &\abs{\Pr\sbra{\mathcal D^{F'_k(\cdot)}(1^n)=1}-\Pr\sbra{\mathcal D^{f(\cdot)}(1^n)=1}}\\
        \leq
        &\abs{\Pr\sbra{\mathcal D^{F'_k(\cdot)}(1^n)=1}-\Pr\sbra{\mathcal D^{F_k(\cdot)}(1^n)=1}}
        +\abs{\Pr\sbra{\mathcal D^{F_k(\cdot)}(1^n)=1}-\Pr\sbra{\mathcal D^{f(\cdot)}(1^n)=1}}\\
        \leq&\pbra{\sum_R\Pr\sbra{R}\times 2^{-n}\times\pbra{2^n-\abs{S_R}}}+\negl(n)\\
        \leq&\pbra{\sum_R\Pr\sbra{R}\times\Big(q(n)2^{-n+1}+\negl(n)\Big)}+\negl(n)\\
        \leq&q(n)2^{-n+1}+2\negl(n),
    \end{align*}
    which proves $F'$ is PRP.
\end{exercise}

\noindent\textbf{Remark:} $F'_k(r):=F_k(r)\oplus F_k(k)$ is not a PRP because $F''_k(r)$ is PRP, where
$$
    F''_k(r)=\begin{cases}
        F_k(0^n)\oplus 1 & r=k\\
        F_k(k) & r=F_k^{-1}\pbra{F_k(0^n)\oplus 1}\\
        F_k(r) & \text{otherwise}.
    \end{cases}
$$

\begin{exercise}{3.18}
For $k,c\in\bin^n$, the decryption $\Dec_k(c)$ is defined as the second half of $F_k^{-1}(c)$.
Now we shall prove the CPA-security for message of length $n/2$.
\begin{proof}
    Let $\Pi$ be the fixed-length encryption scheme proposed in the exercise and $\widetilde\Pi$
    be the scheme same as $\Pi$ except that FRP $f$ is used in place of $F_k$.
    For any \ppt adversary $\mathcal A$ of $\Pi$, develop a \ppt algorithm $\mathcal D$ as follows:
    \begin{enumerate}
        \item $\mathcal D$ is given input $1^n$ and access to oracle $\mathcal O:\bin^n\to\bin^n$.
        \item Run $\mathcal A(1^n)$. Whenever $\mathcal A$ queries the encryption of $m\in\bin^{n/2}$, 
            choose a uniform $r\in\bin^{n/2}$ and return $\mathcal O(r\|m)$ to $\mathcal A$.
        \item When $\mathcal A$ outputs $m_0,m_1\in\bin^{n/2}$, choose a uniform bit $b\in\bin$ and
            uniform $r^*\in\bin^{n/2}$, then return $\mathcal O(r^*\|m_b)$ to $\mathcal A$.
        \item Continue answering the encryption query from $\mathcal A$ until $\mathcal A$ gives a bit $b'$.
            Output $1$ if $b'=b$, and $0$ otherwise.
    \end{enumerate}
    Assume $\mathcal A$ makes $q(n)$ queries.
    Let $\Used$ denote $r^*$ is chosen in some encryption query.
    \begin{itemize}
        \item If $\mathcal O$ is a FRP, we have
            \begin{align*}
                \Pr\sbra{\mathcal D^{f(\cdot)}(1^n)=1}&=\Pr\sbra{\PrivK_{\mathcal A,\widetilde\Pi}^\cpa(n)=1}\\
                &=\Pr\sbra{\PrivK_{\mathcal A,\widetilde\Pi}^\cpa(n)=1,\Used}
                +\Pr\sbra{\overline\Used}\Pr\sbra{\PrivK_{\mathcal A,\widetilde\Pi}^\cpa(n)=1\middle|\overline\Used}\\
                &\leq\Pr\sbra{\Used}+\Pr\sbra{\PrivK_{\mathcal A,\widetilde\Pi}^\cpa(n)=1\middle|\overline\Used}\\
                &\leq q(n)2^{-n/2}+\frac12.
            \end{align*}
        \item If $\mathcal O$ is a PRP, we have
            $$
            \Pr\sbra{\mathcal D^{F_k(\cdot)}(1^n)=1}=\Pr\sbra{\PrivK_{\mathcal A,\Pi}^\cpa(n)=1}.
            $$
    \end{itemize}
    Combining the definition of PRP, there exists a negligible function $\negl$ such that
    \begin{align*}
        \Pr\sbra{\PrivK_{\mathcal A,\Pi}^\cpa(n)=1}&\leq\frac12+q(n)2^{-n/2}+\Pr\sbra{\mathcal D^{F_k(\cdot)}(1^n)=1}
        -\Pr\sbra{\mathcal D^{f(\cdot)}(1^n)=1}\\
        &\leq\frac12+q(n)2^{-n/2}+\negl(n),
    \end{align*}
    which indicates the CPA-security of $\Pi$.
\end{proof}
The CCA-secure part is equivalent to Exercise $4.25$.
\end{exercise}

\begin{exercise}{3.29}
\begin{construction}
    Construct $\Pi=(\Enc,\Dec)$, where
    \begin{itemize}
        \item $\Enc_{k_1,k_2}(m)=(\Enc_1(k_1,r),\Enc_2(k_2,m\oplus r))$. Here $r$ is uniformly picked from $\bin^{|m|}$.
        \item $\Dec_{k_1,k_2}(c_1,c_2)=\Dec_1(k_1,c_1)\oplus\Dec_2(k_2,c_2)$.
    \end{itemize}
\end{construction}
\begin{proof}
    Without loss of generality, assume $\Pi_2$ is CPA-secure. For any \ppt adversary $\mathcal A$ of $\Pi$, develop
    a \ppt adversary $\mathcal A_2$ of $\Pi_2$ as follows:
    \begin{enumerate}
        \item $\mathcal A_2$ is given $1^n$ and oracle access to $\Enc_2(k_2,\cdot)$, where $k_2$ is not explicit. 
            Then uniformly pick $k_1\in\{0,1\}^n$.
        \item Build oracle $\Enc_{k_1,k_2}(\cdot)$ as the construction shows. 
            Give oracle $\Enc_{k_1,k_2}(\cdot)$ to $\mathcal A$ and start to run $\mathcal A(1^n)$.
        \item When $\mathcal A$ outputs $m_0,m_1$, choose $r\in\bin^{|m_0|}$ uniformly.
            Let $m'_0=m_0\oplus r,m'_1=m_1\oplus r$. $\mathcal A_2$ gives $m'_0,m'_1$ to the CPA experiment, and obtain
            challenge text $c$.
        \item Return $(\Enc_1(k_1,r),c)$ and oracle access $\Enc_{k_1,k_2}(\cdot)$ to $\mathcal A$ until it
            answers $b\in\{0,1\}$. Then $\mathcal A_2$ outputs $b$ as well.
    \end{enumerate}
    Therefore, the CPA experiment of $\Pi_2$ actually simulates the CPA experiment of $\Pi$.
    Combining the CPA-security of $\Pi_2$, there exists a negligible function $\negl$ such that
    $$
        \Pr\sbra{\PrivK_{\mathcal A,\Pi}^\cpa(n)=1}=\Pr\sbra{\PrivK_{\mathcal A_2,\Pi_2}^\cpa(n)=1}\leq\frac12+\negl(n),
    $$
    which proves $\Pi$ is CPA-secure.
\end{proof}
\end{exercise}

\newpage
\section*{Appendix}
\begin{theorem}\label{ap1}
    Assume $G_1,G_2$ are two PRGs, then $G$ is also a PRG, where
    $$
        G(s):=G_1(s_1\cdots s_{\lfloor n/2\rfloor})\|G_2(s_{\lfloor n/2\rfloor+1}\cdots s_n).
    $$
\end{theorem}
\begin{proof}
    For any \ppt algorithm $\mathcal D$, we have
    \begin{align*}
        \Pr\sbra{\mathcal D(G(s))=1}
        =&\Pr\sbra{\mathcal D(G_1(\hat s)\|G_2(\tilde s))=1}\\
        =&\sum_{u}\Pr\sbra{G_2(\tilde s)=u}
        \Pr\sbra{\mathcal D(G_1(\hat s)\|u)=1}\\
        =&\sum_{u}\Pr\sbra{G_2(\tilde s)=u}
        \Pr\sbra{\mathcal D_u(G_1(\hat s))=1}.
    \end{align*}
    Since $G_1$ is PRG, then for any \ppt algorithm $\mathcal D'$, there exists negligible function $\negl$ such that
    $$
    \Big|\Pr_{|s|=n}\sbra{\mathcal D'(G_1(s))=1}-\Pr\sbra{\mathcal D'(r)=1}\Big|\leq\negl(n).
    $$
    Therefore
    \begin{align*}
        \negl(\lfloor n/2\rfloor)&\geq
        \Big|\Pr\sbra{\mathcal D(G(s))=1}-\sum_{u}\Pr\sbra{G_2(\tilde s)=u}\Pr\sbra{\mathcal D_u(\hat r)=1}\Big|\\
        &=
        \Big|\Pr\sbra{\mathcal D(G(s))=1}-\Pr\sbra{\mathcal D(\hat r\|G_2(\tilde s))=1}\Big|\\
        &=
        \Big|\Pr\sbra{\mathcal D(G(s))=1}-\sum_{v}\Pr\sbra{\hat r=v}\Pr\sbra{\mathcal D(v\|G_2(\tilde s))=1}\Big|\\
        &=
        \Big|\Pr\sbra{\mathcal D(G(s))=1}-\sum_{v}\Pr\sbra{\hat r=v}\Pr\sbra{\mathcal D_v(G_2(\tilde s))=1}\Big|
    \end{align*}
    Since $G_2$ is PRG, then for any \ppt algorithm $\mathcal D'$, there exists negligible function $\negl'$ such that
    $$
    \Big|\Pr\sbra{\mathcal D'(G_2(s))=1}-\Pr\sbra{\mathcal D'(r)=1}\Big|\leq\negl'(n).
    $$
    Thus
    \begin{align*}
        \negl(\lfloor n/2\rfloor)+\negl'(\lceil n/2\rceil)&\geq
            \Big|\Pr\sbra{\mathcal D(G(s))=1}-\sum_{v}\Pr\sbra{\hat r=v}\Pr\sbra{\mathcal D_v(\tilde r)=1}\Big|\\
        &=
            \Big|\Pr\sbra{\mathcal D(G(s))=1}-\sum_{v}\Pr\sbra{\hat r=v}\Pr\sbra{\mathcal D(v\|\tilde r)=1}\Big|\\
        &=
            \Big|\Pr\sbra{\mathcal D(G(s))=1}-\Pr\sbra{\mathcal D(r)=1}\Big|,
    \end{align*}
    which indicates $G$ is a PRG. 
\end{proof}

\begin{lemma}\label{ap2}
    Assume $F$ is a length-preserving PRF.
    Let $\Pi=(\Gen,\Enc,\Dec)$ be a private-key encryption scheme, where
    \begin{itemize}
        \item $\Gen$ generates a random key from $\bin^n$ given input $1^n$.
        \item $\Enc$ is a deterministic algorithm, where if $k,m_i\in\bin^{n},\forall i\in[t]$,
            $$
            \Enc_k(m_1,m_2,\cdots,m_t)=(F_k(1)\oplus m_1,\cdots,F_k(i)\oplus m_i,\cdots,F_k(t)\oplus m_t).
            $$
        \item $\Dec$ is a deterministic algorithm, where if $k,c_i\in\bin^{n},\forall i\in[t]$,
            $$
            \Dec_k(c_1,c_2,\cdots,c_t)=(F_k(1)\oplus c_1,\cdots,F_k(i)\oplus m_i,\cdots,F_k(t)\oplus m_t).
            $$
    \end{itemize}
    Then $\Pi$ is an indistinguishable multiple encryption in the presence of an eavesdropper.
\end{lemma}
\begin{proof}
    Let $\widetilde\Pi$ be an encryption scheme same as $\Pi$ except that a FRF $f$ is used instead of $F_k$.
    For any \ppt adversary $\mathcal A$ of $\Pi$, develop a \ppt algorithm $\mathcal D$ as follows:
    \begin{enumerate}
        \item $\mathcal D$ is given input $1^n$ and access to oracle $\mathcal O:\bin^n\to\bin^n$.
        \item Run $\mathcal A(1^n)$ to obtain $\bm m_0=(m_{0,1},\cdots,m_{0,t})$ and $\bm m_1=(m_{1,1},\cdots,m_{1,t})$.
        \item Generate a uniform bit $b$ from $\bin$.
        \item Return $\bm c_b=(\mathcal O(1)\oplus m_{b,1},\cdots,\mathcal O(t)\oplus m_{b,t})$ to $\mathcal A$.
        \item Check the answer $b'$ of $\mathcal A$. Output $1$ if $b'=b$, and $0$ otherwise.
    \end{enumerate}
    Therefore, if $\mathcal O$ is a FRF $f$,     
    we have $\mathcal O(i)\sim\bin^n$. 
    Thus $\mathcal O(i)\oplus m\sim\bin^n$ holds for any $i,m\in\bin^n$, 
    indicating the distribution of $\bm c_0$ is identical with $\bm c_1$. 
    Therefore 
    $$
    \Pr\sbra{\mathcal D^{f(\cdot)}(1^n)=1}=\Pr\sbra{\PrivK_{\mathcal A,\widetilde\Pi}^\mult(n)=1}=\frac12.
    $$
    On the other hand, if $\mathcal O$ is a PRF $F_k$, we have
    $$
    \Pr\sbra{\mathcal D^{F_k(\cdot)}(1^n)=1}=\Pr\sbra{\PrivK_{\mathcal A,\Pi}^\mult(n)=1}.
    $$
    Combining the definition of PRF, there exists a negligible function $\negl$ such that
    \begin{align*}
        \Pr\sbra{\PrivK_{\mathcal A,\Pi}^\mult(n)=1}&=\Pr\sbra{\mathcal D^{F_k(\cdot)}(1^n)=1}
        -\Pr\sbra{\mathcal D^{f(\cdot)}(1^n)=1}+\frac12\leq\frac12+\negl(n),
    \end{align*}
    which gives the indistinguishability of $\Pi$.
\end{proof}

\begin{theorem}\label{ap3}
    Assume $G$ is a PRG with expansion factor $\ell$, 
    define $G'$ as $G'(s)=G(s_1\cdots s_{\lfloor n/2\rfloor})$, where $s=s_1\cdots s_n$.
    If the expansion factor $\ell'$ of $G'$ satisfies $\ell'(n)>n$, $G'$ is a PRG.
\end{theorem}
\begin{proof}
    Since $G$ is a PRG, there exists a negligible function $\negl$ such that
    for any \ppt algorithm $\mathcal D$, we have
    \begin{align*}
        &\abs{\Pr_{|s|=n}\sbra{\mathcal D(G'(s))=1}-\Pr_{|r|=\ell'(n)}\sbra{\mathcal D(r)=1}}\\
        =&
        \abs{\Pr_{s_1,\cdots,s_{\lfloor n/2\rfloor}}\sbra{\mathcal D(G(s_1\cdots s_{\lfloor n/2\rfloor}))=1}
        -\Pr_{|r|=\ell(\lfloor n/2\rfloor)}\sbra{\mathcal D(r)=1}}\\
        =&
        \abs{\Pr_{|s|=\lfloor n/2\rfloor}\sbra{\mathcal D(G(s))=1}
        -\Pr_{|r|=\ell(\lfloor n/2\rfloor)}\sbra{\mathcal D(r)=1}}\\
        \leq&\negl(\lfloor n/2\rfloor),
    \end{align*}
    which gives $G'$ is a PRG.
\end{proof}

\end{document}
