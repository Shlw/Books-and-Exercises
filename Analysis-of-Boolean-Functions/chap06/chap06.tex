% !TeX encoding = UTF-8
% !TeX program = XeLaTeX
% !TeX spellcheck = LaTeX

\documentclass[a4paper]{article}

\usepackage{amsmath,amsfonts,amssymb}
\usepackage{mathrsfs}
\usepackage{bm}
\usepackage{extarrows}
\usepackage{geometry}
\usepackage{ntheorem}
\usepackage{hyperref}
\usepackage[ruled]{algorithm2e}
\usepackage{caption,subcaption}

\geometry{left=2cm,right=2cm,top=2cm,bottom=2cm}

\def\UrlBreaks{\do\A\do\B\do\C\do\D\do\E\do\F\do\G\do\H\do\I\do\J\do\K\do\L\do\M\do\N\do\O\do\P\do\Q\do\R\do\S\do\T\do\U\do\V\do\W\do\X\do\Y\do\Z\do\[\do\\\do\]\do\^\do\_\do\`\do\a\do\b\do\c\do\d\do\e\do\f\do\g\do\h\do\i\do\j\do\k\do\l\do\m\do\n\do\o\do\p\do\q\do\r\do\s\do\t\do\u\do\v\do\w\do\x\do\y\do\z\do\0\do\1\do\2\do\3\do\4\do\5\do\6\do\7\do\8\do\9\do\.\do\@\do\\\do\/\do\!\do\_\do\|\do\;\do\>\do\]\do\)\do\,\do\?\do\'\do+\do\=\do\#}

\newtheorem{theorem}{Theorem}
\newtheorem{lemma}{Lemma}
\newtheorem{proposition}{Proposition}
\newtheorem{corollary}{Corollary}
\newtheorem{claim}{Claim}
\newtheorem{conjecture}{conjecture}
\newtheorem{definition}{Definition}
\newtheorem{construction}{Construction}
\newtheorem*{proof}{Proof}
\newtheorem*{answer}{Answer}
\newtheorem*{example}{Example}
\newtheorem*{counterexample}{Counterexample}

\newenvironment{exercise}[1]{
	\par
	\noindent\textbf{Exercise #1.}\quad
}{
	\par
	\bigskip
}
\newenvironment{problem}[1]{
	\par
	\noindent\textbf{Problem #1.}\quad
}{
	\par
	\bigskip
}

\DeclareMathAccent{\widehat}{\mathord}{largesymbols}{"62}
\DeclareMathOperator*{\argmax}{\arg\,\max}
\DeclareMathOperator*{\argmin}{\arg\,\min}
\DeclareMathOperator*{\E}{\mathbb E}
\DeclareMathOperator{\Var}{\mathrm{Var}}
\DeclareMathOperator{\tr}{\mathrm{tr}}
\DeclareMathOperator{\poly}{\mathrm{poly}}
\DeclareMathOperator{\sd}{\mathop{d}}
\newcommand{\eps}{\varepsilon}
\newcommand{\abs}[1]{{\left| #1 \right|}}
\newcommand{\vabs}[1]{{\left\| #1 \right\|}}
\newcommand{\hvabs}[1]{{\hat{\|} #1 \hat{\|}}}
\newcommand{\abra}[1]{{\left\langle #1 \right\rangle}}
\newcommand{\pbra}[1]{{\left( #1 \right)}}
\newcommand{\cbra}[1]{{\left\{ #1 \right\}}}
\newcommand{\sbra}[1]{{\left[ #1 \right]}}
\newcommand{\floorbra}[1]{{\left\lfloor #1 \right\rfloor}}
\newcommand{\ceilbra}[1]{{\left\lceil #1 \right\rceil}}
\newcommand{\bin}{{\{0,1\}}}
\newcommand{\pmbin}{{\{-1,1\}}}
\newcommand{\sgn}{\text{sgn}}
\newcommand{\Fbb}{\mathbb{F}}
\newcommand{\Nbb}{\mathbb{N}}
\newcommand{\Rbb}{\mathbb{R}}
\newcommand{\Zbb}{\mathbb{Z}}
\newcommand{\Acal}{\mathcal{A}}
\newcommand{\Bcal}{\mathcal{B}}
\newcommand{\Ccal}{\mathcal{C}}
\newcommand{\Fcal}{\mathcal{F}}
\newcommand{\Gcal}{\mathcal{G}}
\newcommand{\Ncal}{\mathcal{N}}
\newcommand{\Scal}{\mathcal{S}}

\newcommand{\Inf}{\mathtt{Inf}}
\newcommand{\MaxInf}{\mathtt{MaxInf}}
\newcommand{\Dtt}{\mathtt{D}}
\newcommand{\Itt}{\mathtt{I}}
\newcommand{\Ltt}{\mathtt{L}}
\newcommand{\Wtt}{\mathtt{W}}
\newcommand{\Ttt}{\mathtt{T}}
\newcommand{\Stab}{\mathtt{Stab}}
\newcommand{\NS}{\mathtt{NS}}
\newcommand{\AC}{\textsc{AC}}

\bibliographystyle{plainnat}

\title{Exercise Set --- Chapter $6$}
\date{}

\begin{document}

\maketitle

\begin{exercise}{6.4}
    Let $\gamma\in\Fbb_2^n\setminus\cbra{\bm 0}$. Then
    $$
    \E_{y\sim\phi^{*d}}\sbra{\chi_\gamma(y)}
    =\E_{y_1,\ldots,y_d\sim\phi}\sbra{\chi_\gamma(y_1+\cdots+y_d)}
    =\E_{y_1,\ldots,y_d\sim\phi}\sbra{\chi_\gamma(y_1)\cdots\chi_\gamma(y_d)}
    =\E_{y_1\sim\phi}\sbra{\chi_\gamma(y_1)}^d
    \leq\eps^d.
    $$
\end{exercise}

\begin{exercise}{6.5}
    \begin{itemize}
        \item[(a)] For any $S\neq\emptyset$, we have
            $$
            \abs{\hat f(S)}\leq\sqrt{|S|\hat f(S)^2}\leq\sqrt{\Itt[f]}\leq\sqrt\eps.
            $$
    \end{itemize}
\end{exercise}

\begin{exercise}{6.6}
    Since $\E[f_{J|z}]=\sum_{S\cap J=\emptyset}\hat f(J\cup S)z^S$, we have
    $$
    \Var_z[\E[f_{J|z}]]=\Var_z\sbra{\sum_{S\cap J=\emptyset}\hat f(J\cup S)z^S}=\sum_{S\cap J=\emptyset,S\neq\emptyset}\hat f(J\cup S)^2.
    $$
\end{exercise}

\begin{exercise}{6.12}
    Let $g:\Fbb_2^n\to\Fbb_2$ be the $\Fbb_2$-polynomial of $f$, and $d=\deg_{\Fbb_2}(f)$.
    \begin{itemize}
        \item[(a)] Since restricting input bits does not increase sparsity, we assume $g$ has one maximal degree term and $n=d$.
            This means $g$ has odd number (denote by $t$) of inputs with value $1$. Then for any $S$,
            $$
            \hat f(S)=\E\sbra{f(x)x^S}=2^{-d}\sbra{\underbrace{\#\cbra{x:f(x)=1,x^S=1}}_a-\underbrace{\#\cbra{x:f(x)=1,x^S=-1}}_{t-a}}\neq0,
            $$
            which means $2^d\leq\text{sparsity}(\hat f)$.
        \item[(b)] Since restricting input bits does not decrease granularity, we assume $g$ has one maximal degree term and $n=d$ as well. Assume towards contradiction $\deg_{\Fbb_2}(f)=d>k$. Using similar analysis, we have
            $$
            \hat f(S)=\E\sbra{f(x)x^S}=2^{-d}\sbra{\underbrace{\#\cbra{x:f(x)=1,x^S=1}}_a-\underbrace{\#\cbra{x:f(x)=1,x^S=-1}}_{t-a}}=2^{-d}\underbrace{(2a-t)}_\text{odd}.
            $$
            Thus $\hat f$ should be $2^{-d}\cdot(\text{some odd number})$-granular. 
    \end{itemize}
\end{exercise}

\begin{exercise}{6.18}
    $$
    \widehat{2^{n/2}\hat f}(x)=\E_S\sbra{2^{n/2}\hat f(S)(-1)^{x\cdot S}}=2^{n/2}\E_{S,y}\sbra{f(y)(-1)^{y\cdot S}(-1)^{x\cdot S}}
    =2^{n/2}\E_y\sbra{f(y)\E_S\sbra{(-1)^{(x+y)\cdot S}}}
    =\pm2^{-n/2}.
    $$
\end{exercise}

\begin{exercise}{6.19}
    Since $g(x)=(-1)^{p(x)}$, for any $S$ we have
    $$
    \hat g(S)=\E\sbra{(-1)^{p(x)}(-1)^{x\cdot S}}=\E\sbra{(-1)^{\ell_0(x)+S\cdot x+\sum\ell_j(x)\ell'_j(x)}}.
    $$
    Since $\ell_1,\ell_1',\ldots,\ell_k,\ell_k'$ are linearly independent, we know $k\leq n/2$. 
    Applying a linear transformation to input bits, we may rewrite $y_j=\ell_j(x),y_j'=\ell_j'(x)$ and 
    $\ell_0(x)+S\cdot x=:y\cdot T$. Thus
    $$
    \hat g(S)=\E\sbra{(-1)^{y\cdot T+\sum y_jy'_j}}=\pm2^{-k}.
    $$
\end{exercise}

\begin{exercise}{6.27}
    Denote the new matrix as $H''$.
    Since $n=2^\ell$, we can view $\Fbb_n$ as $\Fbb_2[x]/p(x)$ where $p(x)$ is a degree-$\ell$ irreducible polynomial.
    Hence for any $\cbra{\beta_j}_j\subset\Fbb_n$, we have $\pbra{\sum_j\beta_j}^2=\sum_j\beta_j$.
    Now assume without loss of generality there exists $\gamma\in\bin^n,|\gamma|\leq k$ such that $H''\gamma=0$.
    Then we have
    \begin{align*}
        &\sum_j\gamma_j=0\\
        &\sum_j\alpha^{2i-1}\gamma_j=0,\quad\forall i,
    \end{align*}
    which also implies 
    $$
    \sum_j\alpha^{2^t(2i-1)}\gamma_j=
    \sum_j\alpha^{2^t(2i-1)}\gamma_j^{2^t}=
    \pbra{\sum_j\alpha^{2i-1}\gamma_j}^t=0,\quad\forall t,i.
    $$
    Hence $H\gamma=0$. A contradiction.
\end{exercise}

\begin{exercise}{6.28}
    \begin{itemize}
        \item[(a)] Since $\phi_A$ is $k$-wise independent, for any $S,|S|\leq k$ we know
            $$
            0=\hat{\phi_A}(S)=\E\sbra{\phi_A(x)x^S}=|A|^{-1}\E\sbra{\phi_A(x)x^S\mid x\in A}.
            $$
            Now for any distinct $S,T\in\Fcal$, we compute
            $$
            \abra{\chi_S^A,\chi_T^A}=\sum_{a\in A}a^{S\oplus T}=|A|\cdot \hat{\phi_A}(S\oplus T)=0.
            $$
        \item[(b)] For even number $k$, we construct $\Fcal=\dbinom{[n]}{\leq k/2}$. 
            For odd number $k$, we construct $\Fcal=\dbinom{[n]}{\leq (k-1)/2}\cup\pbra{\cbra{n}\times\dbinom{[n-1]}{(k-1)/2}}$.
    \end{itemize}
\end{exercise}

\begin{exercise}{6.34}
    \begin{itemize}
        \item[(a)]
            $$
            \abra{f_0,f_1}_{U^1}=\E_{x,y_1}\sbra{f_0(x)f_1(x+y_1)}=\E[f_0]\E[f_1].
            $$
        \item[(b)]
            \begin{align*}
            \abra{f_{00},f_{01},f_{10},f_{11}}_{U^2}
                &=\E_{x,y_1,y_2}\sbra{f_{00}(x)f_{01}(x+y_1)f_{10}(x+y_2)f_{11}(x+y_1+y_2)}\\
                &=\E_{x,y_1,y_2}\sbra{f_{00}(x)f_{01}(x+y_1)f_{10}(y_2)f_{11}(y_1+y_2)}\\
                &=\E_{y_1}\sbra{(f_{00}*f_{01})(y_1)(f_{10}*f_{11})(y_1)}\\
                &=\sum_\gamma\widehat{(f_{00}*f_{01})}(\gamma)\widehat{(f_{10}*f_{11})}(\gamma)\\
                &=\sum_\gamma\hat f_{00}(\gamma)\hat f_{01}(\gamma)\hat f_{10}(\gamma)\hat f_{11}(\gamma).
            \end{align*}
        \item[(c)]
            \begin{align*}
                \abra{(f_s)_s}_{U^k}
                &=\E_{x,y_1,\ldots,y_k}\sbra{\prod_sf_s(x+\sum_{i:s_i=1}y_i)}\\
                &=\E_{y_1,\ldots,y_{k-1}}\sbra{\E_{x,y_k}\sbra{\prod_sf_s(x+\sum_{i:s_i=1}y_i)}}\\
                &=\E_{y_1,\ldots,y_{k-1}}\sbra{\E_x\sbra{\prod_{s:s_k=0}f_s(x+\sum_{i:s_i=1}y_i)\E_{y_k}\sbra{\prod_{s:s_k=1}f_s(x+y_k+\sum_{i,i<k:s_i=1}y_i)}}}\\
                &=\E_{y_1,\ldots,y_{k-1}}\sbra{\E_x\sbra{\prod_{s:s_k=0}f_s(x+\sum_{i:s_i=1}y_i)}\E_{x'}\sbra{\prod_{s:s_k=1}f_s(x'+\sum_{i,i<k:s_i=1}y_i)}}.
            \end{align*}
        \item[(d)] 
            \begin{align*}
            \abra{f,\ldots,f}_{U^k}
                &=\E_{y_1,\ldots,y_{k-1}}\sbra{\E_x\sbra{\prod_{s:s_k=0}f(x+\sum_{i:s_i=1}y_i)}\E_{x'}\sbra{\prod_{s:s_k=1}f(x'+\sum_{i,i<k:s_i=1}y_i)}}\\
                &=\E_{y_1,\ldots,y_{k-1}}\sbra{\E_x\sbra{\prod_{s:s_k=0}f(x+\sum_{i:s_i=1}y_i)}\E_{x'}\sbra{\prod_{s:s_k=0}f(x'+\sum_{i:s_i=1}y_i)}}\\
                &=\E_{y_1,\ldots,y_{k-1}}\sbra{\E_x\sbra{\prod_{s:s_k=0}f(x+\sum_{i:s_i=1}y_i)}^2}\\
                &\geq0.
            \end{align*}
        \item[(e)] Since $\E[a_ib_i]\leq\sqrt{\E[a_i^2]\E[b_i^2]}$, we have
            \begin{align*}
            \abra{(f_s)_s}_{U^k}
                &=\E_{y_1,\ldots,y_{k-1}}\sbra{\E_x\sbra{\prod_{s:s_k=0}f_s(x+\sum_{i:s_i=1}y_i)}\E_{x'}\sbra{\prod_{s:s_k=1}f_s(x'+\sum_{i,i<k:s_i=1}y_i)}}\\
                &\leq\sqrt{\E_{y_1,\ldots,y_{k-1}}\sbra{\E_x\sbra{\prod_{s:s_k=0}f_s(x+\sum_{i:s_i=1}y_i)}^2}}
                \sqrt{\E_{y_1,\ldots,y_{k-1}}\sbra{\E_{x'}\sbra{\prod_{s:s_k=1}f_s(x'+\sum_{i,i<k:s_i=1}y_i)}^2}}.
            \end{align*}
            Then it suffices to show 
            \begin{align*}
                &\E_{y_1,\ldots,y_{k-1}}\sbra{\E_x\sbra{\prod_{s:s_k=0}f_s(x+\sum_{i:s_i=1}y_i)}^2}\\
                =&\E_{y_1,\ldots,y_{k-1}}\sbra{\E_x\sbra{\prod_{s:s_k=0}f_s(x+\sum_{i:s_i=1}y_i)}\E_z\sbra{\prod_{s:s_k=0}f_s(z+\sum_{i:s_i=1}y_i)}}\\
                =&\E_{y_1,\ldots,y_{k-1}}\sbra{\E_x\sbra{\prod_{s:s_k=0}f_s(x+\sum_{i:s_i=1}y_i)}\E_{y_k}\sbra{\prod_{s:s_k=0}f_s(x+y_k+\sum_{i:s_i=1}y_i)}}\\
                =&\abra{f_{0\cdots00},\ldots,f_{1\cdots10},f_{0\cdots00},\ldots,f_{1\cdots10}}_{U^k}.
            \end{align*}
        \item[(f)] We show $\abra{f_{00},f_{01},f_{10},f_{11}}_{U^2}\leq\vabs{f_{00}}_{U^2}\vabs{f_{01}}_{U^2}\vabs{f_{10}}_{U^2}\vabs{f_{11}}_{U^2}$ and the argument is similar for larger $k$.
            By (e), it suffices to prove $\abra{f_{00},f_{10},f_{00},f_{10}}_{U^2}\leq\vabs{f_{00}}_{U^2}^2\vabs{f_{10}}_{U^2}^2$.
            This follows from
            \begin{align*}
                \abra{f,g,f,g}_{U^2}
                &=\E_{x,y_1,y_2}\sbra{f(x)g(x+y_1)f(x+y_2)g(x+y_1+y_2)}\\
                &=\E_{x,y_1,y_2}\sbra{f(x)f(x+y_2)g(x+y_1)g(x+y_1+y_2)}\\
                &=\E_{y_1,y_2}\sbra{\E_x\sbra{f(x)f(x+y_2)}\E_{x'}\sbra{g(x')g(x'+y_2)}}\\
                &\leq\sqrt{\E_{y_1,y_2}\E_x\sbra{f(x)f(x+y_2)}^2}\sqrt{\E_{y_1,y_2}\E_{x'}\sbra{g(x')g(x'+y_2)}^2}\\
                &=\sqrt{\abra{f,f,f,f}_{U^2}}\sqrt{\abra{g,g,g,g}_{U^2}}\\
                &=\vabs{f}_{U^2}^2\vabs{g}_{U^2}^2.
            \end{align*}
        \item[(g)] Observe that
            \begin{align*}
                \vabs{f}_{U^k}^{2^k}=
                \abra{\underbrace{f,\ldots,f}_{2^k},\underbrace{1,\ldots,1}_{2^k}}_{U^{k+1}}
                \leq\vabs{f}_{U^{k+1}}^{2^k}\vabs{1}_{U^{k+1}}^{2^k}
                =\vabs{f}_{U^{k+1}}^{2^k}.
            \end{align*}
        \item[(h)] Since Gowers inner product is multilinear, we have
            \begin{align*}
                \vabs{f_0+f_1}_{U^k}^{2^k}
                &=\abra{f_0+f_1,\ldots,f_0+f_1}_{U^k}\\
                &=\sum_{s\in\bin^{2^k}}\abra{f_{s_1},\ldots,f_{s_k}}_{U^k}\\
                &\leq\sum_{s\in\bin^{2^k}}\vabs{f_{s_1}}_{U^k}\cdots\vabs{f_{s_k}}_{U^k}\\
                &=\sum_{i=0}^{2^k}\binom{2^k}i\vabs{f_0}_{U^k}^i\vabs{f_1}_{U^k}^{2^k-i}\\
                &=\pbra{\vabs{f_0}_{U^k}+\vabs{f_1}_{U^k}}^{2^k}.
            \end{align*}
        \item[(i)] Since $\vabs{f}_{U^2}\leq\vabs{f}_{U^k},\forall k\geq2$, we know $0=\vabs{f}_{U^2}=\hvabs{f}_4$.
    \end{itemize}
\end{exercise}

\end{document}
