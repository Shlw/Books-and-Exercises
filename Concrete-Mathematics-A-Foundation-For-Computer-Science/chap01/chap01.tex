% !TeX encoding = UTF-8
% !TeX program = XeLaTeX
% !TeX spellcheck = LaTeX

\documentclass[a4paper]{article}

\usepackage[UTF8]{ctex}
\usepackage{amsmath,amsfonts,amssymb}
\usepackage{mathrsfs}
\usepackage{bm}
\usepackage{extarrows}
\usepackage{geometry}
\usepackage{ntheorem}
\usepackage{hyperref}
\usepackage[ruled]{algorithm2e}
\usepackage{caption,subcaption}

\geometry{left=2cm,right=2cm,top=2cm,bottom=2cm}

\def\UrlBreaks{\do\A\do\B\do\C\do\D\do\E\do\F\do\G\do\H\do\I\do\J\do\K\do\L\do\M\do\N\do\O\do\P\do\Q\do\R\do\S\do\T\do\U\do\V\do\W\do\X\do\Y\do\Z\do\[\do\\\do\]\do\^\do\_\do\`\do\a\do\b\do\c\do\d\do\e\do\f\do\g\do\h\do\i\do\j\do\k\do\l\do\m\do\n\do\o\do\p\do\q\do\r\do\s\do\t\do\u\do\v\do\w\do\x\do\y\do\z\do\0\do\1\do\2\do\3\do\4\do\5\do\6\do\7\do\8\do\9\do\.\do\@\do\\\do\/\do\!\do\_\do\|\do\;\do\>\do\]\do\)\do\,\do\?\do\'\do+\do\=\do\#}

\newtheorem{theorem}{Theorem}
\newtheorem{lemma}{Lemma}
\newtheorem{proposition}{Proposition}
\newtheorem{corollary}{Corollary}
\newtheorem{claim}{Claim}
\newtheorem{conjecture}{conjecture}
\newtheorem{definition}{Definition}
\newtheorem{construction}{Construction}
\newtheorem*{proof}{Proof}
\newtheorem*{answer}{Answer}
\newtheorem*{refute}{Refute}
\newtheorem*{example}{Example}
\newtheorem*{counterexample}{Counterexample}

\newenvironment{exercise}[1]{
	\par
	\noindent\textbf{Exercise #1.}\quad
}{
	\par
	\bigskip
}


\DeclareMathAccent{\widehat}{\mathord}{largesymbols}{"62}
\newcommand{\abs}[1]{\left| #1 \right|}
\newcommand{\pbra}[1]{\left( #1 \right)}
\newcommand{\cbra}[1]{\left\{ #1 \right\}}
\newcommand{\sbra}[1]{\left[ #1 \right]}
\newcommand{\bin}{\{0,1\}}

\title{Exercise Set --- Chapter $1$}
\date{}

\begin{document}

\maketitle

\begin{exercise}{2}
    Let $f(n)$ be the answer to $n$ disks, then $f(1)=2$. For $n>1$, consider the largest disk.
    \begin{enumerate}
        \item Before $n$ can be moved, $1\sim (n-1)$ must be piled on B.
        \item Move $n$ to the middle.
        \item Before $n$ can be moved to B, $1\sim (n-1)$ must be moved to A.
        \item Move $n$ to B.
        \item Move $1\sim (n-1)$ to B.
    \end{enumerate}
    Thus $f(n)=f(n-1)+1+f(n-1)+1+f(n-1)=3^n-1$.
\end{exercise}

\begin{exercise}{7}
    $H(1)=J(2)-J(1)=0\neq 2$. The induction fails at the beginning.
\end{exercise}

\begin{exercise}{12}
    Consider the largest disks, we have
    $$
    A(m_1,\cdots,m_n)=2A(m_1,\cdots,m_{n-1})+m_n=\sum_{i=1}^nm_i2^{n-i}.
    $$
\end{exercise}

\begin{exercise}{17}
    Assume now we have $n(n+1)/2$ disks, and we use the following strategy:
    \begin{enumerate}
        \item Move the top $n(n-1)/2$ disks to second peg. (with cost $W_{n(n-1)/2}$)
        \item Move the bottom $n$ disks to third peg without using second one. (with cost $T_n$)
        \item Move the top $n(n-1)/2$ disks to third peg. (with cost $W_{n(n-1)/2}$)
    \end{enumerate}
    Thus the inequality holds.
\end{exercise}

\begin{exercise}{22}
    Let $G$ be a regular polygon with $2^n$ sides and label the sides with a de Bruijin cycle of length $2^n$.
    Now construct a base set $S$ by attaching a tiny margin to the sides labeled $1$ on $G$.
    Then rotate $S$ about the center for angle $2\pi\frac kn,k=0,1,\cdots,n-1$ and we get all $2^n$ possible subsets
    along the $2^n$ sides of $G$.

    A de Bruijin cycle of length $2^n$ is $s_0s_1\cdots s_{2^n-1}\in\bin^{2^n}$ such that for any $k=0,1,\cdots,2^n-1$,
    $s_ks_{k+1}\cdots s_{k+n}$ are distinct (here $+$ is computed modulo $2^n$).
    It can be constructed as follows (here we assume $n>1$):
    \begin{itemize}
        \item Define digraph $H=(V,E)$, where $V=\bin^{2^{n-1}}$ and
            $$
            E=\cbra{s_1\cdots s_{n-1}\xlongrightarrow{b} s_2\cdots s_{n-1}b\middle|b\in\bin}.
            $$
        \item Since any vertex in $H$ has in-degree $2$ and out-degree $2$, it has a Euler cycle $\mathcal C$.
        \item The label sequence of $\mathcal C$ is a desired de Bruijin cycle.
    \end{itemize}
\end{exercise}

\end{document}
