% !TeX encoding = UTF-8
% !TeX program = XeLaTeX
% !TeX spellcheck = LaTeX

\documentclass[a4paper]{article}

\usepackage{amsmath,amsfonts,amssymb}
\usepackage{mathrsfs}
\usepackage{bm}
\usepackage{extarrows}
\usepackage{geometry}
\usepackage{ntheorem}
\usepackage{hyperref}
\usepackage[ruled]{algorithm2e}
\usepackage{caption,subcaption}

\geometry{left=2cm,right=2cm,top=2cm,bottom=2cm}

\def\UrlBreaks{\do\A\do\B\do\C\do\D\do\E\do\F\do\G\do\H\do\I\do\J\do\K\do\L\do\M\do\N\do\O\do\P\do\Q\do\R\do\S\do\T\do\U\do\V\do\W\do\X\do\Y\do\Z\do\[\do\\\do\]\do\^\do\_\do\`\do\a\do\b\do\c\do\d\do\e\do\f\do\g\do\h\do\i\do\j\do\k\do\l\do\m\do\n\do\o\do\p\do\q\do\r\do\s\do\t\do\u\do\v\do\w\do\x\do\y\do\z\do\0\do\1\do\2\do\3\do\4\do\5\do\6\do\7\do\8\do\9\do\.\do\@\do\\\do\/\do\!\do\_\do\|\do\;\do\>\do\]\do\)\do\,\do\?\do\'\do+\do\=\do\#}

\newtheorem{theorem}{Theorem}
\newtheorem{lemma}{Lemma}
\newtheorem{proposition}{Proposition}
\newtheorem{corollary}{Corollary}
\newtheorem{claim}{Claim}
\newtheorem{conjecture}{conjecture}
\newtheorem{definition}{Definition}
\newtheorem{construction}{Construction}
\newtheorem*{proof}{Proof}
\newtheorem*{answer}{Answer}
\newtheorem*{example}{Example}
\newtheorem*{counterexample}{Counterexample}

\newenvironment{exercise}[1]{
	\par
	\noindent\textbf{Exercise #1.}\quad
}{
	\par
	\bigskip
}
\newenvironment{problem}[1]{
	\par
	\noindent\textbf{Problem #1.}\quad
}{
	\par
	\bigskip
}

\DeclareMathAccent{\widehat}{\mathord}{largesymbols}{"62}
\DeclareMathOperator*{\argmax}{\arg\,\max}
\DeclareMathOperator*{\argmin}{\arg\,\min}
\DeclareMathOperator{\E}{\mathbb E}
\newcommand{\abs}[1]{\left| #1 \right|}
\newcommand{\pbra}[1]{\left( #1 \right)}
\newcommand{\cbra}[1]{\left\{ #1 \right\}}
\newcommand{\sbra}[1]{\left[ #1 \right]}
\newcommand{\floorbra}[1]{\left\lfloor #1 \right\rfloor}
\newcommand{\ceilbra}[1]{\left\lceil #1 \right\rceil}
\newcommand{\bin}{\{0,1\}}
\newcommand{\ZPP}{\mathtt{ZPP}}
\newcommand{\RP}{\mathtt{RP}}
\newcommand{\coRP}{\mathtt{co}\text{-}\mathtt{RP}}
\newcommand{\per}{\text{per}}
\newcommand{\Nbb}{\mathbb{N}}
\newcommand{\Zbb}{\mathbb{Z}}
\newcommand{\Acal}{\mathcal{A}}
\newcommand{\Bcal}{\mathcal{B}}
\newcommand{\Ccal}{\mathcal{C}}
\newcommand{\Fcal}{\mathcal{F}}

\bibliographystyle{plainnat}

\title{Exercise Set --- Chapter $3$}
\date{}

\begin{document}

\maketitle

\begin{exercise}{1}
    $$
    R(k,k)>n-\binom nk2^{1-\binom k2}\geq n-\pbra{\frac{en}k}^k2^{1-\binom k2}.
    $$
    Let 
    $$
    n=\frac1{\sqrt[k-1]{2e}}\frac ke2^{k/2}
    $$
    then
    $$
    R(k,k)\geq\pbra{1-\frac1k}\frac1{\sqrt[k-1]{2e}}\frac ke2^{k/2}=(1-o(1))\frac ke2^{k/2}.
    $$
\end{exercise}

\begin{exercise}{2}
    As shown in Section 3.1,
    $$
    R(4,k)>n-\binom n4p^6-\binom nk(1-p)^{\binom k2}\geq n-n^4p^6-\frac{n^k}{k!}e^{-p\frac{k^2}4}.
    $$
    Let $p=\frac{4\ln n}k\delta$, then 
    $$
    R(4,k)>n-n^4\pbra{\frac{4\delta\ln n}k}^6-\frac{n^{k(1-\delta)}}{k!}.
    $$
    Let $n=\Theta\pbra{\pbra{\frac k{\ln k}}^2}$ and $\delta=O(1)$, we have $R(4,k)\geq\Omega\pbra{(k/\ln k)^2}$.
\end{exercise}

\begin{exercise}{3}
    Independently select every vertex with probability $\sqrt{\frac n{3m}}\leq 1$ to form set $A$. 
    Let $B$ be the set of edges induced by vertices in
    $A$. Then 
    $$
    \E\sbra{|A|-|B|}=np-mp^3=\frac{2n^{3/2}}{3\sqrt3\sqrt m}.
    $$
    By removing at most $1$ vertex in every edge in $B$, there exists an independent set of size at least
    $\frac{2n^{3/2}}{3\sqrt3\sqrt m}$.
\end{exercise}

\begin{exercise}{4}
    Let $G$ be a digraph of order $n$ and each vertex has outdegree at least $\delta=\log n-\frac1{10}\log\log n$.
    For any $v\in G$, define $N_G(v)=\cbra{u\in G\middle|v\to u\in E(G)}$.
    Independently for any vertex $v$ choose its color $c_v$ as black or white with equal possibility; thus
    \begin{align*}
    \E\sbra{\#\cbra{v\middle|\exists u,\pbra{c_u\neq c_v}\land\pbra{u\in N_G(v)}}}
        \geq n\pbra{\frac12}^\delta=\pbra{\log n}^{1/10}=o(\log n).
    \end{align*}
    Then fix such coloring and remove $o(\log n)$ vertices in 
    $\cbra{v\middle|\exists u,\pbra{c_u\neq c_v}\land\pbra{u\in N_G(v)}}$. Every vertex $v$ in the remaining graph $G'$ 
    has outdegree at least $1$ and at least one vertex in $N_{G'}(v)$ has different color.
    
    Choose an arbitrary vertex $v_1\in G'$ and go to another one of different color repeatedly, then we have
    $$
    v_1\to v_2\to\cdots\to v_k\to v_{k+1}=v_i,i\leq k.
    $$
    The simple directed cycle $v_i\to v_{i+1}\to\cdots\to v_k\to v_i$ has even length since $c_{v_i}\neq c_{v_k}$.
\end{exercise}

\end{document}
