% !TeX encoding = UTF-8
% !TeX program = XeLaTeX
% !TeX spellcheck = LaTeX

\documentclass[a4paper]{article}

\usepackage{amsmath,amsfonts,amssymb}
\usepackage{mathrsfs}
\usepackage{bm}
\usepackage{extarrows}
\usepackage{geometry}
\usepackage{ntheorem}
\usepackage{hyperref}
\usepackage[ruled]{algorithm2e}
\usepackage{caption,subcaption}

\geometry{left=2cm,right=2cm,top=2cm,bottom=2cm}

\def\UrlBreaks{\do\A\do\B\do\C\do\D\do\E\do\F\do\G\do\H\do\I\do\J\do\K\do\L\do\M\do\N\do\O\do\P\do\Q\do\R\do\S\do\T\do\U\do\V\do\W\do\X\do\Y\do\Z\do\[\do\\\do\]\do\^\do\_\do\`\do\a\do\b\do\c\do\d\do\e\do\f\do\g\do\h\do\i\do\j\do\k\do\l\do\m\do\n\do\o\do\p\do\q\do\r\do\s\do\t\do\u\do\v\do\w\do\x\do\y\do\z\do\0\do\1\do\2\do\3\do\4\do\5\do\6\do\7\do\8\do\9\do\.\do\@\do\\\do\/\do\!\do\_\do\|\do\;\do\>\do\]\do\)\do\,\do\?\do\'\do+\do\=\do\#}

\newtheorem{theorem}{Theorem}
\newtheorem{lemma}{Lemma}
\newtheorem{proposition}{Proposition}
\newtheorem{corollary}{Corollary}
\newtheorem{claim}{Claim}
\newtheorem{conjecture}{conjecture}
\newtheorem{definition}{Definition}
\newtheorem{construction}{Construction}
\newtheorem*{proof}{Proof}
\newtheorem*{answer}{Answer}
\newtheorem*{example}{Example}
\newtheorem*{counterexample}{Counterexample}

\newenvironment{exercise}[1]{
	\par
	\noindent\textbf{Exercise #1.}\quad
}{
	\par
	\bigskip
}
\newenvironment{problem}[1]{
	\par
	\noindent\textbf{Problem #1.}\quad
}{
	\par
	\bigskip
}

\DeclareMathAccent{\widehat}{\mathord}{largesymbols}{"62}
\DeclareMathOperator*{\argmax}{\arg\,\max}
\DeclareMathOperator*{\argmin}{\arg\,\min}
\DeclareMathOperator{\E}{\mathbb E}
\DeclareMathOperator{\Var}{\mathrm{Var}}
\newcommand{\abs}[1]{\left| #1 \right|}
\newcommand{\vabs}[1]{\left\| #1 \right\|}
\newcommand{\pbra}[1]{\left( #1 \right)}
\newcommand{\cbra}[1]{\left\{ #1 \right\}}
\newcommand{\sbra}[1]{\left[ #1 \right]}
\newcommand{\floorbra}[1]{\left\lfloor #1 \right\rfloor}
\newcommand{\ceilbra}[1]{\left\lceil #1 \right\rceil}
\newcommand{\bin}{\{0,1\}}
\newcommand{\ZPP}{\mathtt{ZPP}}
\newcommand{\RP}{\mathtt{RP}}
\newcommand{\coRP}{\mathtt{co}\text{-}\mathtt{RP}}
\newcommand{\per}{\text{per}}
\newcommand{\sgn}{\text{sgn}}
\newcommand{\Fbb}{\mathbb{F}}
\newcommand{\Nbb}{\mathbb{N}}
\newcommand{\Rbb}{\mathbb{R}}
\newcommand{\Zbb}{\mathbb{Z}}
\newcommand{\Acal}{\mathcal{A}}
\newcommand{\Bcal}{\mathcal{B}}
\newcommand{\Ccal}{\mathcal{C}}
\newcommand{\Fcal}{\mathcal{F}}

\bibliographystyle{plainnat}

\title{Exercise Set --- Chapter $5$}
\date{}

\begin{document}

\maketitle

\begin{exercise}{2}
    Fix an arbitrary $n>2\ell_0$, we show there exists a binary sequence $a\in\bin^n$ satisfying the requirement.
    Construct $a$ by independently pick $a_i\in\bin$ uniformly.
    Let $A_{i,k},k>\ell_0,i+2k-1\leq n$ be the event $u=(a_i,\cdots,a_{i+k-1})$ and $v=(a_{i+k},\cdots,a_{i+2k-1})$
    differ in less than $(1/2-\epsilon)k$ coordinates.
    Then
    $$
    \Pr\sbra{A_{i,k}}=\Pr\sbra{\frac1k\sum_{j=i}^{i+k-1}[a_j\neq a_{j+k}]-\frac12<-\epsilon}
    \leq e^{-2\epsilon^2k}.
    $$
    On the other hand, $A_{i,k}$ is (at most) correlated with $A_{i',t},i+k\in[i',i'+2t-1]$.
    By Local Lemma, it suffices to show there exists $x_{\ell_0+1},\cdots,x_{n/2}\in[0,1]$ such that
    $$
    \Pr\sbra{A_{i,k}}\leq C^k\leq x_k\prod_{\substack{i',t\\i+k\in[i',i'+2t-1]}}(1-x_t)
    $$
    holds for any $i,k$, where $C=e^{-2\epsilon^2}$ is a constant smaller than $1$.

    Let $x_t=C^{\delta_t},\delta_t\geq0$ for all $t=\ell_0+1,\cdots,n/2$.
    Then
    \begin{align*}
        &x_k\prod_{\substack{i',t\\i+k\in[i',i'+2t-1]}}(1-x_t)\\
        \geq&C^{\delta_k}\prod_{t=\ell_0+1}^{n/2}\pbra{1-C^{\delta_t}}^{2t}\\
        \geq&C^{\delta_k}\prod_{t=\ell_0+1}^{n/2}\exp\cbra{-4tC^{\delta_t}},\quad 1-x\geq e^{-2x}
        \text{ for $x\geq0$ sufficiently small.}
    \end{align*}
    Thus it suffices to show there is an assignment satisfying
    $$
        k\ln C\leq\delta_k\ln C-4\sum_{t=\ell_0+1}^{n/2}tC^{\delta_t}.
    $$
    Let $\delta_t=t/2$, then
    $$
    \text{RHS}=\frac k2\ln C-O(\ell_0C^{\ell_0/2}).
    $$
    Therefore, when $\epsilon$ (thus $C$) is fixed, choose $\ell_0$ sufficiently large such that
    $$
    \begin{cases}
        1-C^{\ell_0/2}\geq\exp\cbra{-2C^{\ell_0/2}}\\
        O(\ell_0C^{\ell_0/2})<-\frac k2\ln C,
    \end{cases}
    $$
    we are guaranteed a desired binary string of length $n$.

    This should be generalized to the infinite case by some ``compactness argument'', but I'm not familiar with that.
\end{exercise}

\begin{exercise}{3}
    Without loss of generality, assume $|S(v)|=10d,\forall v\in V$. Uniformly choose color $c_v\in S(v)$ independently
    for any $v$. For any $u\in V,v\in N(u),c\in S(u)\cap S(v)$, define event $A[\{u,v\},c]$ be $c_u=c_v=c$. Then
    $$
    \Pr\sbra{A[\{u,v\},c]}=\frac1{100d^2}.
    $$
    Also, $A[\{u,v\},c]$ is mutually independent of all $A[\{u',v'\},c']$ but $\{u,v\}\cap\{u',v'\}\neq\varnothing$. 
    Therefore in the dependency digraph the maximum degree is at most $2\times10d\times d-1$.
    By Local Lemma and 
    $$
    e\times20d^2\times\frac1{100d^2}\leq1,
    $$
    there exists a valid coloring.
\end{exercise}

\begin{exercise}{4}
    The result follows directly from the following lemma.
    \begin{lemma}
        Assume $G=(V,E)$ can be decomposed into $m$ graphs $G_1,\cdots,G_m$, each of which has maximum degree $1$.
        Let $V_1,\cdots,V_n$ be $n$ pairwise disjoint subset of $V$, each of which has size at least $2^m$.
        Then $G$ has an independent set $S$ containing precisely one vertex from each $V_i$.
    \end{lemma}
    \begin{proof}
        Without loss of generality, we may assume $|V_i|=2^m,\forall i$.
        Now we prove the lemma by induction on $m$.
        \begin{itemize}
            \item $m=1$ : $G$ itself is a matching. For any $V_i=\{u_i,v_i\}$, add an edge $\{u_i,v_i\}$ to $G$ if $u_i,v_i$
                is not previously linked. Then $G$ consists of paths and cycles and the claim holds naturally.
            \item $m>1$ : Let $V_i=V_i^1\cup V_i^2,|V_i^1|=|V_i^2|=2^{m-1}$ and $G'=G_1\cup\cdots\cup G_{m-1}$.
                By induction, $G'$ has two independent sets $S_1,S_2$ satisfying $S_1\cap V_i^1=v_i^1,S_2\cap V_i^2=v_i^2$.
                Since $S_1\cap S_2=\varnothing$, $V_i':=\{v_i^1,v_i^2\},i\in[n]$ are $n$ pairwise disjoint subset of $V$ on
                graph $G_m$, which falls into the above case.
        \end{itemize}
    \end{proof}
\end{exercise}

\begin{exercise}{5}
    We first show $c<4$, i.e., if $k$ is sufficiently large, then for any $|S_k|\geq4k\ln k$ there exists a coloring
    $r$ such that $x+S,\forall x$ intersects all $k$ colors.

    Assume towards contradiction such $S_k$ does exist. Then without loss of generality we may assume $|S_k|=4k\ln k$ by dropping
    several terms in $S_k$; and assume $\min_{x\in S_k}x=0$ by shifting it to the origin. Let $M=1+\max_{x\in S_k}x$.

    Now we choose a random local coloring $s_1:[0,2M-1]\to[k]$ by picking $s_1(i)\in[k]$ independently and uniformly.
    Let $s:\Zbb\to[k]$ be repeated $s_1$, i.e., $s(i)=s_1(i\mod 2M)$.
    Then, to verify if $S_k$ fails on $s$, it suffices to verify for all $x\in[0,2M-1]$.

    Denote $A_x,x\in[0,2M-1]$ as the event $x+S_k$ intersects $<k$ colors in $s$, then
    $$
    \Pr\sbra{A_x}\leq\frac{k\times(k-1)^{4k\ln k}}{k^{4k\ln k}}=k\pbra{1-\frac1k}^{4k\ln k}\leq\frac1{k^3},
    $$
    since $x+S_k$ is contained in $[x,x+M-1]$.
    Note that $A_x$ is correlated with less than $|S_k|^2=16k^2\ln^2k$ kinds of events $A_y$, as $(x+S_k)\cap(y+S_k)$ 
    (after modulo $2M$) must be non-empty. 
    Thus by Local Lemma, 
    $$
    e\times16k^2\ln^2k\times\frac1{k^3}<1
    $$
    with sufficiently large $k$, we show there exists coloring $s$ where $S_k$ fails.

    % upper bound
\end{exercise}

\begin{exercise}{6}
    The following algorithm runs in $O(m+(d+1)\mathrm{TLOG})$.

    \begin{algorithm}[H]
        \caption{Fix-it Algorithm}
        \DontPrintSemicolon
        Initialize $C[v]$\;
        $\mathrm{Inset}[\beta]\gets1,S\gets\{\beta\}_\beta$\;
        \While{$S\neq\varnothing$}{
            Select $\alpha\in S$\;
            \If{$\mathrm{BAD}[\alpha]$}{
                Reset $C[v],v\in A[\alpha]$\;
                \ForEach{$\beta\sim\alpha$}{
                    \If{$\mathrm{Inset}[\beta]=0$}{
                        $\mathrm{Inset}[\beta]\gets1$\;
                        $S\gets S\cup\{\beta\}$\;
                    }                
                }
            }\Else{
                $S\gets S\backslash\{\alpha\}$\;
                $\mathrm{Inset}[\alpha]\gets0$\;
            }
        }
    \end{algorithm}
\end{exercise}

\begin{exercise}{7}
    Let $A[i],i\in[30m]$ be the event that the $i$-th clause is unsatisfied. Then under a random assignment we have
    $$
    \Pr\sbra{A[i]}=2^{-10}.
    $$
    Since each variable appears in $30$ clauses, the maximum degree of dependency graph is at most $300-1=299$.
    By Local Lemma and 
    $$
    e\times(299+1)\times2^{-10}<1,
    $$
    there exists an assignment that no bad events occur.
\end{exercise}

\begin{exercise}{8}
    It is easy to see for any prefix of $D_M$, the number of $+$ in it is strictly larger than the number of $-$ in it.
    Thus the number of possibilities of $D_M$ is bounded by the Catalan number
    $$
    \frac1M\binom{2M-2}{M-1}=\Theta\pbra{\frac{4^M}{M^{1.5}}},
    $$
    thus the possibility that the algorithm does not terminate in $M$ iterations is 
    $$
    \leq O\pbra{\frac{4^M}{M^{1.5}}\times4^{-M}}=o(1).
    $$
\end{exercise}
\noindent\textbf{\it Remark:} 
I don't know how to solve the second part of the exercise, and I posted this problem on the website 
\url{https://math.stackexchange.com/questions/3115038/expected-iterations-of-finding-a-non-repetitive-sequence}.

\end{document}
