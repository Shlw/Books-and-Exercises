% !TeX encoding = UTF-8
% !TeX program = XeLaTeX
% !TeX spellcheck = LaTeX

\documentclass[a4paper]{article}

\usepackage{amsmath,amsfonts,amssymb}
\usepackage{mathrsfs}
\usepackage{bm}
\usepackage{extarrows}
\usepackage{geometry}
\usepackage{ntheorem}
\usepackage{hyperref}
\usepackage[ruled]{algorithm2e}
\usepackage{caption,subcaption}

\geometry{left=2cm,right=2cm,top=2cm,bottom=2cm}

\def\UrlBreaks{\do\A\do\B\do\C\do\D\do\E\do\F\do\G\do\H\do\I\do\J\do\K\do\L\do\M\do\N\do\O\do\P\do\Q\do\R\do\S\do\T\do\U\do\V\do\W\do\X\do\Y\do\Z\do\[\do\\\do\]\do\^\do\_\do\`\do\a\do\b\do\c\do\d\do\e\do\f\do\g\do\h\do\i\do\j\do\k\do\l\do\m\do\n\do\o\do\p\do\q\do\r\do\s\do\t\do\u\do\v\do\w\do\x\do\y\do\z\do\0\do\1\do\2\do\3\do\4\do\5\do\6\do\7\do\8\do\9\do\.\do\@\do\\\do\/\do\!\do\_\do\|\do\;\do\>\do\]\do\)\do\,\do\?\do\'\do+\do\=\do\#}

\newtheorem{theorem}{Theorem}
\newtheorem{lemma}{Lemma}
\newtheorem{proposition}{Proposition}
\newtheorem{corollary}{Corollary}
\newtheorem{claim}{Claim}
\newtheorem{conjecture}{conjecture}
\newtheorem{definition}{Definition}
\newtheorem{construction}{Construction}
\newtheorem*{proof}{Proof}
\newtheorem*{answer}{Answer}
\newtheorem*{example}{Example}
\newtheorem*{counterexample}{Counterexample}

\newenvironment{exercise}[1]{
	\par
	\noindent\textbf{Exercise #1.}\quad
}{
	\par
	\bigskip
}
\newenvironment{problem}[1]{
	\par
	\noindent\textbf{Problem #1.}\quad
}{
	\par
	\bigskip
}

\DeclareMathAccent{\widehat}{\mathord}{largesymbols}{"62}
\DeclareMathOperator*{\argmax}{\arg\,\max}
\DeclareMathOperator*{\argmin}{\arg\,\min}
\DeclareMathOperator{\E}{\mathbb E}
\DeclareMathOperator{\Var}{\mathrm{Var}}
\DeclareMathOperator{\tr}{\mathrm{tr}}
\DeclareMathOperator{\poly}{\mathrm{poly}}
\newcommand{\abs}[1]{\left| #1 \right|}
\newcommand{\vabs}[1]{\left\| #1 \right\|}
\newcommand{\abra}[1]{\left\langle #1 \right\rangle}
\newcommand{\pbra}[1]{\left( #1 \right)}
\newcommand{\cbra}[1]{\left\{ #1 \right\}}
\newcommand{\sbra}[1]{\left[ #1 \right]}
\newcommand{\floorbra}[1]{\left\lfloor #1 \right\rfloor}
\newcommand{\ceilbra}[1]{\left\lceil #1 \right\rceil}
\newcommand{\bin}{\{0,1\}}
\newcommand{\ZPP}{\mathtt{ZPP}}
\newcommand{\RP}{\mathtt{RP}}
\newcommand{\coRP}{\mathtt{co}\text{-}\mathtt{RP}}
\newcommand{\per}{\text{per}}
\newcommand{\sgn}{\text{sgn}}
\newcommand{\Fbb}{\mathbb{F}}
\newcommand{\Nbb}{\mathbb{N}}
\newcommand{\Rbb}{\mathbb{R}}
\newcommand{\Zbb}{\mathbb{Z}}
\newcommand{\Acal}{\mathcal{A}}
\newcommand{\Bcal}{\mathcal{B}}
\newcommand{\Ccal}{\mathcal{C}}
\newcommand{\Fcal}{\mathcal{F}}
\newcommand{\Gcal}{\mathcal{G}}

\bibliographystyle{plainnat}

\title{Exercise Set --- Chapter $12$}
\date{}

\begin{document}

\maketitle

\begin{exercise}{1}
    Each gate is one of the following possibilities: \emph{And gate, Or gate, Negation gate}. Each gate has at most two input wires. Thus the number of binary Boolean circuits of size $s$ on $n$ variables is at most
    $$
    \prod_{i=1}^s\underbrace{3}_{\text{gate kinds}}\times\underbrace{(s+n)^2}_{\text{input wire}}
    =\pbra{\sqrt3(s+n)}^{2s}.
    $$

\end{exercise}
\noindent\textbf{\it Remark:} The bound in the exercise is not true. It can be easily refuted by $s=1$.

\begin{exercise}{2}
    Directly applying Local Lemma, we have
    $$
    e\times 2^{-10}\times 100<1.
    $$
    Thus there exists a satisfying assignment.
\end{exercise}

\begin{exercise}{3}
    We describe layers of And gates and Or gates, and use And-Or iterations to amplify the threshold gap.
    Assume input as $pn$ ones.
    \begin{itemize}
        \item The first (bottom) layer consists of $\bigwedge_{\log n}$ gates and each has uniformly random input.
            Then the probability of being one shifts
            $$
            \begin{cases}
                p\geq\frac12+\frac1{\log n}\\
                p\leq\frac12-\frac1{\log n}
            \end{cases}
            \Rightarrow
            \begin{cases}
                p'\geq\pbra{\frac12+\frac1{\log n}}^{\log n}\approx\frac{e^2}n\\
                p'\leq\pbra{\frac12-\frac1{\log n}}^{\log n}\approx\frac1{e^2n}.
            \end{cases}
            $$
        \item The second layer consists of $\bigvee_{n}$ gates and each has disjoint input from first layer.
            Then the probability of being one shifts
            $$
            \begin{cases}
                p\geq\frac{e^2}{n}\\
                p\leq\frac1{e^2n}
            \end{cases}
            \Rightarrow
            \begin{cases}
                p'\geq1-\pbra{1-\frac{e^2}{n}}^{n}\approx1-e^{-e^2}>\frac78\\
                p'\leq1-\pbra{1-\frac1{e^2n}}^{n}\approx1-e^{-\frac1{e^2}}<\frac17.
            \end{cases}
            $$
        \item The third layer consists of $\bigwedge_{\log n}$ gates and each has disjoint input from second layer.
            Then the probability of being one shifts
            $$
            \begin{cases}
                p\geq\frac78\\
                p\leq\frac17
            \end{cases}
            \Rightarrow
            \begin{cases}
                p'\geq\pbra{\frac78}^{\log n}=n^{-3+\log7}\geq n^{-0.2}\\
                p'\leq\pbra{\frac17}^{\log n}=n^{-\log 7}\leq n^{-2.8}.
            \end{cases}
            $$
        \item The fourth layer consists of $\bigvee_{n^{1.5}}$ gates and each has disjoint input from third layer.
            Then the probability of being one shifts
            $$
            \begin{cases}
                p\geq n^{-0.2}\\
                p\leq n^{-2.8}
            \end{cases}
            \Rightarrow
            \begin{cases}
                p'\geq1-\pbra{1-n^{-0.2}}^{n^{1.5}}\approx 1-e^{-n^{1.3}}\\
                p'\leq1-\pbra{1-n^{-2.8}}^{n^{1.5}}\leq n^{-1.3}.
            \end{cases}
            $$
        \item The fifth layer consists of $\bigwedge_{n}$ gates and each has disjoint input from fourth layer.
            Then the probability of being one shifts
            $$
            \begin{cases}
                p\geq 1-e^{-n^{1.3}}\\
                p\leq n^{-1.3}
            \end{cases}
            \Rightarrow
            \begin{cases}
                p'\geq\pbra{1-e^{-n^{1.3}}}^n\geq1-ne^{-n^{1.3}}>1-2^{-n}\\
                p'\leq n^{-1.3n}< 2^{-n}.
            \end{cases}
            $$
    \end{itemize}
    Thus, for any given input, the bounded-depth, polynomial-size, monotone circuit above satisfies the condition
    with probability at least $1-2^{-n}$. 
    By union bound, there exists a desired circuit coincides with all $f(\bm x)$ for 
    $$
    \pbra{\sum_{i=1}^nx_i\geq\frac n2+\frac n{\log n}}\lor
    \pbra{\sum_{i=1}^nx_i\leq\frac n2-\frac n{\log n}}.
    $$
\end{exercise}

\end{document}
