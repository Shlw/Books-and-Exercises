% !TeX encoding = UTF-8
% !TeX program = XeLaTeX
% !TeX spellcheck = LaTeX

\documentclass[a4paper]{article}

\usepackage{amsmath,amsfonts,amssymb}
\usepackage{mathrsfs}
\usepackage{bm}
\usepackage{extarrows}
\usepackage{geometry}
\usepackage{ntheorem}
\usepackage{hyperref}
\usepackage[ruled]{algorithm2e}
\usepackage{caption,subcaption}

\geometry{left=2cm,right=2cm,top=2cm,bottom=2cm}

\def\UrlBreaks{\do\A\do\B\do\C\do\D\do\E\do\F\do\G\do\H\do\I\do\J\do\K\do\L\do\M\do\N\do\O\do\P\do\Q\do\R\do\S\do\T\do\U\do\V\do\W\do\X\do\Y\do\Z\do\[\do\\\do\]\do\^\do\_\do\`\do\a\do\b\do\c\do\d\do\e\do\f\do\g\do\h\do\i\do\j\do\k\do\l\do\m\do\n\do\o\do\p\do\q\do\r\do\s\do\t\do\u\do\v\do\w\do\x\do\y\do\z\do\0\do\1\do\2\do\3\do\4\do\5\do\6\do\7\do\8\do\9\do\.\do\@\do\\\do\/\do\!\do\_\do\|\do\;\do\>\do\]\do\)\do\,\do\?\do\'\do+\do\=\do\#}

\newtheorem{theorem}{Theorem}
\newtheorem{lemma}{Lemma}
\newtheorem{proposition}{Proposition}
\newtheorem{corollary}{Corollary}
\newtheorem{claim}{Claim}
\newtheorem{conjecture}{conjecture}
\newtheorem{definition}{Definition}
\newtheorem{construction}{Construction}
\newtheorem*{proof}{Proof}
\newtheorem*{answer}{Answer}
\newtheorem*{example}{Example}
\newtheorem*{counterexample}{Counterexample}

\newenvironment{exercise}[1]{
	\par
	\noindent\textbf{Exercise #1.}\quad
}{
	\par
	\bigskip
}
\newenvironment{problem}[1]{
	\par
	\noindent\textbf{Problem #1.}\quad
}{
	\par
	\bigskip
}

\DeclareMathAccent{\widehat}{\mathord}{largesymbols}{"62}
\DeclareMathOperator*{\argmax}{\arg\,\max}
\DeclareMathOperator*{\argmin}{\arg\,\min}
\DeclareMathOperator{\E}{\mathbb E}
\DeclareMathOperator{\Var}{\mathrm{Var}}
\newcommand{\abs}[1]{\left| #1 \right|}
\newcommand{\vabs}[1]{\left\| #1 \right\|}
\newcommand{\pbra}[1]{\left( #1 \right)}
\newcommand{\cbra}[1]{\left\{ #1 \right\}}
\newcommand{\sbra}[1]{\left[ #1 \right]}
\newcommand{\floorbra}[1]{\left\lfloor #1 \right\rfloor}
\newcommand{\ceilbra}[1]{\left\lceil #1 \right\rceil}
\newcommand{\bin}{\{0,1\}}
\newcommand{\ZPP}{\mathtt{ZPP}}
\newcommand{\RP}{\mathtt{RP}}
\newcommand{\coRP}{\mathtt{co}\text{-}\mathtt{RP}}
\newcommand{\per}{\text{per}}
\newcommand{\sgn}{\text{sgn}}
\newcommand{\Fbb}{\mathbb{F}}
\newcommand{\Nbb}{\mathbb{N}}
\newcommand{\Rbb}{\mathbb{R}}
\newcommand{\Zbb}{\mathbb{Z}}
\newcommand{\Acal}{\mathcal{A}}
\newcommand{\Bcal}{\mathcal{B}}
\newcommand{\Ccal}{\mathcal{C}}
\newcommand{\Fcal}{\mathcal{F}}
\newcommand{\Gcal}{\mathcal{G}}

\bibliographystyle{plainnat}

\title{Exercise Set --- Chapter $6$}
\date{}

\begin{document}

\maketitle

\begin{exercise}{1}
    Let $f,g:2^{G}\to\bin$ be the indicator functions of the connectivity, i.e.,
    \begin{gather*}
        f(G')=\begin{cases}
            1 & G'\text{ is connected and spanning}\\
            0 & \text{otherwise},
        \end{cases}\\
        g(G')=\begin{cases}
            1 & G\backslash G'\text{ is connected and spanning}\\
            0 & \text{otherwise}.
        \end{cases}
    \end{gather*}
    Define $\mu:2^{G}\to\Rbb^+$ as $\mu(S)=2^{-|G|}$. Then $\mu$ is log-supermodular, $f$ is increasing, and $g$ is decreasing.
    By FKG inequality, we have
    $$
    \pbra{\sum_{G'\subseteq G}\mu(G')f(G')}\pbra{\sum_{G'\subseteq G}\mu(G')g(G')}\geq
    \pbra{\sum_{G'\subseteq G}\mu(G')f(G')g(G')}\pbra{\sum_{G'\subseteq G}\mu(G')},
    $$
    which means
    \begin{align*}
        P^2&=\Pr\sbra{\text{connected }G'}\times\Pr\sbra{\text{connected }G'}\\
        &=\Pr\sbra{\text{connected }G'}\times\Pr\sbra{\text{connected }G\backslash G'}\\
        &=\pbra{\sum_{G'\subseteq G}\mu(G')f(G')}\pbra{\sum_{G'\subseteq G}\mu(G')g(G')}\\
        &\geq\pbra{\sum_{G'\subseteq G}\mu(G')f(G')g(G')}\pbra{\sum_{G'\subseteq G}\mu(G')}\\
        &=\Pr\sbra{\text{connected }G'\land\text{connected }G\backslash G'}\\
        &=Q.
    \end{align*}
\end{exercise}

\begin{exercise}{2}
    Without loss of generality, assume for any $i\in[k]$, $\pbra{S\in\Fcal_i}\land\pbra{S\subseteq S'}\implies S'\in\Fcal_i$.
    Let $\Gcal_i=2^{[n]}\backslash\Fcal_i$ and $g_i:2^{[n]}\to\bin$ be the indicator function, i.e.,
    $$
    g_i(S)=[S\in\Gcal_i].
    $$
    Define $\mu:2^{[n]}\to\Rbb^+$ as $\mu(S)=2^{-n}$. Then $\mu$ is log-supermodular and $g_i$'s are decreasing.
    By FKG inequality, we have 
    $$
    \prod_{i=1}^k\pbra{\sum_{S\subseteq[n]}\mu(S)g_i(S)}\leq\pbra{\sum_{S\subseteq[n]}\mu(S)\prod_{i=1}^kg_i(S)}
    \pbra{\sum_{S\subseteq[n]}\mu(S)}^{k-1}=\abs{\bigcap_{i=1}^k\Gcal_i}2^{-n}.
    $$
    On the other hand, by pigeonhole principle $\abs{\Fcal_i}\leq2^{n-1}$ since $S$ and $[n]\backslash S$ 
    can not be simultaneously in $\Fcal_i$. Thus $\sum_{S\subseteq[n]}\mu(S)g_i(S)\geq\frac12$. Therefore, we have
    $$
    \abs{\bigcup_{i=1}^k\Fcal_i}=2^n-\abs{\bigcap_{i=1}^k\Gcal_i}\leq2^n-2^{n-k}.
    $$
\end{exercise}

\begin{exercise}{3}
    Let $n=2k$ and $f_i:2^{\binom n2}\to\bin,i\in[n]$ be the degree indicator function of the $i$-th vertex, i.e.,
    $$
    f_i(S)=\begin{cases}
        1 & \abs{S\cap\cbra{\cbra{i,j}\middle|j\neq i}}<k\\
        0 & \text{otherwise}.
    \end{cases}
    $$
    Define $\mu:2^{\binom n2}\to\Rbb^+$ as $\mu(S)=2^{-n}$. Then $\mu(S)\mu(T)=\mu(S\cap T)\mu(S\cup T)$ and $f_i$'s are 
    decreasing.
    By FKG inequality, we have
    $$
    2^{-n}=\prod_{i=1}^n\pbra{\sum_{S\subseteq[n]}\mu(S)f_i(S)}\leq\pbra{\sum_{S\subseteq[n]}\mu(S)\prod_{i=1}^nf_i(S)}
    \pbra{\sum_{S\subseteq[n]}\mu(S)}^{n-1}=\Pr\sbra{\Delta\leq k-1}.
    $$
\end{exercise}

\end{document}
