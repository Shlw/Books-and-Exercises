% !TeX encoding = UTF-8
% !TeX program = XeLaTeX
% !TeX spellcheck = LaTeX

\documentclass[a4paper]{article}

\usepackage{amsmath,amsfonts,amssymb}
\usepackage{mathrsfs}
\usepackage{bm}
\usepackage{extarrows}
\usepackage{geometry}
\usepackage{ntheorem}
\usepackage{hyperref}
\usepackage[ruled]{algorithm2e}
\usepackage{caption,subcaption}

\geometry{left=2cm,right=2cm,top=2cm,bottom=2cm}

\def\UrlBreaks{\do\A\do\B\do\C\do\D\do\E\do\F\do\G\do\H\do\I\do\J\do\K\do\L\do\M\do\N\do\O\do\P\do\Q\do\R\do\S\do\T\do\U\do\V\do\W\do\X\do\Y\do\Z\do\[\do\\\do\]\do\^\do\_\do\`\do\a\do\b\do\c\do\d\do\e\do\f\do\g\do\h\do\i\do\j\do\k\do\l\do\m\do\n\do\o\do\p\do\q\do\r\do\s\do\t\do\u\do\v\do\w\do\x\do\y\do\z\do\0\do\1\do\2\do\3\do\4\do\5\do\6\do\7\do\8\do\9\do\.\do\@\do\\\do\/\do\!\do\_\do\|\do\;\do\>\do\]\do\)\do\,\do\?\do\'\do+\do\=\do\#}

\newtheorem{theorem}{Theorem}
\newtheorem{lemma}{Lemma}
\newtheorem{proposition}{Proposition}
\newtheorem{corollary}{Corollary}
\newtheorem{claim}{Claim}
\newtheorem{conjecture}{conjecture}
\newtheorem{definition}{Definition}
\newtheorem{construction}{Construction}
\newtheorem*{proof}{Proof}
\newtheorem*{answer}{Answer}
\newtheorem*{example}{Example}
\newtheorem*{counterexample}{Counterexample}

\newenvironment{exercise}[1]{
	\par
	\noindent\textbf{Exercise #1.}\quad
}{
	\par
	\bigskip
}
\newenvironment{problem}[1]{
	\par
	\noindent\textbf{Problem #1.}\quad
}{
	\par
	\bigskip
}

\DeclareMathAccent{\widehat}{\mathord}{largesymbols}{"62}
\DeclareMathOperator*{\argmax}{\arg\,\max}
\DeclareMathOperator*{\argmin}{\arg\,\min}
\DeclareMathOperator{\E}{\mathbb E}
\DeclareMathOperator{\Var}{\mathrm{Var}}
\DeclareMathOperator{\tr}{\mathrm{tr}}
\newcommand{\abs}[1]{\left| #1 \right|}
\newcommand{\vabs}[1]{\left\| #1 \right\|}
\newcommand{\pbra}[1]{\left( #1 \right)}
\newcommand{\cbra}[1]{\left\{ #1 \right\}}
\newcommand{\sbra}[1]{\left[ #1 \right]}
\newcommand{\floorbra}[1]{\left\lfloor #1 \right\rfloor}
\newcommand{\ceilbra}[1]{\left\lceil #1 \right\rceil}
\newcommand{\bin}{\{0,1\}}
\newcommand{\ZPP}{\mathtt{ZPP}}
\newcommand{\RP}{\mathtt{RP}}
\newcommand{\coRP}{\mathtt{co}\text{-}\mathtt{RP}}
\newcommand{\per}{\text{per}}
\newcommand{\sgn}{\text{sgn}}
\newcommand{\Fbb}{\mathbb{F}}
\newcommand{\Nbb}{\mathbb{N}}
\newcommand{\Rbb}{\mathbb{R}}
\newcommand{\Zbb}{\mathbb{Z}}
\newcommand{\Acal}{\mathcal{A}}
\newcommand{\Bcal}{\mathcal{B}}
\newcommand{\Ccal}{\mathcal{C}}
\newcommand{\Fcal}{\mathcal{F}}
\newcommand{\Gcal}{\mathcal{G}}

\bibliographystyle{plainnat}

\title{Exercise Set --- Chapter $9$}
\date{}

\begin{document}

\maketitle

\begin{exercise}{1}
    \textbf{Remark: This is not the complete solution.}

    Let $\pi_1,\pi_2,\pi_3$ be three independent random permutations in $S_n$.
    Construct a bipartite graph $G=(V_1,V_2,E)$, where $V_1=V_2=[n]$ and 
    $$
    E=\cbra{(i,\pi_1(n)),(i,\pi_2(n)),(i,\pi_3(n))\middle|i\in[n]}.
    $$
    Let $S\subseteq V_1,T\subseteq V_2,|S|+|T|\leq n$ and $0<\delta<1/2$ be a constant to be determined later, then 
    \begin{itemize}
        \item $\max\cbra{|S|,|T|}\geq(\delta+1/2)n$. Then 
            $$
            \frac{|N(S\cup T)|}{|S|+|T|}\geq\frac{N(S)-|T|}{|S|+|T|}\geq\frac{|S|-|T|}{|S|+|T|}\geq2\delta.
            $$
        \item $\max\cbra{|S|,|T|}<\kappa n$. Let $0<\lambda<1/2$ be a constant to be determined later.
            \begin{align*}
                &\Pr\sbra{\exists S,T,\frac{|N(S\cup T)|}{|S|+|T|}<\lambda\middle|\max\cbra{|S|,|T|}<\kappa n}\\
                \leq&2\sum_{|T|\leq|S|<\kappa n}\Pr\sbra{\frac{|N(S\cup T)|}{|S|+|T|}<\lambda}\\
                \leq&2\sum_{|T|\leq|S|<\kappa n}\Pr\sbra{|N(S)|\leq(1+\lambda)|T|+\lambda|S|}\\
                \leq&2\sum_{|S|<\kappa n}\Pr\sbra{|N(S)|\leq(1+2\lambda)|S|}\\
                \leq&2\sum_{i=1}^{\kappa n}\binom ni\binom n{(1+2\lambda)i}\pbra{\frac{(1+2\lambda)i}n}^{3i}.
            \end{align*}
            It can be shown, with $\lambda,\kappa$ sufficiently small, say $\kappa=1/10$, the probability is less than $1$ as $n$ goes to infinity.
        \item $\kappa n\leq\max\cbra{|S|,|T|}<(\delta+1/2)n$. I don't know.
    \end{itemize}
\end{exercise}

\begin{exercise}{2}
    Since $G$ is an $(n,d,\lambda)$-graph, for any $B\subseteq V$ of size $bn$, we have
    $$
    \sum_{v\in V}\abs{|N_B(v)|-bd}^2\leq\lambda^2b(1-b)n.
    $$
    Let $B_c$ be the set of vertices with color $c$,
    then
    \begin{align*}
        \#\cbra{v\middle|N(v)\text{ has $<k$ colors}}
        \leq\sum_{c=1}^k\#\cbra{v\middle|N_{B_c}(v)=\emptyset}
        \leq\sum_{c=1}^k\frac{\lambda^2\frac1k\pbra{1-\frac1k}n}{\frac{d^2}{k^2}}<n.
    \end{align*}
\end{exercise}

\begin{exercise}{3}
    For any fixed $Y,|Y|=5a$.
    Let $X\subseteq V,|X|\leq a$ be any largest subset such that $|N_Y(X)|<2|X|$.
    Then consider any $Z\subset V,Z\cap\pbra{X\cup Y}=\emptyset,|Z|\leq a$.
    \begin{itemize}
        \item $|Z|>|X|$. By the choice of $X$, we have $|N_Y(Z)|\geq2|Z|$.
        \item $|Z|\leq|X|$. Then $X\neq\emptyset$ and $|Z\cup X|>|X|$. 
            \begin{itemize}
                \item $|Z\cup X|\leq a$.
                    Thus by the choice of $X$,
                    $$
                    |N_Y(Z\cup X)|\geq2|Z\cup X|=2|Z|+2|X|.
                    $$
                    Since 
                    $$
                    |N_Y(Z\cup X)|\leq|N_Y(Z)|+|N_Y(X)|<|N_Y(Z)|+2|X|,
                    $$
                    the claim holds as well.
                \item $|Z\cup X|>a$.
                    By assumption, $N_Y(Z\cup X)\neq\emptyset$ and we can repeatedly remove $N_Y(Z\cup X)$ from $Y$
                    until $|Y|\leq a$. Thus
                    $$
                    |N_Y(Z\cup X)|\geq4a.
                    $$
                    On the other hand, $|Z\cup X|\leq 2a$, thus
                    $$
                    |N_Y(Z\cup X)|\leq|N_Y(Z)|+|N_Y(X)|<|N_Y(Z)|+2x\leq|N_Y(Z)|+2(2a-|Z|)
                    $$
                    and then the claim holds as well.
            \end{itemize}
    \end{itemize}
\end{exercise}

\begin{exercise}{4}
    Observe that
    $$
    M=\frac13\sum_{u\in G}|e(N(u))|.
    $$
    Since $G$ is an $(n,n/2,2\sqrt n)$-graph, we have $|N(u)|=n/2$ and
    $$
    \abs{|e(N(u))|-\frac12\times\frac n2\times\pbra{\frac12}^2\times n}\leq\frac12\times2\sqrt n\times\frac n2.
    $$
    Thus 
    $$
    |e(N(u))-\frac{n^2}{16}|=o(n^2)
    $$
    and then
    $$
    \abs{M-\frac{n^3}{48}}=o(n^3),
    $$
    which is equivalent to the original claim.
\end{exercise}

\begin{exercise}{8}
    Here we restate the theorem for convenience.
    \begin{theorem}
        Assume $0<p<1$. Let $G=G_n$ be a sequence of graphs on $n$ vertices with $p\binom n2(1+o(1))$ edges. 
        Assume further that all but $o(n)$ vertices have degree $pn(1+o(1))$. Then the following $6$ properties are 
        equivalent.
        \begin{itemize}
            \item \textbf{Property} $P^p_1$: For every graph $H(s,e)$ on $s$ vertices with $e$ edges
                $$
                N_G^*(H(s,e))=(1+o(1))n^sp^e(1-p)^{\binom s2-e}.
                $$
            \item \textbf{Property} $P^p_2$: For the cycle $C(4)$ with $4$ vertices, $N_G(C(4))\leq(1+o(1))p^4n^4$. 
            \item \textbf{Property} $P^p_3$: $\abs{\lambda_2}=o(n)$.
            \item \textbf{Property} $P^p_4$: For every set $S$ of vertices of $G$, $e(S)=\frac p2|S|^2+o(n^2)$.
            \item \textbf{Property} $P^p_5$: For every two sets of vertices $B,C$, $e(B,C)=p|B||C|+o(n^2)$.
            \item \textbf{Property} $P^p_6$: $\sum_{u,v\in V}\abs{|N(u)\cap N(v)|-p^2n}=o(n^3)$.
        \end{itemize}
    \end{theorem}
    \begin{proof}[$P_1^p\implies P^p_2$]
        Since $N_G(H)=\sum_{H\subseteq L}N_G^*(L)$, where $L$ is a graph on $|V(H)|$ vertices,
        we have
        \begin{align*}
            N_G(C(4))&=N_G^*(H(4,4))+2N_G^*(H(4,5))+N_G^*(H(4,6))\\
            &=(1+o(1))p^4n^4\sbra{(1-p)^2+2p(1-p)+p^2}\\
            &=(1+o(1))p^4n^4.
        \end{align*}
    \end{proof}
    \begin{proof}[$P_2^p\implies P^p_3$]
        Let $A$ be the adjacency matrix of $G$.
        By Perron-Frobenius Theorem, $\lambda_1>0$; thus
        $$
        \lambda_1=\sup_{x\in\Rbb^n}\frac{x^TAx}{x^Tx}\geq\frac{\bm 1^TA\bm 1}{\bm 1^T\bm 1}=pn(1+o(1)).
        $$
        On the other hand,
        $$
        (1+o(1))p^4n^4\geq N_G(C(4))=\tr\pbra{A^4}-o(n^4)=\sum_{i=1}^n\lambda_i^4-o(n^4)\geq p^4n^4(1+o(1))+\lambda_2^4-o(n^4),
        $$
        thus $|\lambda_2|=o(n)$.
    \end{proof}
\end{exercise}

\end{document}
